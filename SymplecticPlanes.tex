\documentclass{amsart}

\usepackage{amsmath,amssymb,amsfonts,amscd}
\usepackage[all]{xy}
\usepackage{%appendix,listings,
hyperref}

\DeclareMathOperator{\rank}{rank}
\DeclareMathOperator{\trace}{tr}
\DeclareMathOperator{\Tor}{Tor}
\DeclareMathOperator{\Ext}{Ext}
\DeclareMathOperator{\Aut}{Aut}
\DeclareMathOperator{\Sp}{Sp}
\DeclareMathOperator{\GL}{GL}
\DeclareMathOperator{\End}{End}
\DeclareMathOperator{\id}{id}
\DeclareMathOperator{\Hom}{Hom}
\DeclareMathOperator{\im}{Im}
\DeclareMathOperator{\Ker}{Ker}
\DeclareMathOperator{\Sym}{Sym}
\DeclareMathOperator{\Hilb}{Hilb}
\DeclareMathOperator{\ch}{ch}
\DeclareMathOperator{\rk}{rk}
\DeclareMathOperator{\ad}{ad}
\DeclareMathOperator{\td}{td}
\DeclareMathOperator{\supp}{supp}


\newcommand{\hilb}[1]{^{[#1]}}
\newcommand{\ie}{{\it i.e. }}
\newcommand{\eg}{{\it e.g. }}
\newcommand{\loccit}{{\it loc. cit. }}
\newcommand{\vac}{|0\rangle}
\newcommand{\odd}{{\rm{odd}}}
\newcommand{\even}{{\rm{even}}}
\newcommand{\tors}{{\rm{tors}}}


\newcommand{\coloneqq}{:=}
\newcommand{\kum}[2]{K_{ #2 }( #1 )}


%%%%%%%%%%%%%%%%%%%%%%%%%%%%%%

\newcommand{\C}{\mathbb{C}}
\renewcommand{\H}{\mathbb{H}}
\newcommand{\R}{\mathbb{R}}
\newcommand{\Q}{\mathbb{Q}}
\newcommand{\Z}{\mathbb{Z}}
\newcommand{\F}{{\mathbb{ F }_3}}

%%%%%%%%%%%%%%%%%%%%%%%%%%%%%

\newcommand{\kS}{\mathfrak{S}}

\newcommand{\km}{\mathfrak{m}}
\newcommand{\kq}{\mathfrak{q}}

\newcommand{\vect}[1]{\left( \begin{smallmatrix} #1 \end{smallmatrix} \right)}
\newcommand{\plan}[2]{\left< \vect{ #1 }, \vect{ #2 } \right>}

%%%%%%%%%%%%%%%%%%%%%%%%%%%%%%

\newcommand{\lra}{\longrightarrow}
\newcommand{\ra}{\rightarrow}

%%%%%%%%%%%%%%%%%%%%%%%%%%%%%

\theoremstyle{plain}
\newtheorem{theorem}{Theorem}[section]
\newtheorem{lemma}[theorem]{Lemma}
\newtheorem{proposition}[theorem]{Proposition}
\newtheorem{conjecture}[theorem]{Conjecture}
\newtheorem{question}[theorem]{Question}
\theoremstyle{definition}
\newtheorem{definition}[theorem]{Definition}
\newtheorem{problem}[theorem]{Problem}
\theoremstyle{remark}
\newtheorem{remark}[theorem]{Remark}
\newtheorem{example}[theorem]{Example}


%%%%%%%%%%%%%%%%%%%%%%%%%%%%%

\begin{document}

\title{Planes in symplectic vector spaces}

\author{Simon Kapfer}

\date{\today}

%\keywords{}

\maketitle
\section{Symplectic linear algebra}
Let $V$ be a symplectic vector space of dimension $n\in 2\mathbb{N}$ over a field $F$ with a nondegenerate symplectic form $\omega : \Lambda^2 V \rightarrow F$. A line is a one-dimensional subspace ov $V$, a plane is a two-dimensional subspace of $V$. A plane $P\subset V$ is called isotropic, if $\omega (x,y)=0$ for any $x,y\in P$, otherwise non-isotropic.  The symplectic group $\Sp V$ is the set of all linear maps $\phi : V\rightarrow V$ with the property $\omega(\phi(x),\phi(y)) = \omega(x,y)$ for all $ x,y\in V$.
\begin{proposition}
The symplectic group $\Sp V$ acts transitively on the set of non-isotropic planes as well as on the set of isotropic planes.
\end{proposition}
\begin{proof}
Given two planes $P_1$ and $P_2$, we may choose vectors $v_1,v_2,w_1,w_2$ such that $v_1,v_2$ span $P_1$, $w_1,w_2$ span $P_2$ and $\omega(u_1,u_2) =\omega(w_1,w_2)$. We complete $\{v_1,v_2\}$ as well as $\{w_1,w_2\}$ to a symplectic basis of $V$.
Then define $\phi(v_1)=w_1$ and $\phi(v_2)=w_2$. 
It is now easy to see that the definition of $\phi$ can be extended to the remaining basis elements to give a symplectic morphism.
\end{proof}
\begin{remark}
The set of planes in $V$ can be identified with the simple tensors in $\Lambda^2V$ up to multiples. Indeed, given a simple tensor $v\wedge w \in \Lambda^2 V$, the span of $v$ and $w$ yields the corresponding plane. Conversely, any two spanning vectors $v$ and $w$ of a plane give the same element $v\wedge w$ (up to multiples).
\end{remark}
\begin{proposition}
If $\phi\in\Sp V$ acts through multiplication of a scalar, $\phi(v) = \alpha v$, then $\alpha = \pm 1$ (this is immediate from the definition). Moreover, if $\phi(v)\wedge \phi(w) = \alpha v\wedge w$, then $\alpha=1$.
\end{proposition}
\begin{proof}
We may assume that $V$ is two-dimensional, generated by $v$ and $w$. Our condition on $\phi$ reads then $\det\phi = \alpha$. But the condition on $\phi$ being symplectic is $\det\phi = 1$, because on a two-dimensional vector space there is only one symplectic form up to scalar multiple. 
\end{proof}
\begin{remark}
 If $F$ is the field with two elements, then the set of planes in $V$ can be identified with the set $\{\{x,y,z\}\;|\;x,y,z\in V\backslash\{0\},\,x+y+z=0\}$. Observe that for such a $\{x,y,z\}$, $\omega(x,y)=\omega(x,y)=\omega(y,x)$ and this value gives the criterion for isotropy.
\end{remark}

\begin{proposition} Assume that $F$ is finite of cardinality $q$.
\begin{align}
&\text{The number of lines in $V$ is }\frac{q^n-1}{q-1}, \\
&\text{the number of planes in $V$ is }\frac{(q^n-1)(q^{n-1}-1)}{(q^2-1)(q-1)}, \\
&\text{the number of isotropic planes in $V$ is }\frac{(q^n-1)(q^{n-2}-1)}{(q^2-1)(q-1)}, \\
&\text{the number of non-isotropic planes in $V$ is }\frac{q^{n-2}(q^n-1)}{q^2-1}.
\end{align}
\end{proposition}
\begin{proof}
A line $\ell$ in $V$ is determined by a nonzero vector. There are $q^n - 1$ nonzero vectors in $V$ and $q-1$ nonzero vectors in $\ell$. A plane $P$ is determined by a line $\ell_1 \subset V$ and a unique second line $\ell_2\in V/\ell_1$. We have $\frac{q^2-1}{q-1}$ lines in $P$. The number of planes is therefore
$$
\frac{ \frac{q^n-1}{q-1} \cdot\frac{q^{n-1}-1}{q-1}}{\frac{q^2-1}{q-1} } = \frac{(q^n-1)(q^{n-1}-1)}{(q^2-1)(q-1)}.
$$
For an isotropic plane we have to choose the second line from $\ell_1^\perp/\ell_1$. This is a space of dimension $n-2$, hence the formula. The number of non-isotropic planes is the difference of the two previous numbers.
\end{proof}
\begin{conjecture}
There are $6q$ orbits of the induced action of $\Sp(4,q)$ on $\Lambda^2 \mathbb{F}_q^4$.
\end{conjecture}

\section{Symplectic vector spaces as index sets}
Assume now that $V$ is a four-dimensional vector space over $F=\mathbb F_q$. Consider the free $F$-module $F[V]$ with basis $\{X_i \,|\, i\in V\}$. It carries a natural $F$-algebra structure, given by
$X_i\cdot X_j := X_{i+j}$ with unit $1=X_0$. Let $\mathfrak m $ be the ideal generated by all elements of the form $(X_i-1)$.
Since $F[V]/\mathfrak m = F$, it is a maximal ideal.

\begin{conjecture} We define
$$
L := \left\{\sum_{i\in \ell}X_i=:S_\ell \,|\, \ell\subset V \text{ line}\right\}
$$
Then the ideal generated by $L$ has dimension $q^n - \binom{q+n-2}{n}$.
\end{conjecture}


We introduce an action of $\Sp (4,F)$ on $F[V]$ by setting $\phi(X_i) = X_{\phi(i)}$. Furthermore, the underlying additive group of $V$ acts on $F[V]$ by $v( X_i) = X_{i+v} =X_iX_v$. 

\begin{definition} We define subsets of $F[V]$:
\begin{align*}
B_N :=  & \left\{\sum_{i\in P}X_i \,|\, P\subset V \text{ non-isotropic plane}\right\}, \\
B_I := &  \left\{\sum_{i\in P}X_i \,|\, P\subset V \text{ isotropic plane}\right\}.
\end{align*}
Denote by $\left< B_\alpha \right>$ and by $(B_\alpha)$ the linear span of $B_\alpha$ and the ideal generated by $B_\alpha$, respectively. Note that $(B_\alpha) $ is the linear span of $ \{ v\cdot b \,|\, b\in B, v\in V \}$.
Further, let $D_\alpha$ be the linear span of $\{v(b) - b \,|\, b\in B, v\in V \}$. Then $D_\alpha$ is in fact an ideal, namely the product of ideals $\mathfrak m\cdot (B_\alpha)$.
\end{definition}
The following table illustrates the dimensions of these objects:
\vspace{2mm}

\begin{tabular}{c||c|c|c||c|c|c}
 $F$ & $\dim_F \left<B_N\right>$ & $\dim_F(B_N)$ & $\dim_F D_N$ & $\dim_F \left<B_I\right>$ & $\dim_F(B_I)$ & $\dim_F D_I $\\
\hline
$\mathbb F_2$ & 10 & 11 &  5 & 10 & 10 & 10 \\
$\mathbb F_3$ & 30 & 50 & 31 & ? & 45 & 25 \\
$\mathbb F_5$ &121 &355 &270 & ? & ? & 91 
\end{tabular}
\begin{example}
We give the 31 classes spanning $D_N$ over $\mathbb F_3$: 
First, define for a plane $P\subset V$
$$
S_P := \sum_{\tau \in P} X_\tau.
$$
Then $D_N$ is spanned by the classes 
$$S_{P+\tau} - S_P $$
\begin{align}
&\text{for } P=\plan{1\\0\\0\\0}{0\\1\\0\\0} \text{ and } 0\neq \tau\in P^\perp = \plan{0\\0\\1\\0}{0\\0\\0\\1} 
\\
&\text{for } P=\plan{0\\0\\1\\0}{0\\0\\0\\1}  \text{ and } 0\neq \tau \in P^\perp = \plan{1\\0\\0\\0}{0\\1\\0\\0} \setminus \vect{1\\0\\0\\0}
\\
&\text{for } P=\plan{1\\0\\0\\1}{0\\1\\2\\1} \text{ and } \tau \in \left\{ \vect{0\\1\\1\\2},\vect{1\\0\\0\\2},\vect{1\\1\\1\\1},\vect{2\\2\\2\\2} \right\}
\\
&\text{for } P=\plan{1\\0\\0\\0}{0\\1\\0\\1} \text{ and } \tau \in \left\{ \vect{0\\0\\0\\1},\vect{2\\0\\1\\2},\vect{1\\0\\2\\0},\vect{1\\0\\2\\1} \right\}
\\
&\text{for } P=\plan{1\\0\\0\\0}{0\\1\\1\\1} \text{ and } \tau \in \left\{ \vect{0\\0\\1\\1},\vect{1\\0\\0\\1} \right\}
\\
&\text{for } P=\plan{1\\0\\1\\1}{0\\1\\0\\1} \text{ and } \tau \in \left\{ \vect{0\\1\\0\\2},\vect{1\\0\\2\\2} \right\}
\\
&\text{for } P=\plan{1\\0\\1\\0}{0\\1\\0\\1} \text{ and } \tau \in \left\{ \vect{0\\1\\0\\2},\vect{1\\0\\2\\0} \right\}
\\
&\text{for } P=\plan{1\\0\\0\\0}{0\\1\\0\\2} \text{ and } \tau = \vect{1\\0\\1\\0}
\\
&\text{for } P=\plan{1\\0\\1\\1}{0\\1\\2\\2} \text{ and } \tau = \vect{1\\1\\0\\2}
\end{align}
So we get $8+7+4+4+2+2+2+1+1 = 31$ linear independent classes spanning $D_N$.
\end{example}

\begin{conjecture}
For $F=\mathbb F_q$, $\dim_F D_I = \frac{(q+2)(q^2+1)}{2}$.
\end{conjecture}


\section{Orthogonal sums}
Set $S:=\Sym^2 (\Lambda^2V)$. Take two vectors $v,w\in V$ with $\omega(v,w)=1$ and set $x:= (v\wedge w)^2\in S$. Denote $P$ the plane spanned by $v$ and $w$ and set $y:= \sum_{i\in P}X_i\in  F[V]$.
We set $Y':=y\cdot \mathfrak{m} = \{\sum_{i\in P} X_{i+j}-X_i\,|\, j\in V \} $.

We consider now the action of $\Sp V$ on $S\oplus F[V]$. 
\begin{proposition}
The elements $\phi(x)\oplus \phi(z),$ for $\phi \in \Sp V,$ $z \in (y)$ span a vector space of dimension
\begin{itemize}
\item 11, if $F=\mathbb F_2$,
\item 51, if $F=\mathbb F_3$,
\item 375, if $F=\mathbb F_5$.
\end{itemize}
\end{proposition}
%isotropic case :10,25,105

\begin{proposition}
The elements $\phi(x)\oplus \phi(y'),$ for $\phi \in \Sp V,$ $y' \in Y'$ span a vector space of dimension
\begin{itemize}
\item 10, if $F=\mathbb F_2$,
\item 50, if $F=\mathbb F_3$,
\item 289, if $F=\mathbb F_5$.
\end{itemize}
\end{proposition}
%non-isotropic: 10,50,289
%isotropic : 10,25,105
\begin{remark}
If $\omega(v,w)=0$, we would have the dimensions 10, 25, 105 instead.
\end{remark}


%\section{Application to generalized Kummer fourfolds}
%In our setting, $X$ is a generalized Kummer fourfold over a torus $A$, and $H^2(X,\Z)$ contains $\Lambda^2(H^1(A,\Z))$. If we now choose $v,w$ such that $\omega(v,w)/=0$.

%Let us look at Proposition 7.1.~ of Hassett and Tschinkel. They give us a class divisible by 3 in $\Sym H^2 \oplus \Pi'$

\end{document}
