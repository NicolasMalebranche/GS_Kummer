\documentclass{amsart}

\usepackage{amsmath,amssymb,amsfonts,amscd}
\usepackage[all]{xy}
\usepackage{appendix,listings,hyperref}

\DeclareMathOperator{\rank}{rank}
\DeclareMathOperator{\trace}{tr}
\DeclareMathOperator{\Tor}{Tor}
\DeclareMathOperator{\Ext}{Ext}
\DeclareMathOperator{\Aut}{Aut}
\DeclareMathOperator{\Sp}{Sp}
\DeclareMathOperator{\GL}{GL}
\DeclareMathOperator{\End}{End}
\DeclareMathOperator{\id}{id}
\DeclareMathOperator{\Hom}{Hom}
\DeclareMathOperator{\im}{Im}
\DeclareMathOperator{\Ker}{Ker}
\DeclareMathOperator{\Sym}{Sym}
\DeclareMathOperator{\Hilb}{Hilb}
\DeclareMathOperator{\ch}{ch}
\DeclareMathOperator{\rk}{rk}
\DeclareMathOperator{\ad}{ad}
\DeclareMathOperator{\td}{td}
\DeclareMathOperator{\supp}{supp}


\newcommand{\hilb}[1]{^{[#1]}}
\newcommand{\ie}{{\it i.e. }}
\newcommand{\eg}{{\it e.g. }}
\newcommand{\loccit}{{\it loc. cit. }}
\newcommand{\vac}{|0\rangle}
\newcommand{\odd}{{\rm{odd}}}
\newcommand{\even}{{\rm{even}}}
\newcommand{\tors}{{\rm{tors}}}


\newcommand{\coloneqq}{:=}
\newcommand{\kum}[2]{K_{ #2 }( #1 )}


%%%%%%%%%%%%%%%%%%%%%%%%%%%%%%

\newcommand{\C}{\mathbb{C}}
\renewcommand{\H}{\mathbb{H}}
\newcommand{\R}{\mathbb{R}}
\newcommand{\Q}{\mathbb{Q}}
\newcommand{\Z}{\mathbb{Z}}
\newcommand{\F}{{\mathbb{ F }_3}}

%%%%%%%%%%%%%%%%%%%%%%%%%%%%%

\newcommand{\kS}{\mathfrak{S}}

\newcommand{\km}{\mathfrak{m}}
\newcommand{\kq}{\mathfrak{q}}

%%%%%%%%%%%%%%%%%%%%%%%%%%%%%%

\newcommand{\lra}{\longrightarrow}
\newcommand{\ra}{\rightarrow}

%%%%%%%%%%%%%%%%%%%%%%%%%%%%%

\theoremstyle{plain}
\newtheorem{theorem}{Theorem}[section]
\newtheorem{lemma}[theorem]{Lemma}
\newtheorem{proposition}[theorem]{Proposition}
\newtheorem{conjecture}[theorem]{Conjecture}
\newtheorem{question}[theorem]{Question}
\theoremstyle{definition}
\newtheorem{definition}[theorem]{Definition}
\newtheorem{problem}[theorem]{Problem}
\theoremstyle{remark}
\newtheorem{remark}[theorem]{Remark}
\newtheorem{example}[theorem]{Example}


%%%%%%%%%%%%%%%%%%%%%%%%%%%%%

\begin{document}

\title{Planes in symplectic vector spaces}

\author{Simon Kapfer}

\date{\today}

%\keywords{}

\maketitle
\section{Symplectic linear algebra}
Let $V$ be a symplectic vector space of dimension $n\in 2\mathbb{N}$ over a field $F$ with a nondegenerate symplectic form $\omega : \Lambda^2 V \rightarrow F$. A line is a one-dimensional subspace ov $V$, a plane is a two-dimensional subspace of $V$. A plane $P\subset V$ is called isotropic, if $\omega (x,y)=0$ for any $x,y\in P$, otherwise non-isotropic.  The symplectic group $\Sp V$ is the set of all linear maps $\phi : V\rightarrow V$ with the property $\omega(\phi(x),\phi(y)) = \omega(x,y)$ for all $ x,y\in V$.
\begin{proposition}
The symplectic group $\Sp V$ acts transitively on the set of non-isotropic planes as well as on the set of isotropic planes.
\end{proposition}
\begin{proof}
Given two planes $P_1$ and $P_2$, we may choose vectors $v_1,v_2,w_1,w_2$ such that $v_1,v_2$ span $P_1$, $w_1,w_2$ span $P_2$ and $\omega(u_1,u_2) =\omega(w_1,w_2)$. We complete $\{v_1,v_2\}$ as well as $\{w_1,w_2\}$ to a symplectic basis of $V$.
Then define $\phi(v_1)=w_1$ and $\phi(v_2)=w_2$. 
It is now easy to see that the definition of $\phi$ can be extended to the remaining basis elements to give a symplectic morphism.
\end{proof}
\begin{remark}
The set of planes in $V$ can be identified with the simple tensors in $\Lambda^2V$ up to multiples. Indeed, given a simple tensor $v\wedge w \in \Lambda^2 V$, the span of $v$ and $w$ yields the corresponding plane. Conversely, any two spanning vectors $v$ and $w$ of a plane give the same element $v\wedge w$ (up to multiples).
\end{remark}
\begin{proposition}
If $\phi\in\Sp V$ acts through multiplication of a scalar, $\phi(v) = \alpha v$, then $\alpha = \pm 1$ (this is immediate from the definition). Moreover, if $\phi(v)\wedge \phi(w) = \alpha v\wedge w$, then $\alpha=1$.
\end{proposition}
\begin{proof}
We may assume that $V$ is two-dimensional, generated by $v$ and $w$. Our condition on $\phi$ reads then $\det\phi = \alpha$. But the condition on $\phi$ being symplectic is $\det\phi = 1$, because on a two-dimensional vector space there is only one symplectic form up to scalar multiple. 
\end{proof}
\begin{remark}
 If $F$ is the field with two elements, then the set of planes in $V$ can be identified with the set $\{\{x,y,z\}\;|\;x,y,z\in V\backslash\{0\},\,x+y+z=0\}$. Observe that for such a $\{x,y,z\}$, $\omega(x,y)=\omega(x,y)=\omega(y,x)$ and this value gives the criterion for isotropy.
\end{remark}


From now on we assume that $F$ is finite of cardinality $q$.
\begin{proposition}
\begin{align}
&\text{The number of lines in $V$ is }\frac{q^n-1}{q-1}, \\
&\text{the number of planes in $V$ is }\frac{(q^n-1)(q^{n-1}-1)}{(q^2-1)(q-1)}, \\
&\text{the number of isotropic planes in $V$ is }\frac{(q^n-1)(q^{n-2}-1)}{(q^2-1)(q-1)}, \\
&\text{the number of non-isotropic planes in $V$ is }\frac{q^{n-2}(q^n-1)}{q^2-1}.
\end{align}
\end{proposition}
\begin{proof}
A line $\ell$ in $V$ is determined by a nonzero vector. There are $q^n - 1$ nonzero vectors in $V$ and $q-1$ nonzero vectors in $\ell$. A plane $P$ is determined by a line $\ell_1 \subset V$ and a unique second line $\ell_2\in V/\ell_1$. We have $\frac{q^2-1}{q-1}$ lines in $P$. The number of planes is therefore
$$
\frac{ \frac{q^n-1}{q-1} \cdot\frac{q^{n-1}-1}{q-1}}{\frac{q^2-1}{q-1} } = \frac{(q^n-1)(q^{n-1}-1)}{(q^2-1)(q-1)}.
$$
For an isotropic plane we have to choose the second line from $\ell_1^\perp/\ell_1$. This is a space of dimension $n-2$, hence the formula. The number of non-isotropic planes is the difference of the two previous numbers.
\end{proof}
\begin{conjecture}
There are $6q$ orbits of the induced action of $\Sp(4,q)$ on $\Lambda^2 \mathbb{F}_q^4$.
\end{conjecture}

\section{Ranks of orbit spaces}
Assume now that $V$ is a four-dimensional vector space over $F=\mathbb F_q$. Consider the free $F$-module $F[V]$ with basis $\{X_i \,|\, i\in V\}$. It carries a natural $F$-algebra structure given by
$X_i\cdot X_j := X_{i+j}$ with unit $1=X_0$. Let $J$ be the ideal generated by all elements of the form $(X_i-1)$.

We introduce an action of $\Sp (4,F)$ on $F[V]$ by setting $\phi(X_i) = X_{\phi(i)}$. Furthermore, the underlying additive group of $V$ acts on $F[V]$ by $v( X_i) = X_{i+v} =X_vX_i$. 

\begin{definition} We define subsets of $F[V]$:
\begin{align*}
B_N :=  & \left\{\sum_{i\in P}X_i \,|\, P\subset V \text{ non-isotropic plane}\right\} \\
B_I := &  \left\{\sum_{i\in P}X_i \,|\, P\subset V \text{ isotropic plane}\right\}.
\end{align*}
Denote by $\left< B_\alpha \right>$ and by $(B_\alpha)$ the linear span of $B_\alpha$ and the ideal generated by $B_\alpha$, respectively. Note that $(B_\alpha) $ is the linear span of $ \{ v\cdot b \,|\, b\in B, v\in V \}$.
Further, let $D_\alpha$ be the linear span of $\{v(b) - b \,|\, b\in B, v\in V \}$. Then $D_\alpha$ is in fact an ideal, namely the product of ideals $J\cdot (B_\alpha)$.
\end{definition}
The following table illustrates the dimensions of these objects:
\vspace{2mm}

\begin{tabular}{c||c|c|c||c|c|c}
 $F$ & $\dim_F \left<B_N\right>$ & $\dim_F(B_N)$ & $\dim_F D_N$ & $\dim_F \left<B_I\right>$ & $\dim_F(B_I)$ & $\dim_F D_I $\\
\hline
$\mathbb F_2$ & 10 & 11 &  5 & 10 & 10 & 10 \\
$\mathbb F_3$ & 30 & 50 & 31 & 25 & 25 & 25 \\
$\mathbb F_5$ &121 &355 &270 & 91 & 91 & 91 
\end{tabular}
\begin{conjecture}
For $F=\mathbb F_q$, $\dim_F \left<B_I\right>=\dim_F(B_I)=\dim_F D_I = \frac{(q+2)(q^2+1)}{2}$.
\end{conjecture}



\end{document}
