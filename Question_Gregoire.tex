\documentclass{amsart}

\usepackage{amsmath,amssymb,amsfonts,amscd}
\usepackage[all]{xy}
\usepackage{appendix,listings,hyperref}

\DeclareMathOperator{\rank}{rank}
\DeclareMathOperator{\trace}{tr}
\DeclareMathOperator{\Tor}{Tor}
\DeclareMathOperator{\Ext}{Ext}
\DeclareMathOperator{\Aut}{Aut}
\DeclareMathOperator{\End}{End}
\DeclareMathOperator{\id}{id}
\DeclareMathOperator{\Hom}{Hom}
\DeclareMathOperator{\im}{Im}
\DeclareMathOperator{\Ker}{Ker}
\DeclareMathOperator{\Sym}{Sym}
\DeclareMathOperator{\Hilb}{Hilb}
\DeclareMathOperator{\ch}{ch}
\DeclareMathOperator{\rk}{rk}
\DeclareMathOperator{\ad}{ad}
\DeclareMathOperator{\td}{td}
\DeclareMathOperator{\supp}{supp}


\newcommand{\hilb}[1]{^{[#1]}}
\newcommand{\ie}{{\it i.e. }}
\newcommand{\eg}{{\it e.g. }}
\newcommand{\loccit}{{\it loc. cit. }}
\newcommand{\vac}{|0\rangle}
\newcommand{\odd}{{\rm{odd}}}
\newcommand{\even}{{\rm{even}}}
\newcommand{\tors}{{\rm{tors}}}

\newcommand{\p}{\mathfrak{p}}
\newcommand{\G}{\mathfrak{G}}
\newcommand{\q}{\mathfrak{q}}
\newcommand{\pone}{ \mathfrak{p}_{ - 1} }

\newcommand{\coloneqq}{:=}
\newcommand{\kum}[2]{K_{ #2 }( #1 )}


%%%%%%%%%%%%%%%%%%%%%%%%%%%%%%

\newcommand{\C}{\mathbb{C}}
\renewcommand{\H}{\mathbb{H}}
\newcommand{\R}{\mathbb{R}}
\newcommand{\Q}{\mathbb{Q}}
\newcommand{\Z}{\mathbb{Z}}


%%%%%%%%%%%%%%%%%%%%%%%%%%%%%

\newcommand{\kS}{\mathfrak{S}}

\newcommand{\km}{\mathfrak{m}}
\newcommand{\kq}{\mathfrak{q}}

%%%%%%%%%%%%%%%%%%%%%%%%%%%%%%

\newcommand{\lra}{\longrightarrow}
\newcommand{\ra}{\rightarrow}

%%%%%%%%%%%%%%%%%%%%%%%%%%%%%

\theoremstyle{plain}
\newtheorem{theorem}{Theorem}[section]
\newtheorem{lemma}[theorem]{Lemma}
\newtheorem{proposition}[theorem]{Proposition}
\newtheorem{question}[theorem]{Question}
\theoremstyle{definition}
\newtheorem{definition}[theorem]{Definition}
\newtheorem{problem}[theorem]{Problem}
\theoremstyle{remark}
\newtheorem{remark}[theorem]{Remark}
\newtheorem{example}[theorem]{Example}


%%%%%%%%%%%%%%%%%%%%%%%%%%%%%

\begin{document}

\title{Question on odd cohomology of $A\hilb{2}$}

\author{Simon Kapfer}

\date{\today}

%\keywords{}

\maketitle

Let $A$ be a complex torus of dimension $2$ and $A\hilb{2}$ the Hilbert scheme of 2 points. It can be constructed as follows: Consider the direct product $A\times A$. Denote 
$$b: \widetilde{A\times A} \rightarrow A\times A $$ 
the blow-up along the diagonal $\Delta \cong A$ with exceptional divisor $E$, so we have $i: E\rightarrow \widetilde{A\times A}  $. Since the normal bundle of $\Delta$ in $A\times A$ is trivial, we have:
$$
E \cong  \Delta\times \mathbb{P}^1.
$$
The action of $\mathfrak{S}_2$ on $A\times A$ lifts to an action on $\widetilde{A\times A}$. 
We have the pushforward $i_*:H^*(E,\Z)\rightarrow H^*(\widetilde{A\times A} ,\Z) $.

The quotient by the action of $\mathfrak{S}_2$ is 
$$ \pi:\widetilde{A\times A} \rightarrow A\hilb{2}.$$ 
Now, $A\hilb{2}$ is a compact complex manifold with torsion-free cohomology.
By \cite{Menet}, we have an exact sequence
$$
0 \rightarrow \pi_*(H^k(\widetilde{A\times A,\Z})) \rightarrow H^k(A\hilb{2},\Z) \rightarrow \left(\frac{\Z}{2{Z}}\right)^{\alpha_k}\rightarrow 0.
$$
\begin{question}
What is $\alpha_k$ for $k= 1,\ldots, 8$? 
\end{question}
I think, the pushforward of a class $a\otimes 1 \in H^{k}(E,\Z)$ is given by 
$$
\pi_* i_*(a\otimes 1) = \q_2(a)\vac \in H^{k+2}(A\hilb{n},\Z)
$$
We know, that $\q_2(1)\vac$ is divisible by 2.
\begin{problem}
Let $a\in H^1(A,\Z)$ and $b\in H^3(A,\Z)$ basis elements. We can interpret $\q_2(a)\vac\in H^3(A\hilb{2},\Z)$ and $\q_2(b)\vac\in H^5(A\hilb{2},\Z)$ as classes concentrated on the exceptional divisor, that is, as elements of $\pi_* i_*H^*(E,\Z)$. By Poincar\'e duality, the intersection matrix between $H^3(A\hilb{2},\Z)$ and $H^5(A\hilb{2},\Z)$ has determinant one. Now 
$$
 \int_{A\hilb{2}}\left(\q_2(a)\vac\cdot\q_2(b)\vac \right)= 2\int_A \left(a\cdot b\right).
$$
Moreover, $\q_2(a)\vac $ and $\q_2(b)\vac$ are orthogonal to any other element of the form $\q_1(x)\q_1(y)\vac$ under the intersection pairing. It follows, that one of $\q_2(a)\vac $ and $\q_2(b)\vac$ should be divisible by 2. But which one?
\end{problem}
Maybe  \cite{Totaro} is useful to answer this question, but I don't yet understand all of it.




\bibliographystyle{amsplain}
\begin{thebibliography}{10}

\bibitem{Menet}
G.~Menet, \emph{On the integral cohomology of quotients of complex manifolds}.
\bibitem{Totaro}
B.~Totaro, \emph{The integral cohomology of the Hilbert scheme of two points}.
\end{thebibliography}
\end{document}
