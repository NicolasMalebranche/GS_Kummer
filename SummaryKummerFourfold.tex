\subsection{Conclusion on the morphism to the Hilbert scheme}
Let us summarize our results on $\theta^*$:
\begin{theorem}\label{thetaTheorem}
Let $A$ be an abelian variety and $(b_i)\subset H^2(A,\Z)$ an integral basis. Let $\theta: \kum{A}{2}\hookrightarrow A^{[3]}$ be the embedding. We also use Notation \ref{TorusClasses}.

The homomorphism $\theta^*:H^*(A\hilb{3},\Z)\rightarrow H^*(\kum{A}{2},\Z)$ of graded rings is surjective in every degree except $4$. Moreover, the image of $H^4(A\hilb{3},\Z)$ is the primitive overlattice of $\Sym^2(H^2(\kum{A}{2},\Z))$. 
The kernel of $\theta^*$ is the ideal generated by $H^1(A\hilb{3},\Z)$.
The image by $\theta^*$ of the following integral classes provide a basis of $\im\theta^*$:
\begin{center}
\begin{tabular}{c|l|l}
degree & preimage of class & alternative name  \\
\hline
0 & $\frac{1}{6} \q_1(1)^3\vac$ & 1 \\
\hline
2 &  $\frac{1}{2}\q_1(b_i) \q_1(1)^2\vac$ for $1\leq i\leq 6$ & $j(b_i)$ \\
 & $\frac{1}{2} \q_2(1)\q_1(1)\vac $  & $e$\\
\hline
3 & $\frac{1}{2}\q_1(a^*_i)\q_1(1)^2\vac$ & \\
  & $\frac{1}{2}\q_2(a_i)\q_1(1)\vac$ & \\
\hline
4 & $\q_1(b_i)\q_1(b_j)\q_1(1)\vac$ for $1\leq i\leq j\leq 6$, but $(b_i,b_j)\neq(a_1a_2,a_3a_4)$ &\\
  & $\frac{1}{2}\q_1(x)\q_1(1)^2\vac$ (instead of the missing case above)  & $Y_p$\\
  & $\frac{1}{2}\left(\q_1(b_i)^2-\q_2(b_i)\right)\q_1(1)\vac$ & \\
  & $\frac{1}{3} \q_3(1)\vac$ & $W$ \\
\hline
5 & $\q_1(a_ia_j)\q_1(a_j^*)\q_1(1) \vac$ for any choice of $j\neq i$ &\\
  & $\q_2(a^*_i)\q_1(1)\vac $ &\\
\hline
6 & $\q_1(a_i^*)\q_1(a_j^*)\q_1(1)\vac$ for $1\leq i< j\leq 4$ & \\
  & $\q_2(x)\q_1(1)\vac$ & \\
\hline
8 & $\q_1(x)^3\vac$ & top class
\end{tabular}
\end{center}
\end{theorem}
\begin{proof}
The table is established by the following results:
For degree 2, see Proposition \ref{H2Sur}. Since the Poincar\'e duality pairing on $\kum{A}{2}$ can be evaluated using projection formula (\ref{projectionFormula}), the dual classes of degree 6 are easily computed.
The odd degrees are treated by Proposition \ref{oddcohomology}. Classes of degree 4 are studied in Sections \ref{SyminH4} and \ref{integralbasisH4}. The classes are chosen in a way that they give a basis of $\Sym^{sat}$, which is possible by Corollary \ref{SymSatImage}. The condition $(b_i,b_j)\neq(a_1a_2,a_3a_4)$ is more or less arbitrary, but we had to remove one class to avoid a relation of linear dependence.

The kernel of $\theta^*$ is described by the Propositions \ref{annihilator} and \ref{Annihideal}.
\end{proof}
