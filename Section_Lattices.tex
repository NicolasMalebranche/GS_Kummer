
\section{Lattices}\label{latticeSubsection}A reference for this section is Chapter 8.2.1 of \cite{Dolgachev}. 
\begin{definition}
By a lattice $L$ we mean a free $\Z$--module of finite rank, equipped with a non--degenerate, integer--valued symmetric bilinear form, denoted by $B$ or $\left<\ ,\;\right>$. 
By a homomorphism or embedding $L\subset M$ of lattices we mean a map $:L\rightarrow M$ that preserves the bilinear forms on $L$ and $M$ respectively. It is automatically injective. We always have the injection of a lattice $L$ into its dual space $L^*\coloneqq \Hom(L,\Z)$, given by $x \mapsto \left<x,\ \right>$. A lattice is called unimodular, if this injection is an isomorphism, \ie if it is surjective. By tensoring with $\Q$, we can interpret $L$ as well as $L^*$ as a discrete subset of the $\Q$--vector space $L\otimes \Q$. Note that this gives a kind of lattice structure to $L^*$, too, but the symmetric bilinear form on $L^*$ may now take rational coefficients.


If $L\subset M$ is an embedding of lattices of the same rank, then the index $|M\DP L|$ of $L$ in $M$ is defined as the order of the finite group $M/L$.
There is a chain of embeddings $L\subset M \subset M^* \subset L^*$ with $|L^*\DP M^*| =|M\DP L| $.

The quotient $L^*/L$ is called the discriminant group. The index of $L$ in $L^*$ is called $\discr L$, the discriminant of $L$.
Choosing a basis $(x_i)_i$ of $L$, we may express $\discr L$ as the absolute value of the determinant of the so--called Gram matrix $G$ of $L$, which is defined by $G_{ij}\coloneqq \left<x_i,x_j\right>$. $L$ is unimodular, iff $\det G =\pm 1$.
\end{definition}
\begin{proposition} \label{squareDiscr}Let $M$ be a unimodular lattice. Let $L\subset M$ be a sublattice of the same rank. Then $|M\DP L|$ equals $\sqrt{\discr L}$.
\end{proposition}
\begin{proof}
Since $M$ is unimodular, $|L^*\DP M|=|L^*\DP M^*| =|M\DP L| $ and therefore $|L^*\DP L| = |L^*\DP M||M\DP L|  = |M\DP L|^2$.
\end{proof}
An embedding $L\subset M$ is called primitive, if the quotient $M/L$ is free. We denote by $L^\perp$ the orthogonal complement of $L$ within $M$. Since an orthogonal complement is always primitive, the double orthogonal complement $ L^{\perp\perp}$ is a primitively embedded overlattice of $L$. It is clear that $\discr( L^{\perp\perp})$ divides $\discr L$. 
\begin{proposition}\label{TorsionQuotient} Let $L\subset M$ be an embedding of lattices. Then the order of the torsion part of $M/L$ 
divides $\discr L$.
\end{proposition}
\begin{proof}
The torsion part is the index of $M/( L^{\perp\perp})$ in $M/L$. But this is equal to $|L^{\perp\perp}\DP L| = |L^* \DP (L^{\perp\perp})^*|$ and clearly divides $|L^*\DP L|$.
\end{proof}
\begin{proposition}\label{discrOrthPrim}
Let $M$ be unimodular. Let $L\subset M$ be a primitive embedding. Then $\discr L = \discr L^\perp$.
\end{proposition}
\begin{proof}
Consider the orthogonal projection $ : M\otimes\Q \rightarrow L\otimes \Q$. Its restriction to $M$ has kernel equal to $L^\perp$ and image in $L^*$. Hence we have an embedding of lattices $M/L^\perp \subset L^*$. Quotienting by $L$, we get an injective map $: M/(L\oplus L^\perp) \rightarrow L^*/L$. 
Now by Proposition~\ref{squareDiscr}, $\sqrt{\discr(L) \discr (L^\perp)} =|M \DP (L\oplus L^\perp)| \leq |L^*\DP L| = \discr L$. So we get $\discr L^\perp \leq \discr L$. Exchanging the roles of $L=L^{\perp\perp}$ and $L^\perp$ gives the inequality in the opposite direction.
\end{proof}
\begin{corollary}\label{latticeCor}
Let $L\subset M$ be an embedding of lattices with unimodular $M$. Let $n$ be the order of the torsion part of $M/L$. Then $\discr L^\perp =\discr L^{\perp\perp} = \frac{1}{n^2}\discr L$.
\end{corollary}
The lattice $L$ is called odd, if there exists a $v\in L$, such that $B(v,v)$ is odd, otherwise it is called even. 
If the map $v \mapsto B(v,v)$ takes both negative and positive values on $L$, the lattice is called indefinite. 


\begin{example}
Given a free $\Z$-module $L$ with the structure of a commutative ring and a linear form $I:V\rightarrow \Z$, the setting $\left<v,w\right>=I(vw)$ defines a bilinear form giving $L$ the structure of a lattice if it is non-degenerate. This is the case in topology:
For a compact complex manifold $X$ of dimension $d$ Poincar\'e duality induces a non-degenerate bilinear form on $H^d(X,\Z)$: $$\left<\alpha,\beta\right> = \int_X\alpha\beta.$$ 
This unimodular lattice will be referred to as the Poincar\'e lattice.
\end{example}