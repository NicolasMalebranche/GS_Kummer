\section{Symplectic involution on $K_{2}(A)$}\label{Involution} 
Let $X$ be an irreducible symplectic manifold. Let $$\nu: \Aut (X)\rightarrow H^{2}(X,\Z)$$
the natural morphism. Hassett and Tschinkel (Theorem 2.1 in \cite{Hassett}) have shown that $\Ker \nu$ is a deformation invariant. 
Let $X$ be an irreducible symplectic fourfold of Kummer type. Then Oguiso in \cite{Oguiso} has shown that $\Ker \nu =(\Z/3\Z)^3\rtimes\Z/2\Z$.

Let $A$ be an abelian variety and $g$ an automorphism of $A$. Let denote by $T_{A[3]}$ the group of translation of $A$ by elements of $A[3]$. If $g\in  T_{A[3]}\rtimes\Aut_{\Z} (X)$ then $g$ induces a natural automorphism on $K_2(A)$. We denote the induced automorphism by $g^{[[3]]}$. If there is no ambiguity, we also denote the \emph{induced automorphism} by the same letter $g$.

When $X=K_2(A)$, we have more precisely, by Corollary 3.3 of \cite{BNS2}, $$\Ker \nu = T_{A[3]}\rtimes(-\id_A)^{[[3]]}.$$ 
\subsection{Uniqueness and fixed locus}
\begin{thm}\label{SymplecticInvo}
Let $X$ be an irreducible symplectic fourfold of Kummer type and $\iota$ a symplectic involution on $X$ then:
\begin{itemize}
\item[(1)]
We have $\iota\in \Ker \nu$.
\item[(2)]
Let $A$ be an abelian surface then 
the couple $(X,\iota)$ is deformation equivalent to $(K_2(A),t_\tau \circ (-\id_A)^{[[3]]})$,
where $t_\tau$ is the morphism induced on $K_2(A)$ by the translation by $\tau\in A[3]$.
\item[(3)]
The fixed locus of $\iota$ is given by a K3 surface and 36 isolated points.
\end{itemize}
\end{thm}
\begin{proof}
\begin{itemize}
\item[(1)]
If $\iota\notin \Ker \nu$, by Section 5 of \cite{MongWanTari}, the unique possible action of $\iota$ on $H^{2}(X,\Z)$ is given by $H^{2}(X,\Z)^{\iota}=U\oplus A_1^2\oplus(-6)$. We will show that it is impossible. Let assume that $H^{2}(X,\Z)^{\iota}=U\oplus A_1^2\oplus(-6)$, we will find a contradiction.

As done in Section 3 of \cite{Mongardi}, consider a local universal deformation space of $X$:
$$\Phi:\mathcal{X}\rightarrow \Delta,$$
where $\Delta$ is a small polydisk and $\mathcal{X}_0=X$. By eventually restricting $\Delta$, we can assume that $\iota$ extends to an automorphisms
$M$ on $\mathcal{X}$ and $m$ on $\Delta$ such that, we have the following commutative diagram: 
$$\xymatrix{
 \mathcal{X}\ar[d]\ar[r]^{M} & \mathcal{X}\ar[d]\\
  \Delta \ar[r]^{m} & \Delta
   }$$
Moreover, the differential of $m$ at 0 is given by the action of $\iota$ on $H^1(T_X)$ which is the same as the action on $H^{1,1}(X)$ since the symplectic holomorphic 2-form induces an isomorphism between those two and the symplectic holomorphic 2-form is preserved by the action of $\iota$. The morphism $m$ is linearizable, then $\Delta^m$ is smooth and $\dim \Delta^m=\rk H^{2}(X,\Z)^{\iota}-2=3$.  
Moreover, by \cite{Markmanou} that we can find $x\in \Delta^m$ such that %$\NS (\mathcal{X}_x)=A_1(-1)^2\oplus(-6)$. 
$\mathcal{X}_x$ is bimeromorphic to a Kummer fourfold $K_2(A)$.
%with $\NS(A)=A_1(-1)^2$ and 
Since $H^{2}(X,\Z)^{\iota}=U\oplus A_1^2\oplus(-6)$, $\iota_x:= M_{\mathcal{X}_x}$ induces a bimeromorphic involution $i$ on $K_2(A)$ with $H^{2}(K_2(A),\Z)^{i}=U\oplus A_1^2\oplus(-6)$.

Necessary, we have $\NS(K_2(A))\supset  A_1(-1)^2\oplus(-6)$. It follows that $\NS(A)\supset A_1(-1)^2$.
Let consider the involution $g$ defined by $-\id$ on $A_1(-1)^2$ and $\id$ on $(A_1(-1)^2)^{\bot}$. By Corollary 1.5.2 of \cite{Lattice}, $g$ can be extend to an involution on $H^{2}(A,\Z)$. Then by Theorem 1 of \cite{Shioda}, $g$ provides a symplectic automorphism on $A$ with: $H^{2}(A,\Z)^{g}=(A_1(-1)^2)^\bot=U\oplus A_1^2$.
It follows from classification of Section 4 of \cite{MongWanTari0}, that $A=\C/\Lambda$ with $\Lambda=\left\langle (1,0),(0,1),(x,-y),(y,x)\right\rangle$, $(x,y)\in \C^2\smallsetminus \R^2$ and $g=\left(
\begin{array}{cc}
0 & -1\\
1 & 0 
\end{array} \right)$. 

Let also denote $g$ the automorphism on $K_2(A)$ induces by $g$. By construction, $g\circ i$ acts trivially on $H^{2}(K_2(A),\Z)$. Hence by Corollary 3.3 and Lemma 3.4 of \cite{FujikiK}, $g\circ \iota$ extends to an automorphism of $K_2(A)$. In particular, $i$ extends to a symplectic involution on $K_2(A)$. Then $g\circ i\in \Ker \nu$.

By Corollary \ref{actionH3}, $t_\tau$ acts trivially on $H^{3}(K_2(A),\Z)$. Hence by Corollary 3.3 of \cite{BNS2}, we have necessary:
$$g^{*}_{|H^{3}(K_2(A),\Z)}=i^{*}_{|H^{3}(K_2(A),\Z)}\circ (-\id_A)^{*}_{|H^{3}(K_2(A),\Z)}\ \ou\ g^{*}_{|H^{3}(K_2(A),\Z)}=i^{*}_{|H^{3}(K_2(A),\Z)}.$$
But $g^{*}_{|H^{3}(K_2(A),\Z)}$ has order 4 and $i^{*}_{|H^{3}(K_2(A),\Z)}\circ (-\id_A)^{*}_{|H^{3}(K_2(A),\Z)}$ and $i^{*}_{|H^{3}(K_2(A),\Z)}$ have order 2, which is a contradiction.
\item[(2)]
It follows from (1), Theorem 2.1 of \cite{Hassett} and Corollary 3.3 of \cite{BNS2}.
\item[(3)]
It follows from (2) and Section 1.2.1 of \cite{Tari}.
\end{itemize}
\end{proof}
\begin{rmk}\label{RemarkSymplecticInv}
\begin{itemize}
\item[(1)]
We also remark that the K3 surface fixed by $(t_\tau \circ (-\id_A))$ is given by the sub-manifold $Z_\tau$ defined in Section 4 of \cite{Hassett}. 
\item[(2)]
Considering the involution $-\id_A$,
the set $$\mathcal{P}:=\left\{\left.\xi\in K_{2}(A)\right|\ \Supp \xi= \left\{a_{1},a_{2},a_{3}\right\},\ a_{i}\in A[2]\smallsetminus \left\{0\right\}, 1\leq i\leq 3 \right\}$$ provides 35 fixed points and the vertex of $$W_{0}:=\left\{\left.\xi\in K_{2}(A)\right|\ \Supp \xi=\left\{0\right\}\right\}$$ supplies the 36th point. We denote by $p_1,...,p_{35}$ the points of $\mathcal{P}$ and by $p_{36}$ the vertex of $W_{0}$.
\end{itemize}
\end{rmk}
\subsection{Action on the cohomology}\label{actioncoh}
From Theorem \ref{SymplecticInvo}, we can assume that $X=K_2(A)$ and $\iota=-\id_A$. To consider $t_\tau \circ (-\id_A)$ instead of  $-\id_A$ will only has the effect to exchange the role of $[Z_0]$ and $[Z_\tau]$.
Hence we do not lose any generality assuming that $\iota=-\id_A$.
\begin{prop}\label{invariants}
\begin{itemize}
\item[(1)]
The involution $\iota$ acts trivially on $H^{2}(K_{2}(A),\Q)$.
It follows $l_{2}^2(K_{2}(A))=l_{1,-}^2(K_{2}(A))=0$ and $l_{1,+}^2(K_{2}(A))=7$.
\item[(2)]
The involution $\iota$ acts as $-\id$ on $H^{3}(K_{2}(A),\Q)$.
%We have $\rk H^{3}(K_{2}(A),\Z)^{\iota}= 2$.
It follows $l_{2}^3(K_{2}(A))=l_{1,+}^3(K_{2}(A))=0$ and $l_{1,-}^3(K_{2}(A))=8$.
\item[(3)]
By Corollary \ref{Pi'}, we have:
$$H^{4}(K_{2}(A),\Q)=\Sym^{2} H^{2}(K_{2}(A),\Q)\oplus^{\bot} \Pi'\otimes\Q,$$
where $\Pi'=\left\langle Z_{\tau}-Z_{0},\ \tau\in A[3]\smallsetminus \left\{0\right\}\right\rangle$. The involution
$\iota^*$ fixes $\Sym^{2} H^{2}(K_{2}(A),\Z)$ and $\iota^*(Z_{\tau}-Z_{0})=Z_{-\tau}-Z_{0}$.
It follows that $l_{1,-}^4(K_{2}(A))=0$, $l_{1,+}^4(K_{2}(A))=28$ and $l_{2}^4(K_{2}(A))=40$.
%We have $H^{4}(K_{2}(A),\Z)^{\iota}=\Sym^{2} H^{2}(K_{2}(A),\Q)\oplus \Pi'^{\iota}\otimes\Q$, with
%$\rk \Pi'^{\iota}=40$.
\end{itemize}
\end{prop}
\begin{proof}
\begin{itemize}
\item[(1)]
It follows from Corollary 3.3 of \cite{BNS2}.
\item[(2)]
%We denote $\alpha=\tr \iota^{*}_{H^{1,1}(K_{2}(A))}$ and $\beta=\tr \iota^{*}_{H^{2,1}(K_{2}(A))}$.
%We apply Holomorphic Lefschetz-Riemann-Roch formula Theorem 1 of \cite{Camere} to the bundle $\Omega_{K_{2}(A)}$ and we find the following formula:
%$$\beta-2\alpha=-\frac{N}{4}-12K+\frac{1}{4}\sum_{\iota(S_{i})=S_{i}} c_{2}(K_{2}(A))\cdot S_{i},$$
%where $N$ is the number of isolated fixed points of $\iota$, $K$ is the number of K3 surfaces fixed by $\iota$ and $S_{i}$ are fixed surfaces by $\iota$.
%Hence, it follows from Proposition \ref{fix}:
%$$\beta-2\alpha=-\frac{36}{4}-12+\frac{1}{4}c_{2}(K_{2}(A))\cdot Z_{0}.$$
%Moreover by Proposition 4.3 of \cite{Kummer}, we have:
%$c_{2}(K_{2}(A))\cdot Z_{0}=28$, hence:
%$$\beta-2\alpha=-14.$$
%By (1), we know that $\alpha=5$. 
%It follow that $\beta=-4$.
%Since $\dim H^{2,1}(K_{2}(A))=4$, it follows that $\iota$ acts as $-\id$ on $H^{3}(K_{2}(A),\Q)$.
%Let $\iota_{0}$ be the involution on $A^{[3]}$ induced by $-\id$.
%We have $\iota_{0}^{*}(a_{i}^{(1)})=-a_{i}^{(1)}$ and $\iota_{0}^{*}(a_{\overline{i}}^{(0)})=-a_{\overline{i}}^{(0)}$.
%Moreover, we have by definition, $\iota^{*}\circ \theta^{*}=\theta^{*}\circ \iota_{0}^{*}$.
%It follows by Proposition 3.18 \ref{} that the involution $\iota$ acts as $-\id$ on $H^{3}(K_{2}(A),\Q)$.
It follows from Corollary \ref{actionH3}.
\item[(3)]

Let $\mathcal{S}$ be the over-lattice of $\Sym^{2} H^{2}(K_{2}(A),\Z)$ where we add all the classes divisible by 2 in $H^{4}(K_{2}(A),\Z)$.
We can calculate that the discriminant of $\Pi'$ is not divisible by 2. Since $H^{4}(K_{2}(A),\Z)$ is unimodular, it follows that the discriminant of $\mathcal{S}$ is also not divisible by 2.
Hence, we have:
$$H^{4}(K_{2}(A),\F)=\mathcal{S}\otimes\F\oplus \Pi'\otimes\F.$$
Moreover, we have: $$\iota^{*}(Z_{\tau}-Z_{0})=Z_{-\tau}-Z_{0},$$
for all $\tau\in A[3]\smallsetminus \left\{0\right\}$.
Hence $\Vect_{\F}(Z_{\tau}-Z_{0},Z_{-\tau}-Z_{0})$ is isomorphic to $N_{2}$ as a $\F[G]$-module.
Moreover $H^{2}(K_{2}(A),\Z)$ is invariant by the action of $\iota$, hence $\Sym^{2} H^{2}(K_{2}(A),\Z)$ and $\mathcal{S}$ is also invariant by the action of $\iota$. 
It follows that $\mathcal{S}\otimes\F=\mathcal{N}_{1}$ and $\Pi'\otimes\F=\mathcal{N}_{2}$.
Since $\rk \mathcal{S}=28$, we have $l_{1,+}^{4}+l_{1,-}^{4}=28$.
However, $\mathcal{S}$ is invariant by the action of $\iota$, it follows that $l_{1,-}^{4}=0$ and $l_{1,+}^{4}=28$.
On the other hand $\rk \Pi'=80$, it follows that $l_{2}^{4}=40$.
%Moreover, we have:
%$$\iota^{*}(Z_{\tau})=Z_{-\tau}.$$
%So 
%\begin{equation}
%\Pi'^{\iota}=\left\langle Z_{\tau}+Z_{-\tau}-2Z_{0},\ \tau\in A[3]\smallsetminus \left\{0\right\}\right\rangle.
%\label{Pi'i}
%\end{equation}
%Hence, we have $\rk \Pi'^{\iota}=\frac{\# A[3] -1}{2}=40$.
\end{itemize}
\end{proof}
\section{Application to Beauville--Bogomolov form}\label{BeauvilleForm}
\subsection{Statement of the main theorem}\label{statement}

In \cite{Nanikawa}, Namikawa propose a definition of the Beauville-Bogomolov form for some singular irreducible symplectic varieties. He Assumes that the singularities are only $\mathbb{Q}$-factorial singularities with a singular locus of codimension of $\geq 4$. Under these assumptions, he proves a local Torelli theorem. This result was completed by a generalization of the Fujiki formula by Matsushita in \cite{Mat}. 
\begin{thm}\label{Form}
Let $Z$ be a projective irreducible symplectic variety of dimension $2n$ with only $\mathbb{Q}$-factorial singularities, and $\codim\sing Z\geq 4$.
There exists a unique indivisible integral symmetric bilinear form $B_{Z}\in S^{2}(H^2(Z,\Z))^{*}$ and a unique positive constant $c_{Z}\in \mathbb{Q}$, such that for any $\alpha\in H^2(Z,\mathbb{C})$,
$$\alpha^{2n}=c_{Z}B_{Z}(\alpha,\alpha)^n. \ \ \ \ \ \ \ \ \ \ \ (1)$$
For $0\neq \omega\in H^{0}(\Omega_{U}^{2})$
$$B_{Z}(\omega+\overline{\omega},\omega+\overline{\omega})>0.\ \ \ \ \ \ \ \ \ \ \ (2)$$
Moreover the signature of $B_{Z}$ is $(3,h^{2}(Z,\mathbb{C})-3)$.

The form $B_{Z}$ is proportional to $q_{Z}$ and is called the Beauville--Bogomolov form of $Z$.
\end{thm}
\begin{proof}
The statement of the theorem in \cite{Mat} does not say that the form is integral. However, let $Z_{s}$ be a fiber of the Kuranishi family of $Z$, with the same idea as Matsushita's proof, we can see that $q_{Z}$ and $q_{Z_{s}}$ are proportional. Then, it follows using the proof of Theorem 5 a), c) of \cite{Beauville}.
\end{proof}
We can also consider its polarizeed form.
\begin{prop}\label{beauville}
Let $X$ be a projective irreducible symplectic variety of dimension $2n$ with $\codim\sing X\geq 4$.
The equality (1) of Theorem \ref{Form} implies that
$$\alpha_{1}\cdot...\cdot\alpha_{2n}=\frac{c_{X}}{(2n)!}\sum_{\sigma\in S_{2n}}B_{X}(\alpha_{\sigma(1)},\alpha_{\sigma(2)})...B_{X}(\alpha_{\sigma(2n-1)},\alpha_{\sigma(2n)}).$$
for all $\alpha_{i}\in H^{2}(X,\Z)$.
\end{prop}
These results were then generalized by Kirschner for complex spaces in \cite{Tim}. 
In \cite{Lol2} (Theorem 2.5) was propose the first concrete example of Beauville-Bogomolov lattice for a singular irreducible symplectic variety. The variety studied in \cite{Lol2} is a partial resolution of the quotient of the Hilbert scheme of 2 points on a K3 surface quotiented by a symplectic involution. The objective of this section is to provide a new example of Beauville-Bogomolov lattice replacing Hilbert schemes of 2 points on a K3 surface by generalized Kummer fourfolds. Knowing the integral basis of the cohomology group of the generalized Kummer provided in Part \ref{integralcohomology}, this calculation becomes possible. Moreover the calculation will be much more simple as in \cite{Lol2} because of the general techniques for calculating integral cohomology of quotients developed in \cite{Lol} and the new technique using monodromy developed in Lemma \ref{Ddelta}. 
The other techniques developed in \cite{Lol2} are also in \cite{Lol}, so to simplify the reading, we will only cite \cite{Lol2} in the rest of the section.



Concretely, let $X$ be an irreducible symplectic fourfold of Kummer type and $\iota$ a symplectic involution on $X$. By Theorem \ref{SymplecticInvo} the fixed locus of $\iota$ is the union of 36 points and a K3 surface $Z_0$. Then the singular locus of $K:=X/\iota$ is the union of a K3 and 36 points. The singular locus is not of codimension four. We will lift to a partial resolution of singularities,
$K'$ of $K$, obtained by blowing up the image of $Z_0$. By Section 2.3 and Lemma 1.2 of \cite{Fujiki2}, the variety $K'$ is an irreducible symplectic V-manifold which has singular locus of codimension four.


All Section \ref{BeauvilleForm} is devoted to prove the following theorem.
\begin{thm}\label{theorem}
Let $X$ be an irreducible symplectic fourfold of Kummer type and $\iota$ a symplectic involution on $X$.
Let $Z_0$ be the K3 surface which is in the fixed locus of $\iota$.
We denote $K=X/\iota$ and $K'$ the partial resolution of singularities of $K$ obtained by blowing up the image of $Z_0$.
Then the Beauville--Bogomolov lattice $H^2(K',\Z)$ is isomorphic to $U(3)^{3}\oplus\left(
\begin{array}{cc}
-5 & -4\\
-4 & -5 
\end{array} \right)$, and the Fujiki constant $C_{K'}$ is equal to $8$.
\end{thm}
The Beauville-Bogomolov form is a topological invariant, hence from Theorem \ref{SymplecticInvo} we can assume that $X$ is a generalized Kummer fourfold and $\iota=-\id_A$. As it will be useful to prove Lemma \ref{Ddelta}, we can assume even more. All generalized Kummer fourfolds are deformation equivalent, hence, we can assume that $A=E\times E$  with $$E:=\frac{\C}{\left\langle 1,e^{\frac{2i\pi}{6}}\right\rangle}.$$
\subsection{Overview on the proof of Theorem \ref{theorem}}
We first provide all the notation that we will need during the proof in Section \ref{nota}.
Then the proof is divided in the following steps:
\begin{itemize}
\item[(1)]
First Proposition \ref{invariants} and Corollary \ref{utile'} will provide the $H^4$-normality in Section \ref{H4}.
\item[(2)]
The knowledge of the element divisible by 2 in $\Sym^{2} H^{2}(K_2(A),\Z)$ from Section \ref{Middle} and the $H^4$-normality allow us to prove the $H^2$-normality in Section \ref{H2}. 
\item[(3)]
An adaptation of the $H^2$-normality (Lemma \ref{primitive2}) and several lemmas in Section \ref{basisinteK'} will provide an integral basis of $H^{2}(K',\Z)$ (Theorem \ref{fin}).
\item[(4)]
Knowing an integral basis of $H^{2}(K',\Z)$, we end the calculation of the Beauville--Bogomolov form in Section \ref{beauK'} using intersection theory and the generalized Fujiki formula (Theorem \ref{Form}).
\end{itemize}
\subsection{Notation}\label{nota}
Let $K_{2}(A)$ be a generalized Kummer fourfold endowed with the symplectic involution $\iota$ induced by $-\id_A$.
We denote by $\pi$ the quotient map $K_{2}(A)\rightarrow K_{2}(A)/\iota$.
From Theorem \ref{SymplecticInvo}, we know that the singular locus of the quotient $K_{2}(A)/\iota$ is the K3 surface, image by $\pi$ of $Z_{0}$, and 36 isolated points. We denote $\overline{Z_{0}}:=\pi(Z_{0})$. 
We consider $r':K'\rightarrow K_{2}(A)/\iota$ the blow-up of $K_{2}(A)/\iota$ in $\overline{Z_{0}}$ and we denote by $\overline{Z_{0}}'$ the exceptional divisor.
% symplectic variety (see Lemma of \cite{}) with $\Sing K_{2}(A)/\iota= Z_{0} \cup \left\{36\ pts\right\}$.
%By Section 2.3 and Lemma 1.2 of \cite{Fujiki}, $K'$ is a irreducible symplectic variety with only isolated singular points.
%The goal of this section is to calculate the Beauville--Bogomolov form of $K'$.
We also denote by $s_{1}:N_{1}\rightarrow K_{2}(A)$ the blowup of $K_{2}(A)$ in $Z_{0}$; and denote by $Z_{0}'$ the exceptional divisor in $N_{1}$. Denote by $\iota_{1}$ the involution on $N_{1}$ induced by $\iota$. We have $K'\simeq N_{1}/\iota_{1}$, and we denote $\pi_{1}:N_{1}\rightarrow K'$ the quotient map.

Consider the blowup $s_{2}:N_{2}\rightarrow N_{1}$ of $N_{1}$ in the 36 points $p_1,...,p_{36}$ fixed by $\iota_{1}$ and the blowup $\widetilde{r}:\widetilde{K}\rightarrow K'$ of $K'$ in its 36 singulars points. We denote the exceptional divisors by $E_{1},...,E_{36}$ and $D_{1},...,D_{36}$ respectively. We also denote $\widetilde{\overline{Z_{0}}}=\widetilde{r}^{*}(\overline{Z_{0}}')$ and $\widetilde{Z_{0}}=s_{2}^{*}(Z_{0}')$.
Denote $\iota_{2}$ the involution induced by $\iota$ on $N_{2}$ and $\pi_{2}:N_{2}\rightarrow N_{2}/\iota_{2}$ the quotient map. 
We have $N_{2}/\iota_{2}\simeq \widetilde{K}$. We collect this notation in commutative diagram
$$\xymatrix{
 \widetilde{K}\ar[r]^{\widetilde{r}} & K' \ar[r]^{r'}&  K_{2}(A)/\iota\\
  N_{2}\ar@(dl,dr)[]_{\iota_{2}} \ar[r]^{s_{2}} \ar[u]^{\pi_{2}}& N_{1}\ar@(dl,dr)[]_{\iota_{1}} \ar[r]^{s_{1}} \ar[u]^{\pi_{1}} & K_{2}(A)\ar@(dl,dr)[]_{\iota}\ar[u]^{\pi}
   }$$
To finish, we denote $V=K_{2}(A)\smallsetminus \Fix \iota$ and $U=V/\iota$. Also, we set $s=s_2\circ s_1$ and $r=\widetilde{r}\circ r'$. We denote also $\delta$ the half of the class of the diagonal in $H^{2}(K_2(A),\Z)$.

We can commute the push-forward maps and the blow-up maps as proved in Lemma 3.3.21 of \cite{Lol}.
\begin{rmk}\label{commut2}
Let $x\in H^{2}(N_1,\Z)$, $y\in H^{2}(K_2(A),\Z)$, we have:
$$\pi_{2*}(s_2^{*}(x))=\widetilde{r}^{*}(\pi_{1*}(x)),$$
$$\pi_{1*}(s_1^{*}(y))=r'^{*}(\pi_{*}(y)),$$
\end{rmk}
\subsection{The couple $(K_{2}(A),\iota)$ is $H^{4}$-normal}\label{H4}
\begin{prop}\label{H4norm}
The couple $(K_{2}(A),\iota)$ is $H^{4}$-normal.
\end{prop}
\begin{proof}
We apply Corollary \ref{utile'}.
\begin{itemize}
\item[i)]
By Theorem 2 of \cite{Spanier}, $H^{*}(K_{2}(A),\Z)$ is torsion-free. 
\item[ii)]
From Remark \ref{RemarkSymplecticInv} (1), we know that the connected component of dimension 2 of $\Fix \iota$  is given by $Z_{0}$. We know that $Z_{0}$ is a K3 surface, hence is simply connected. 
Moreover by Proposition 4.3 of \cite{Hassett} $Z_{0}\cdot Z_{\tau}=1$ for all $\tau\in A[3]\smallsetminus \left\{0\right\}$. Hence the class of $Z_{0}$ in $H^{4}(K_{2}(A),\Z)$ is primitive. It follows that $\Fix \iota$ is almost negligible (Definition \ref{negligible}). 
\item[iii)]
By Proposition \ref{invariants}, we have $l_{1,-}^{2}(K_{2}(A))=l_{1,-}^{4}(K_{2}(A))=0.$
\item[iv)]
By Proposition \ref{invariants}, we have $l_{1,+}^{3}(K_{2}(A))=0.$ Moreover $H^{1}(K_{2}(A))=0$, so $l_{1,+}^{1}(K_{2}(A))=0.$
\item[v)]
We have to check the following equality:
\begin{align*}
&l_{1,+}^{4}(K_{2}(A))+2\left[l_{1,-}^{1}(X)+l_{1,-}^{3}(X)+l_{1,+}^{0}(X)+l_{1,+}^{2}(X)\right]\\
&= 36h^{0}(pt)+h^{0}(Z_{0},\Z)+h^{2}(Z_{0},\Z)+h^{4}(Z_{0},\Z).
\end{align*}
By Proposition \ref{invariants}:
$$l_{1,+}^{4}(K_{2}(A))+2\left[l_{1,-}^{1}(X)+l_{1,-}^{3}(X)+l_{1,+}^{0}(X)+l_{1,+}^{2}(X)\right]=28+2(8+1+7)=60.$$
Moreover since $Z_{0}$ is a K3 surface, we have:
$$36h^{0}(pt)+h^{0}(Z_{0},\Z)+h^{2}(Z_{0},\Z)+h^{4}(Z_{0},\Z)=36+2+22=60.$$
\end{itemize}
It follows from Corollary \ref{utile'} that $(K_{2}(A),\iota)$ is $H^4$-normal.
\end{proof}
As explained in Proposition 3.5.20 of \cite{Lol}, the proof of Corollary \ref{utile'} provide first that $\pi_{2*}(s^*(H^4(K_2(A),\Z)))$ is primitive in $H^4(\widetilde{K},\Z)$ and then the $H^4$ normality.  
\begin{rmk}\label{primitive1}
The lattice $\pi_{2*}(s^*(H^4(K_2(A),\Z)))$ is primitive in $H^4(\widetilde{K},\Z)$.
\end{rmk}
%Moreover by Corollary 3.5.19 of \cite{Lol}, we also have:
%\begin{rmk}
%The group $H^4(\widetilde{K},\Z)$ is torsion-free.
%\end{rmk}
\subsection{The couple $(K_{2}(A),\iota)$ is $H^{2}$-normal}\label{H2}
\begin{prop}
The couple $(K_{2}(A),\iota)$ is $H^{2}$-normal.
\end{prop}
\begin{proof}
We want to prove that the pushforward  
$\pi_{*}:H^{2}(K_{2}(A),\Z)\rightarrow H^{2}(K_{2}(A)/\iota,\Z)/\tors$ is surjective. 
By Remark \ref{Hnormal}, it is equivalent to prove that for all $x\in H^{2}(K_{2}(A),\Z)^{\iota}$
$\pi_{*}(x)$ is divisible by 2 if and only if there exist $y\in H^{2}(K_{2}(A),\Z)$ such that $x=y+\iota^{*}(y)$. 

Let $x\in H^{2}(K_{2}(A),\Z)^{\iota}=H^{2}(K_{2}(A),\Z)$ such that $\pi_{*}(x)$ is divisible by 2, we will show that there exists $y\in H^{2}(K_{2}(A),\Z)$ such that $x=y+\iota^{*}(y)$.
Then, by Proposition \ref{commut}, $\pi_{*}(x^2)$ is divisible by 2.
However, $x^2\in H^{4}(K_{2}(A),\Z)^{\iota}$; since $(K_{2}(A),\iota)$ is $H^{4}$-normal by Proposition \ref{H4norm}, it means that there is $z\in H^{4}(K_{2}(A),\Z)$ such that
$x^2=z+\iota^{*}(z)$.

Let $\mathcal{S}$ be, as before, the over-lattice of $\Sym^{2} H^{2}(K_{2}(A),\Z)$ where we add all the classes divisible by 2 in $H^{4}(K_{2}(A),\Z)$.
By Corollary \ref{Pi'}, there exist $z_{s}\in \mathcal{S}$, $z_{p}\in \Pi'$ and $\alpha\in \mathbb{N}$ such that:
$5\cdot3^\alpha\cdot z= z_{s}+z_{p}$. 
Hence, we have:
$$5\cdot3^\alpha\cdot x^2=2z_{s}+z_{p}+\iota^{*}(z_{p}).$$
Since $x^2\in \Sym^2 H^{2}(K_{2}(A),\Z)$, by Corollary \ref{Pi'}, $z_{p}+\iota^{*}(z_{p})=0$.
It follows:
\begin{equation}
5\cdot3^\alpha\cdot x^2=2z_{s}.
\label{haha}
\end{equation}
%$$x^2=\frac{2z_{s}}{5\cdot3^\alpha}.$$
%Moreover, since $H^{4}(K_{2}(A),\Z)$ is torsion-free, $z':=\frac{z_{s}}{5\cdot3^\alpha}\in H^{4}(K_{2}(A),\Z)$. 
let $(u_{1},u_{2},v_{1},v_{2},w_{1},w_{2},\delta)$ be the integral basis of $H^{2}(K_{2}(A),\Z)$ introduced in Notation\ref{BasisH2KA}. The couples 
$(u_{1},u_{2})$, $(v_{1},v_{2})$ and $(w_{1},w_{2})$ are basis of the hyperbolic plans $U$ and $\delta$ is half the diagonal of $K_{2}(A)$.
We can write:
$$x=\alpha_{1}u_{1}+\alpha_{2}u_{2}+\beta_{1}v_{1}+\beta_{2}v_{2}+\gamma_{1}w_{1}+\gamma_{2}w_{2}+d\delta.$$
Then $$x^{2}=\alpha_{1}^{2}u_{1}^{2}+\alpha_{2}^{2}u_{2}^{2}+\beta_{1}^{2}v_{1}^{2}+\beta_{2}^{2}v_{2}^{2}+\gamma_{1}^{2}w_{1}^{2}+\gamma_{2}^{2}w_{2}^{2}+d^{2}\delta^{2}\mod 2H^{4}(K_{2}(A),\Z).$$
We also have:
$$5\cdot3^\alpha\cdot x^{2}=\alpha_{1}^{2}u_{1}^{2}+\alpha_{2}^{2}u_{2}^{2}+\beta_{1}^{2}v_{1}^{2}+\beta_{2}^{2}v_{2}^{2}+\gamma_{1}^{2}w_{1}^{2}+\gamma_{2}^{2}w_{2}^{2}+d^{2}\delta^{2}\mod 2H^{4}(K_{2}(A),\Z).$$
It follows by (\ref{haha}) that 
$\alpha_{1}^{2}u_{1}^{2}+\alpha_{2}^{2}u_{2}^{2}+\beta_{1}^{2}v_{1}^{2}+\beta_{2}^{2}v_{2}^{2}+\gamma_{1}^{2}w_{1}^{2}+\gamma_{2}^{2}w_{2}^{2}+d^{2}\delta^{2}$ is divisible by 2.
However by Corollary \ref{Classuvw} and Proposition \ref{classedivisibleSym}, we have:
$$\mathcal{S}=\left\langle \Sym^2 H^{2}(K_{2}(A),\Z); \frac{u_{1}\cdot u_{2}+v_{1}\cdot v_{2}+w_{1}\cdot w_{2}}{2};\frac{u_{i}^{2}-\frac{1}{3}u_{i}\cdot\delta}{2};\frac{v_{i}^{2}-\frac{1}{3}v_{i}\cdot\delta}{2};\frac{w_{i}^{2}-\frac{1}{3}w_{i}\cdot\delta}{2},i\in\left\{1,2\right\}\right\rangle.$$
It follows that all the coefficients of $\alpha_{1}^{2}u_{1}^{2}+\alpha_{2}^{2}u_{2}^{2}+\beta_{1}^{2}v_{1}^{2}+\beta_{2}^{2}v_{2}^{2}+\gamma_{1}^{2}w_{1}^{2}+\gamma_{2}^{2}w_{2}^{2}+d^{2}\delta^{2}$ are divisible by 2.
%Maybe use \otimes\F to a better presentation. 
It means that $x$ is divisible by 2. It is what we wanted to prove.
\end{proof}
With exactly the same proof working in $H^4(\widetilde{K},\Z)$ and using Remark \ref{primitive1}, we provide the following lemma.
\begin{lemme}\label{primitive2}
The lattice $\pi_{2*}(s^*(H^2(K_2(A),\Z)))$ is primitive in $H^2(\widetilde{K},\Z)$.
\end{lemme}
\subsection{Calculation of $H^{2}(K',\Z)$}\label{basisinteK'}
This section is devoted to prove the following theorem.
\begin{thm}\label{fin}
Let $K'$, $\pi_1$, $s_1$ and $\overline{Z_0}'$ be respectively the variety, the maps and the class defined in Section \ref{nota}. 
We have $$H^{2}(K',\Z)=\pi_{1*}(s_{1}^{*}(H^2(K_2(A),\Z)))\oplus\Z\left(\frac{\pi_{1*}(s_{1}^{*}(\delta))+\overline{Z_{0}}'}{2}\right)\oplus\Z\left(\frac{\pi_{1*}(s_{1}^{*}(\delta))-\overline{Z_{0}}'}{2}\right).$$
%where $\delta$ is half the class of the diagonal in $H^2(K_2(A),\Z)$.
\end{thm}

First we need to calculate some intersections.
\begin{lemme}\label{Fulton}
\begin{itemize}
\item[(1)]
We have $E_{l}\cdot E_{k}=0$ if $l\neq k$, $E_{l}^{4}=-1$ and $E_{l}\cdot z=0$ for all $(l,k)\in \left\{1,...,28\right\}^{2}$
and for all $z\in s^{*}(H^{2}(K_2(A),\Z))$.
\item[(2)]
We have $\delta^4=324$.
\end{itemize}
\end{lemme}
\begin{proof}
\begin{itemize}
\item[(1)]
Same proof as Proposition 4.6.16 1) of \cite{Lol}.
\item[(2)]
It follows directly from the Fujiki formula.
\end{itemize}
\end{proof}
We already have properties of primitivity.
\begin{lemme}\label{sublemma}
\begin{itemize}
\item[(1)]
$\pi_{1*}(s_1^{*}(H^{2}(K_{2}(A),\Z)))$ is primitive in $H^{2}(K',\Z)$,
\item[(2)]
The group $\widetilde{\mathcal{D}}=\left\langle \widetilde{\overline{Z_0}},D_{1},...,D_{36},\frac{\widetilde{\overline{Z_0}}+D_{1}+...+D_{36}}{2}\right\rangle$ is primitive in $H^{2}(\widetilde{K},\Z)$.
\item[(3)]
$\overline{Z_{0}}'$ is primitive in $H^{2}(K',\Z)$,
%\item[3)]
%$\pi_{*}(\delta^{2})+\overline{Z_{0}}$ is divisible by 2.
\end{itemize}
\end{lemme}
\begin{proof}
\begin{itemize}
\item[(1)]
By Lemma \ref{primitive2}, $\pi_{2*}(s^*(H^2(K_2(A),\Z)))$ is primitive in $H^{2}(\widetilde{K},\Z)$. Then by Remark \ref{commut2}, $r'^*(\pi_{*}(H^2(K_2(A),\Z)))$ is primitive in $H^{2}(K',\Z)$. Using again Remark \ref{commut2}, we get the result.
\item[(2), (3)]
See proof of Lemma 4.6.14 of \cite{Lol}.
%%If $\overline{Z_{0}}'$ is divisible by 2, by Lemma \ref{Fulton}, $r^{*}(\overline{Z_{0}})$ is divisible by 2. 
%%Since $r^{*}(\pi_{*}(H^{4}(K_{2}(A),\Z)))$ primitive in $H^{4}(K',\Z)$, it means that $Z_{0}=a+\iota^{*}(a)$ with $a\in H^{4}(K_{2}(A),\Z)$. 
%%But this is impossible since $Z_{0}\cdot \frac{u_{1}\cdot u_{2}+v_{1}\cdot v_{2}+w_{1}\cdot w_{2}}{2}=3$ is odd.
%%Hence $\overline{Z_{0}}'$ is primitive in $H^{2}(K',\Z)$.

%%Can also be proven as in proof of Lemma 2.33 of \cite{Lol2}.
%%\item[(3)]
%Now assume that there is $b\in H^{2}(K_{2}(A),\Z)$ such that $r^{*}(\pi_{*}(b))+\overline{Z_{0}}'$ is divisible by 2.
%It follows by Lemma \ref{Fulton} and Lemma 2.18 of \cite{Lol} that $r^{*}(\pi_{*}(b^2-Z_{0}))$ is divisible by 2.
%Since $r^{*}(\pi_{*}(H^{4}(K_{2}(A),\Z)))$ primitive in $H^{4}(K',\Z)$, it means that $b^2-Z_{0}=a+\iota^{*}(a)$ with $a\in H^{4}(K_{2}(A),\Z)$. 
%%Moreover by Proposition 5.1 of \cite{Kummer}, we can write $Z_{0}$ as follows:
%%$$Z_{0}=3c_{2}(K_{2}(A))-\sum_{\tau\in A[3]\smallsetminus \left\{0\right\}}Z_{\tau}.$$
%%Hence by Proposition 3.14 \ref{}:
%%$$Z_{0}=12u_{1}\cdot u_{2}+12v_{1}\cdot v_{2}+12w_{1}\cdot w_{2}-\delta^2-\sum_{\tau\in A[3]\smallsetminus \left\{0\right\}}Z_{\tau}.$$
%%It follows:
%%$$\delta^{2}+Z_{0}=12u_{1}\cdot u_{2}+12v_{1}\cdot v_{2}+12w_{1}\cdot w_{2}-\sum_{\tau\in A[3]\smallsetminus \left\{0\right\}}Z_{\tau}.$$
%It follows that:
%$$81 b^{2}-12u_{1}\cdot u_{2}+12v_{1}\cdot v_{2}+12w_{1}\cdot w_{2}+\delta^2=a'+\iota^{*}(a'),$$
%with $a'\in H^{4}(K_{2}(A),\Z)$.
%Hence we can write:
%$$b^{2}+\delta^2=a''+\iota^{*}(a''),$$
%with $a'\in H^{4}(K_{2}(A),\Z)$.
%Using the same method as in Section \ref{H2}, it proves that 
%$$b=\delta \mod 2 H^{2}(K_{2}(A),\Z).$$
%%Hence $\pi_{*}(\delta^{2}+Z_{0})$ is divisible by 2.
\end{itemize}
\end{proof}
With Lemma \ref{sublemma} (1) and (3), it only remains to prove that $\pi_{1*}(s_{1}^{*}(\delta))+\overline{Z_{0}}'$ is divisible by 2 which will be gone in Lemma \ref{dernierlemme}. To prove this lemma, we first prove that $\pi_{2*}(s^{*}(\delta))+\widetilde{\overline{Z_{0}}}$ is divisible by 2. Knowing that $\widetilde{\overline{Z_0}}+D_{1}+...+D_{36}$ is divisible by 2, we have only to show that $\pi_{2*}(s^{*}(\delta))+D_{1}+...+D_{36}$ is divisible by 2 which is done by Lemma \ref{exist} and \ref{Ddelta}.
%We have the following exact sequence:
%$$\xymatrix@C=15pt{H^{1}(U,\Z)\ar[r] &H^{2}(\widetilde{M},U,\Z)\ar[r]&H^{2}(\widetilde{M},\Z) \ar[r]&H^{2}(U,\Z) \ar[r]&H^{3}(\widetilde{M},U,\Z)
%}.$$
%By Proposition ?, we have $H^{1}(U,\Z)=0$ and $H^{2}(U,\Z)=\Z/2\Z$.
%Moreover, by Thom's isomorphism, we have $H^{3}(\widetilde{M},U,\Z)$ $\simeq H^{1}(\stackrel{2}{Z_{0}},\Z)\oplus(\oplus_{i=1}^{36}H^{1}(D_{i},\Z))=0$.
%We have also $H^{2}(\widetilde{M},U,\Z)$ $\simeq H^{0}(\stackrel{2}{Z_{0}},\Z)\oplus(\oplus_{i=1}^{36}H^{0}(D_{i},\Z))$.
%Then the exact sequence gives:
%$$H^{2}(U,\Z)\simeq H^{2}(\widetilde{M},\Z)/\left\langle \widetilde{\Sigma},D_{1},...,D_{28}\right\rangle.$$
%Let $\widetilde{\mathcal{D}}$ be the primitive overgroup of $\left\langle \stackrel{2}{Z_{0}},D_{1},...,D_{28}\right\rangle$ in $H^{2}(\widetilde{M},\Z)$. Since $H^{2}(U,\Z)=\Z/2\Z$, $\widetilde{\mathcal{D}}/\left\langle \stackrel{2}{Z_{0}},D_{1},...,D_{28}\right\rangle=\Z/2\Z$.
%But, we still know that $\stackrel{2}{Z_{0}}+D_{1}+...+D_{36}$ is divisible by $2$ by Proposition ?. So $\widetilde{\mathcal{D}}=\left\langle \widetilde{\Sigma},D_{1},...,D_{36},\frac{\stackrel{2}{Z_{0}}+D_{1}+...+D_{36}}{2}\right\rangle$.

First we need to know the group $H^{3}(\widetilde{K},\Z)$.
\begin{lemme}\label{H3}
We have $H^{3}(\widetilde{K},\Z)=0.$
\end{lemme}
\begin{proof}
We have the following exact sequence: 
$$\xymatrix@C=10pt@R=0pt{H^{3}(K_2(A),V,\Z)\ar[r] &H^{3}(K_2(A),\Z)\ar[r]^{f} &H^{3}(V,\Z)\ar[r]& 
H^{4}(K_2(A),V,\Z)\ar[r]^{\rho} &H^{4}(K_2(A),\Z).}$$
By Thom isomorphism, $H^{3}(K_2(A),V,\Z)=0$ and $H^{4}(K_2(A),V,\Z)=H^{0}(Z_0,\Z)$.
Moreover $\rho$ is injective, so $H^{3}(V,\Z)=H^{3}(K_2(A),\Z)$. 

Hence by Proposition \ref{invariants} (2) and Proposition 3.2.8 of \cite{Lol}, we find that $H^{3}(U,\Z)=0$.
Since $H^{3}(K_2(A),\Z)^\iota=0$, $H^{3}(\widetilde{K},\Z)$ is a torsion group. 
Hence the result follows from the exact sequence
$$\xymatrix@C=10pt@R=0pt{H^{3}(\widetilde{K},U,\Z)\ar[r] &H^{3}(\widetilde{K},\Z)\ar[r] &H^{3}(U,\Z)}$$
and from the fact that $H^{3}(\widetilde{K},U,\Z)=0$ by Thom isomorphism.
\end{proof}
\begin{lemme}\label{exist}
There exists $D_\delta$ which is a linear combination of the $D_i$ with coefficient 0 or 1 such that $\pi_{2*}(s^{*}(\delta))+D_\delta$ is divisible by 2.
\end{lemme}
\begin{proof}
First, we have to use Smith theory as in Section 4.6.4 of \cite{Lol}.

Look at the following exact sequence:
$$\xymatrix@C=10pt@R0pt{0\ar[r] &H^{2}(\widetilde{K},\widetilde{\overline{Z_{0}}}\cup(\cup_{k=1}^{36}D_{k}),\mathbb{F}_{2}))\ar[r]&H^{2}(\widetilde{K},\mathbb{F}_{2}) \ar[r]& H^{2}(\widetilde{\overline{Z_{0}}}\cup(\cup_{k=1}^{36}D_{k},\mathbb{F}_{2}))\\
\ar[r]&H^{3}(\widetilde{K},\widetilde{\overline{Z_{0}}}\cup(\cup_{k=1}^{36}D_{k}),\mathbb{F}_{2})\ar[r] &0.&
}$$
First, we will calculate the dimension of the vector spaces $H^{2}(\widetilde{K},\widetilde{\overline{Z_{0}}}\cup(\cup_{k=1}^{36}D_{k}),\mathbb{F}_{2})$ and $H^{3}(\widetilde{K},$ $\widetilde{\overline{Z_{0}}}\cup(\cup_{k=1}^{36}D_{k}),\mathbb{F}_{2})$.
By (2) of Proposition \ref{SmithProp}, we have 
$$H^{*}(\widetilde{K},\widetilde{Z_{0}}\cup(\cup_{k=1}^{36}D_{k}),\mathbb{F}_{2})\simeq H^{*}_{\sigma}(N_{2}).$$
%where $H^{*}_{\sigma}(N_{2})$ is the cohomology group of the complex $\sigma C_{*}(N_{2})$.

The previous exact sequence gives us the following equation:

$$h^{2}_{\sigma}(N_{2})-h^{2}(\widetilde{K},\mathbb{F}_{2})+h^{2}(\widetilde{Z_{0}}\cup(\cup_{k=1}^{36}D_{k}),\mathbb{F}_{2})-h^{3}_{\sigma}(N_{2})=0.$$
As $h^{2}(\widetilde{K},\mathbb{F}_{2})=8+36=44$ and $h^{2}(\widetilde{Z_{0}}\cup(\cup_{k=1}^{36}D_{k}),\mathbb{F}_{2})=23+36=59$, we obtain:
%h^{2}_{\sigma}(N_{2})-(16+28)+(23+28)-h^{3}_{\sigma}(N_{2})=0
$$h^{2}_{\sigma}(N_{2})-h^{3}_{\sigma}(N_{2})=-15.$$

%where $h^{*}_{\sigma}(\widetilde{X})$ denote the dimension over $\mathbb{F}_{2}$ of $H^{*}_{\sigma}(\widetilde{X})$.
Moreover by 2) of Proposition \ref{SmithProp}, we have the exact sequence
$$\xymatrix@C=10pt@R0pt{0\ar[r] &H^{1}_{\sigma}(N_{2})\ar[r]&H^{2}_{\sigma}(N_{2}) \ar[r]&H^{2}(N_{2},\mathbb{F}_{2}) \ar[r]&H^{2}_{\sigma}(N_{2})\oplus H^{2}(\widetilde{Z_{0}}\cup(\cup_{k=1}^{36}E_{k}),\mathbb{F}_{2})\\
\ar[r]&H^{3}_{\sigma}(N_{2})\ar[r]&\coker\ar[r] &0. &
}$$
By Lemma 7.4 of \cite{BNS}, $h^{1}_{\sigma}(N_{2})=h^{0}(\widetilde{Z_{0}}\cup(\cup_{k=1}^{36}E_{k}),\mathbb{F}_{2})-1$.
Then we get the equation
\begin{align*}
&h^{0}(\widetilde{Z_{0}}\cup(\cup_{k=1}^{36}E_{k}),\mathbb{F}_{2})-1-h^{2}_{\sigma}(N_{2})+h^{2}(N_{2},\mathbb{F}_{2})\\
&-h^{2}_{\sigma}(N_{2})-h^{2}(\widetilde{Z_{0}}\cup(\cup_{k=1}^{36}E_{k}),\mathbb{F}_{2})+h^{3}_{\sigma}(N_{2})-\alpha=0,\\
\end{align*}
where $\alpha=\dim \coker$.
%&28-h^{2}_{\sigma}(\widetilde{X})+(23+29)-h^{2}_{\sigma}(\widetilde{X})-(23+28)+h^{3}_{\sigma}(\widetilde{X})=0\\
So
$$21-\alpha-2h^{2}_{\sigma}(N_2)+h^{3}_{\sigma}(N_2)=0.$$
From the two equations, we deduce that
$$h^{2}_{\sigma}(N_{2})=36-\alpha,\ \ \ \ \ h^{3}_{\sigma}(N_{2})=51-\alpha.$$

Come back to the exact sequence
$$\xymatrix@C=15pt{0\ar[r] &H^{2}(\widetilde{K},\widetilde{\overline{Z_{0}}}\cup(\cup_{k=1}^{36}D_{k}),\mathbb{F}_{2})\ar[r]&\ar[r]^{\varsigma^{*}\ \ \ \ \ \ \  }H^{2}(\widetilde{K},\mathbb{F}_{2}) & H^{2}(\widetilde{\overline{Z_{0}}}\cup(\cup_{k=1}^{36}D_{k}),\mathbb{F}_{2}),
}$$
where $\varsigma:\widetilde{\overline{Z_{0}}}\cup(\cup_{k=1}^{36}D_{k})\hookrightarrow \widetilde{K}$ is the inclusion.
Since $h^{2}(\widetilde{K},\widetilde{\overline{Z_{0}}}\cup(\cup_{k=1}^{36}D_{k}),\mathbb{F}_{2})=h^{2}_{\sigma}(N_{2})=36-\alpha$, we have $\dim_{\mathbb{F}_{2}} \varsigma^{*}(H^{2}(\widetilde{K},\mathbb{F}_{2}))=(8+36)-36+\alpha=8+\alpha$.
We can interpret this as follows.
Consider the homomorphism
\begin{align*}
\varsigma^{*}_{\Z}:H^{2}(\widetilde{K},\Z)&\rightarrow H^{2}(\widetilde{\overline{Z_{0}}},\Z)\oplus (\oplus_{k=1}^{36} H^{2}(D_{k},\Z))\\
 u&\rightarrow (u\cdot\widetilde{\overline{Z_{0}}},u\cdot D_{1},...,u\cdot D_{36}).
\end{align*}
Since this is a map of torsion free $\Z$-modules (by Lemma \ref{H3} and universal coefficient formula), we can tensor by $\mathbb{F}_{2}$,
$$\varsigma^{*}=\varsigma^{*}_{\Z}\otimes \id_{\mathbb{F}_{2}}: H^{2}(\widetilde{K},\Z)\otimes\mathbb{F}_{2}\rightarrow H^{2}(\widetilde{\overline{Z_{0}}},\Z)\oplus (\oplus_{k=1}^{36} H^{2}(D_{k},\Z))\otimes\mathbb{F}_{2},$$
and we have $8+\alpha$ independent elements such that the intersection with the $D_{k}$ $k\in\left\{1,...,36\right\}$ and $\widetilde{\overline{Z_{0}}}$ are not all zero.
But, $\varsigma^{*}(\pi_{2*}(H^{2}(N_2,\Z)))=0$ and $\varsigma^{*}(\widetilde{\overline{Z_0}},\left\langle D_1,...,D_{36}\right\rangle)$, (it follows from Proposition \ref{commut}). By Lemma \ref{sublemma} (2), the element $\widetilde{\overline{Z_0}}+D_{1}+...+D_{36}$ is divisible by 2. Hence necessary, it remains $7+\alpha$ independent elements in $H^2(\widetilde{K},\Z)$ of the form  $\frac{u+d}{2}$ with $u\in \pi_{2*}(s^{*}(H^{2}(K_{2}(A),\Z)))$ and $d\in\left\langle D_1,...,D_{36}\right\rangle$. 

Let denote by $u_1,...,u_{7+\alpha}$ the $7+\alpha$ elements in $\pi_{2*}(s^{*}(H^{2}(K_{2}(A),\Z)))$ provided above.  
By Lemma \ref{sublemma} (2) $\left\langle D_1,...,D_{36}\right\rangle$ is primitive in $H^2(\widetilde{K},\Z)$. Hence necessary, the element $u_1,...,u_{7+\alpha}$ view as element in $\pi_{2*}(s^{*}(H^{2}(K_{2}(A),\mathbb{F}_{2})))$ are linearly independent. Since $\dim_{\mathbb{F}_{2}} \pi_{2*}(s^{*}(H^{2}(K_{2}(A),\mathbb{F}_{2})))=7$, it follows that $\alpha=0$ and $\Vect_{\mathbb{F}_{2}}(u_1,...,u_{7})=\pi_{2*}(s^{*}(H^{2}(K_{2}(A),\mathbb{F}_{2})))$.
Hence there exists $D_\delta$ which is a linear combination of the $D_i$ with coefficient 0 or 1 such that $\pi_{2*}(s^{*}(\delta))+D_\delta$ is divisible by 2.
\end{proof}
\begin{lemme}\label{Ddelta}
We have:
$$D_\delta=D_1+...+D_{36}.$$
\end{lemme}
\begin{proof}
The know that the monodromy acts on $A[2]$ as the symplectic group $\Sp A[2]$. Hence the monodromy action extends naturally to an action on the divisors $D_1,...,D_{35}$. Also this monodromy action represented by $Sp A[3]$ acts trivially on $D_{36}$ and on $\pi_{2*}(s^{*}(\delta))$. 
As explaining in Remark 1.4 \ref{PlaneTriple} the 2 orbits of the action of $\Sp A[2]$ on the set $\mathfrak{D}:=\left\{D_1,...,D_{35}\right\}$ correspond to the two sets of isotropic and non-isotropic planes in $A[2]$. Hence by Proposition 1.5 (3), (4) \ref{OrbitesSp} the action of $\Sp A[2]$ on the set $\mathfrak{D}$ has 2 orbits: one of 15 elements and another of 20 elements. 

On the other hand, as we mentioned in the end of Section \ref{statement}, we can assume that $A=E\times E$ where $E=\frac{\C}{\left\langle 1,e^{\frac{2i\pi}{6}}\right\rangle}$. Hence there is the following automorphism group acting on $A$:
$$G:=\left\langle \left(
\begin{array}{cc}
\rho & 0\\
0 & 1 
\end{array} \right), 
\left(
\begin{array}{cc}
0 & 1\\
1 & 0 
\end{array} \right),
\left(
\begin{array}{cc}
1 & 1\\
0 & 1 
\end{array} \right)
 \right\rangle,$$
 where $\rho=e^{\frac{2i\pi}{6}}$.
The group $G$ extends naturally to an automorphism group of $N_2$ which we denote also $G$.  Moreover the action of $G$ restricts to the set $\mathfrak{D}$. Then by Proposition \ref{orbitesG} the action of $G$ on $\mathfrak{D}$ has 2 orbits: one of 5 elements and one of 30 elements. Also the group $G$ acts trivially on $D_{36}$ and on $\pi_{2*}(s^{*}(\delta))$. 

Hence the composition of the action of $G$ and the action of $\Sp A[2]$ acts transitively on $\mathfrak{D}$.
Since $\pi_{2*}(s^{*}(\delta))$ is fixed by the action of $G$ and $\Sp A[2]$, $D_\delta$ has also to be fixed by the action of $G$ and $\Sp A[2]$ or it will contradict Lemma \ref{sublemma} (2).
It follows that there are only 3 possibilities for $D_\delta$: 
\begin{itemize}
\item[(1)]
$D_\delta=D_{36}$,
\item[(2)]
$D_\delta=D_1+...+D_{35}$,
\item[(3)]
or $D_\delta=D_1+...+D_{36}$.
\end{itemize}
Let $d$ be the number of $D_i$ with coefficient equal to 1 in the linear decomposition of $D_\delta$. The number $d$ can be 1, 35 or 36.
%Let denote by $E_\delta$ be a linear combination of the $E_i$ with coefficient 0 or 1 such that $\pi_{2*}(E_\delta)=D_\delta$. Let $d$ be the number of $E_i$ with coefficient equal to 1. We know that $d$ can be 1, 35 or 36.

Then from Lemma \ref{Fulton} (1), (2) and Proposition \ref{commut}
$$\left(\frac{\pi_{2*}(s^{*}(\delta))+D_\delta}{2}\right)^4=\frac{324-d}{2}.$$ 
Hence $d$ has to be divisible by 2.
It follows that $D_\delta=D_1+...+D_{36}$.
\end{proof} 
\begin{lemme}\label{dernierlemme}
The class $\pi_{1*}(s_1^{*}(\delta))+\overline{Z_{0}}'$ is divisible by 2.
\end{lemme}
\begin{proof}
We know that $\pi_{2,*}(s^{*}(\delta))+\widetilde{\overline{Z_{0}}}$ is divisible by 2.
Indeed by Lemma \ref{sublemma} (2), $\widetilde{\overline{Z_{0}}}+D_1+...+D_{36}$ is divisible by 2 and by Lemma \ref{exist} and \ref{Ddelta},
$\pi_{2,*}(s^{*}(\delta))+D_1+...+D_{36}$ is divisible by 2. 

We can find a Cartier divisor on $\widetilde{K}$ which corresponds to $\frac{\pi_{2*}(s^{*}(\delta))+\widetilde{\overline{Z_0}}}{2}$ and which does not meet 
$\cup_{k=1}^{36} D_k$.
Then this Cartier divisor induces a Cartier divisor on $K'$ which necessarily corresponds to half the cocycle $\pi_{1*}(s_{1}^{*}(\delta))$ $+\overline{Z_0}'$.
\end{proof}
\subsection{Calculation of $B_{K'}$}\label{beauK'}
We finish the proof of Theorem \ref{theorem}, calculating $B_{K'}$. We use the notation provided in Section \ref{nota}. 
\begin{lemme}\label{Zinter}
We have $$\overline{Z_{0}}'^{2}=-2r^{*}(\overline{Z_{0}}).$$
%We have $$\hat{Z_{0}}^{2}=-s_{1}^{*}(Z_{0}).$$
\end{lemme}
\begin{proof}
%Since $K_{2}(A)$ is hyperk�hler and $Z_{0}$ is a K3 surface, we have $c_{1}(\mathscr{N}_{Z_{0}/K_{2}(A)})=0$. Then it is the same proof as 
We use the same technique as Lemma 4.6.12 of \cite{Lol}.
Consider the following diagram:
$$\xymatrix{
Z_{0}' \ar[d]^{g}\ar@{^{(}->}[r]^{l_{1}}& N_{1} \ar[d]^{s_{1}}\\
   Z_{0} \ar@{^{(}->}[r]^{l_{0}}  & K_{2}(A),
   }$$
where $l_{0}$ and $l_{1}$ are the inclusions and $g:=s_{1|Z_0'}$.
By Proposition 6.7 of \cite{Fulton}, we have:
$$s_{1}^{*}l_{0*}(Z_{0})=l_{1*}(c_{1}(E)),$$
where $E:=g^{*}(\mathscr{N}_{Z_{0}/K_{2}(A)})/\mathscr{N}_{Z_{0}'/N_{1}}$.
Hence $$s_{1}^{*}l_{0*}(Z_{0})=c_{1}(g^{*}(\mathscr{N}_{Z_{0}/K_{2}(A)}))-Z_{0}'^{2}.$$
Since $K_{2}(A)$ is hyperk�hler and $Z_{0}$ is a K3 surface, we have $c_{1}(\mathscr{N}_{Z_{0}/K_{2}(A)})=0$.
So
$$Z_{0}'^{2}=-s_{1}^{*}l_{0*}(Z_{0}).$$
Then the result follows from Proposition \ref{commut}.
\end{proof}
\begin{prop}\label{passage}
We have the formula $$B_{K'}(\pi_{1*}(s_{1}^{*}(\alpha),\pi_{1*}(s_{1}^{*}(\beta)))=6\sqrt{\frac{2}{C_{K'}}}B_{K_2(A)}(\alpha,\beta),$$
where $C_{K'}$ is the Fujiki constant of $K'$ and $\alpha$, $\beta$ are in $H^{2}(K_2(A),\Z)^{\iota}$ and $B_{K_2(A)}$ is the Beauville--Bogomolov form of $K_2(A)$.
\end{prop} 
\begin{proof}
An easy use of the Fujiki formula provide the following proposition (same proof as Proposition 4.6.10 of \cite{Lol}).

By (1) of Theorem \ref{Form}, we have $$(\pi_{1*}(s_{1}^{*}(\alpha)))^{4}=C_{K'}B_{K'}(\pi_{1*}(s_{1}^{*}(\alpha),\pi_{1*}(s_{1}^{*}(\alpha)))^{2}.$$
And $$\alpha^{4}=9B_{K_2(A)}(\alpha,\alpha)^{2}.$$
Moreover, by Proposition \ref{commut}, $$(\pi_{1*}(s^{*}(\alpha)))^{4}=8s^{*}(\alpha)^{4}=8\alpha^{4}.$$
By Point (2) of Theorem \ref{Form}, we get the result.
\end{proof}
In particular, it follows:
\begin{equation}
B_{K'}(\pi_{1*}(s_{1}^{*}(\delta),\pi_{1*}(s_{1}^{*}(\delta)))=-36\sqrt{\frac{2}{C_{K'}}}
\label{deltahaha}
\end{equation}
\begin{lemme}\label{ortho}
$$B_{K'}(\pi_{1*}(s_{1}^{*}(\alpha)),\overline{Z_0}')=0,$$
for all $\alpha\in H^{2}(S^{[2]},\Z)^{\iota}$.
\end{lemme}
\begin{proof}
We have $\pi_{1*}(s_{1}^{*}(\alpha))^{3}\cdot\overline{Z_0}'=8s_{1}^{*}(\alpha)^{3}\cdot\Sigma_{1}$ by Proposition \ref{commut},
and $s_{1*}(s_{1}^{*}(\alpha^{3})\cdot Z_0')=\alpha^{3}\cdot s_{1*}(Z_0')=0$ by the projection formula.
We conclude by Proposition \ref{beauville}.
\end{proof}
\begin{lemme}\label{Z2}
We have:
$$B_{K'}(\overline{Z_0}',\overline{Z_0}')=-4\sqrt{\frac{2}{C_{K'}}}.$$
\end{lemme}
\begin{proof}
We have:
\begin{align*}
\overline{Z_0}'^{2}\cdot\pi_{1*}(s_{1}^{*}(\delta))^{2}
&=\frac{C_{K'}}{3}B_{M'}(\overline{Z_0}',\overline{Z_0}')\times B_{K'}(\pi_{1*}(s_{1}^{*}(\delta)),\pi_{1*}(s_{1}^{*}(\delta)))\\
&=\frac{C_{M'}}{3}B_{K'}(\overline{Z_0}',\overline{Z_0}')\times\left(-36\sqrt{\frac{2}{C_{K'}}}\right)\\
\end{align*}
\vspace{-1cm}
\begin{equation}
=-12\sqrt{2C_{K'}}B_{K'}(\overline{Z_0}',\overline{Z_0}')\ \ \ \ \ \ 
\label{jenesaispas1}
\end{equation}
By Proposition \ref{commut}, we have 
\begin{equation}
\overline{Z_0}'^{2}\cdot\pi_{1*}(s_{1}^{*}(\delta))^{2}=8Z_0'^{2}\cdot (s_{1}^{*}(\delta))^{2}.
\label{jenesaispas2}
\end{equation}
By the projection formula, 
$Z_0'^{2}\cdot (s_{1}^{*}(\delta))^{2}=s_{1*}(Z_0'^{2})\cdot\delta^{2}$.
Moreover by lemma \ref{Zinter}, $s_{1*}(Z_0'^{2})=-Z_0$. 
Hence
\begin{equation}
Z_0'^{2}\cdot (s_{1}^{*}(\delta))^{2}=-Z_0\cdot\delta^{2}.
\label{jenesaispas3}
\end{equation}
It follows from (\ref{jenesaispas1}), (\ref{jenesaispas2}) and (\ref{jenesaispas3}) that
%Indeed, we can write
%$$\omega_{\Sigma_{1}}=\omega_{N_{1}}\otimes\mathscr{N}_{\Sigma_{1}/N_{1}}.$$
%The map $s_{|\Sigma_{1}}:\Sigma_{1}\rightarrow\Sigma$ is a $\mathbb{P}^{1}$-bundle, so $\omega_{\Sigma_{1}}=\mathcal{O}_{\Sigma_{1}}(-2u(\Sigma))$,
%where $u:\Sigma\hookrightarrow\Sigma_{1}$ is the embedding.
%We obtain:
%\begin{align*}
%\mathcal{O}_{\Sigma_{1}}(-2u(\Sigma))&=\mathcal{O}_{N_{1}}(\Sigma_{1})\otimes\mathcal{O}_{N_{1}}(\Sigma_{1})\otimes\mathcal{O}_{\Sigma_{1}}\\
%&=\mathcal{O}_{N_{1}}(2\Sigma_{1})\otimes\mathcal{O}_{\Sigma_{1}}.
%\end{align*}
%So $\Sigma_{1}^{2}=-s_{1}^{*}(\Sigma)$.
%Indead, we know that $\Sigma_{1}\rightarrow \Sigma$ is a $\mathbb{P}^{1}$-bundle, then the canonical bundle of $\Sigma_{1}$ is $K_{\Sigma_{1}}=-2\Sigma$ and the canonical bundle of $N_{1}$ is $K_{N_{1}}=\Sigma_{1}$.
\begin{equation}
-8Z_0\cdot\delta^{2}=-12\sqrt{2C_{K'}}B_{K'}(\overline{Z_0}',\overline{Z_0}').
\label{jenesaispas4}
\end{equation}
Moreover from Section 4 of \cite{Hassett}, we have:
\begin{equation}
Z_0\cdot\delta^{2}=-12.
\label{jenesaispas5}
\end{equation}
So by (\ref{jenesaispas4}) and (\ref{jenesaispas5}):
$$B_{K'}(\overline{Z_0}',\overline{Z_0}')=-8\sqrt{\frac{1}{2C_{K'}}}.$$
\end{proof}
Now we are able to finish the calculation of the Beauville--Bogomolov form on $H^{2}(K',\Z)$.
By (\ref{deltahaha}), Propositions \ref{passage}, Lemma \ref{ortho}, \ref{Z2} and Theorem \ref{fin},
the Beauville--Bogomolov form on $H^{2}(K',\Z)$ gives the lattice:
$$ U^{3}\left(6\sqrt{\frac{2}{C_{K'}}}\right) \oplus -\frac{1}{4}\sqrt{\frac{2}{C_{K'}}}\left(
\begin{array}{cc}
40 & 32\\
32 & 40 
\end{array} \right)$$
$$=U^{3}\left(6\sqrt{\frac{2}{C_{K'}}}\right) \oplus -\sqrt{\frac{2}{C_{K'}}}\left(
\begin{array}{cc}
10 & 8\\
8 & 10 
\end{array} \right)$$
It follows that $C_{K'}=8$, and we get Theorem \ref{theorem}.
\subsection{Betti numbers and Euler characteristic of $K'$}
\begin{prop}\label{b}
We have:
\begin{itemize}
\item $b_2(K')=8$,
\item $b_{3}(K')=0,$
\item $b_{4}(K')=90,$
\item $\chi(K')=108.$
\end{itemize}
\end{prop}
\begin{proof}
It is the same proof as Proposition 4.7.2 of \cite{Lol}.
From Theorem 7.31 of \cite{Voisin} and Proposition \ref{invariants}, we get the betti numbers.
Then $\chi(K')=1-0+8-0+90-0+8-0+1=108$.
\end{proof}