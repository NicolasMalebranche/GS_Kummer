\documentclass{amsart}

\usepackage{amsmath,amssymb,amsfonts,amscd}
\usepackage[all]{xy}
\usepackage{appendix,listings,hyperref}

\DeclareMathOperator{\rank}{rank}
\DeclareMathOperator{\trace}{tr}
\DeclareMathOperator{\Tor}{Tor}
\DeclareMathOperator{\Ext}{Ext}
\DeclareMathOperator{\Aut}{Aut}
\DeclareMathOperator{\End}{End}
\DeclareMathOperator{\id}{id}
\DeclareMathOperator{\Hom}{Hom}
\DeclareMathOperator{\im}{Im}
\DeclareMathOperator{\Ker}{Ker}
\DeclareMathOperator{\Sym}{Sym}
\DeclareMathOperator{\Hilb}{Hilb}
\DeclareMathOperator{\ch}{ch}
\DeclareMathOperator{\rk}{rk}
\DeclareMathOperator{\ad}{ad}

\newcommand{\hilb}[1]{^{[#1]}}
\newcommand{\ie}{{\it i.e. }}
\newcommand{\eg}{{\it e.g. }}
\newcommand{\loccit}{{\it loc. cit. }}
\newcommand{\vac}{|0\rangle}
\newcommand{\odd}{{\rm{odd}}}
\newcommand{\even}{{\rm{even}}}
\newcommand{\tors}{{\rm{tors}}}

\newcommand{\p}{\mathfrak{p}}
\newcommand{\pone}{ \mathfrak{p}_{ - 1} }

\newcommand{\coloneqq}{:=}
\newcommand{\kum}[2]{K_{ #2 }( #1 )}


%%%%%%%%%%%%%%%%%%%%%%%%%%%%%%

\newcommand{\C}{\mathbb{C}}
\renewcommand{\H}{\mathbb{H}}
\newcommand{\R}{\mathbb{R}}
\newcommand{\Q}{\mathbb{Q}}
\newcommand{\Z}{\mathbb{Z}}


%%%%%%%%%%%%%%%%%%%%%%%%%%%%%

\newcommand{\kS}{\mathfrak{S}}

\newcommand{\km}{\mathfrak{m}}
\newcommand{\kq}{\mathfrak{q}}

%%%%%%%%%%%%%%%%%%%%%%%%%%%%%%

\newcommand{\lra}{\longrightarrow}
\newcommand{\ra}{\rightarrow}

%%%%%%%%%%%%%%%%%%%%%%%%%%%%%

\theoremstyle{plain}
\newtheorem{theorem}{Theorem}[section]
\newtheorem{lemma}[theorem]{Lemma}
\newtheorem{proposition}[theorem]{Proposition}
\newtheorem{corollary}[theorem]{Corollary}
\theoremstyle{definition}
\newtheorem{definition}[theorem]{Definition}
\newtheorem{notation}[theorem]{Notation}
\theoremstyle{remark}
\newtheorem{remark}[theorem]{Remark}
\newtheorem{example}[theorem]{Example}


%%%%%%%%%%%%%%%%%%%%%%%%%%%%%

\begin{document}

\title{Integral cohomology of $\kum{A}{2}$}


\author{Simon Kapfer}
\author{Gr\'egoire Menet}
\address{Simon Kapfer, Laboratoire de Math\'ematiques et Applications, UMR CNRS 6086, Universit\'e de Poitiers, T\'el\'eport 2, Boulevard Marie et Pierre Curie, F-86962 Futuroscope Chasseneuil}
\email{simon.kapfer@math.univ-poitiers.fr}
%\urladdr{http://www.math.uni-augsburg.de/alg/}


\date{\today}

%\keywords{}

\begin{abstract} 
What we know already
\end{abstract}

\maketitle

\section{Preliminaries}
\begin{definition}
Let $n$ be a natural number. A partition of $n$ is a decreasing sequence $\lambda = (\lambda_1,\ldots,\lambda_k),\ \lambda_1\geq\ldots\geq\lambda_k>0$ of natural numbers such that $\sum_i \lambda_i =n$. Sometimes it is convenient to write $\lambda = (\ldots,2^{m_2},1^{m_1})$ with multiplicities in the exponent.
We define the weight $\|\lambda\| :=\sum m_i i =n$ and the length $|\lambda| := \sum_i m_i =k$. 
We also define $z_\lambda \coloneqq\prod_i i^{m_i} m_i!$. 
\end{definition}
\begin{definition} \label{SymFun}
Let $\Lambda_n := \Q[x_1,\ldots,x_n]^{S_n}$ be the graded ring of symmetric polynomials. There are canonical projections $: \Lambda_{n+1}\rightarrow\Lambda_n$ which send $x_{n+1}$ to zero. The graded projective limit
$\Lambda:=\lim\limits_{\leftarrow}\Lambda_n$ is called the ring of symmetric functions.
Let $m_\lambda$ and $p_\lambda$ denote the monomial and the power sum symmetric functions. They are defined as follows: For a monomial $x_{i_1}^{\lambda_1}x_{i_2}^{\lambda_2}\ldots x_{i_k}^{\lambda_k}$ of total degree $n$, the (ordered) sequence of exponents $(\lambda_1,\ldots,\lambda_k)$ defines a partition $\lambda$ of $n$, which is called the shape of the monomial. Then we define $m_\lambda$ being the sum of all monomials of shape $\lambda$. 
For the power sums, first define $p_n := x_1^n + x_2^n + \ldots$. 
Then $p_\lambda := p_{\lambda_1}p_{\lambda_2}\ldots p_{\lambda_k}$.
The families $(m_\lambda)_\lambda$ and $(p_\lambda)_\lambda$ form two $\Q$-bases of $\Lambda$, so they are linearly related by $p_\lambda = \sum_{\mu} \psi_{\lambda\mu}m_\mu$. It turns out that the base change matrix $(\psi_{\lambda\mu})$ has integral entries, but its inverse $(\psi_{\lambda\mu}^{-1})$ has not.
\end{definition}

\section{Cohomology of Hilbert schemes of points on a torus surface}
Let $A$ be a complex projective torus of dimension $2$. Its first cohomology $H^1(A,\Z)$ is freely generated by four elements $a_1,a_2,a_3,a_4,$ corresponding to the four different circles on the torus. The cohomology ring is isomorphic to the exterior algebra:
$$
H^*(A,\Z) = \Lambda^* H^1(A,\Z).
$$
We abbreviate for the products $a_i\cdot a_j =: a_{ij}$ and $a_i\cdot a_j\cdot a_k =: a_{ijk}$. We write $a_1\cdot a_2\cdot a_3 \cdot a_4 =:x$ for the class corresponding to a point on $A$. We choose the $a_i$ such that $\int_A x = 1$. We set $a_{\overline{i}}$ for the dual class of $a_i$, \ie  $a_i\cdot a_{\overline{i}} =x$.
The bilinear form, given by $(a_{ij},a_{kl})\mapsto\int_A a_{ij}a_{kl}$ gives $H^2(A,\Z)$ the structure of a unimodular lattice, isomorphic to $U^{\oplus 3}$, three copies of the hyperbolic lattice. 

Let $A\hilb{n}$ the Hilbert scheme of $n$ points on the torus, \ie the moduli space of finite subschemes of $A$ of length $n$. Their rational cohomology can be described in terms of Nakajima's operators. First consider the direct sum
$$
\H := \bigoplus_{n=0}^{\infty} H^*(A\hilb{n},\Q)
$$
This space is bigraded by cohomological \emph{degree} and the \emph{weight}, which is given by the number of points $n$. The unit element in $H^0(A\hilb{0},\Q) \cong \Q$ is denoted by $\vac$, called the \emph{vacuum}.

There are linear operators $\p_m(\alpha)$, for each $m\in \Z,\ \alpha \in H^*(A,\Q)$, acting on $\H$ which have the following properties: They depend linearly on $\alpha$, and if $\alpha\in H^k(A,\Q)$ is homogeneous, the operator $\p_{-m}(\alpha)$ is bihomogeneous of degree $k+2(|m|-1)$ and weight $m$:
$$
\p_{-m}(\alpha) : H^l(A\hilb{n}) \rightarrow H^{l+k+2(|m|-1)}(A\hilb{n+m})
$$
They satisfy the following commutation relations for $\alpha\in H^k(A,\Q),\ \beta\in H^{k'}(A,\Q)$:
$$
\p_{m}(\alpha)\p_{m'}(\beta) - (-1)^{k\cdot k'}\p_{m'}(\beta)\p_{m}(\alpha) = -m\,\delta_{m,-m'} \int_A \alpha\cdot\beta.
$$
Every element in $\H$ can be decomposed uniquely as a linear combination of products of operators $\p_{m}(\alpha),\ m<0$, acting on the vacuum. 
We abbreviate for a partition $\lambda=(\lambda_1,\ldots ,\lambda_k)$: \begin{align}
\kq_\lambda(\alpha) & := \prod_{i=1}^k \p_{-\lambda_i}(\alpha) \\
\kq_{*\lambda}(\alpha) & := \left(\prod_{i=1}^k \p_{-\lambda_i}\right)\Big(\Delta_{(k)}(\alpha)\Big)
\end{align}

For the study of integral cohomology, first note that if $\alpha \in H^*(A,\Z)$ is an integral class, then $\p_{-m}(\alpha) $ maps integral classes to integral classes. 
Moreover, there is the following theorem:
\begin{theorem} \cite{QinWang}
The following operators map integral classes in $\H$ to integral classes:
\begin{itemize}
\item $\frac{1}{z_\lambda}\kq_{\lambda}(1)$ 
\item $\mathfrak{m}_{\lambda}(\alpha)$ for $\alpha \in H^2(A,\Z)$ 
\end{itemize}
Here, $\mathfrak{m}_{\lambda}$ is defined as $\mathfrak{m}_{\lambda}(\alpha):=\sum_{\mu} \psi^{-1}_{\lambda\mu} \kq_{-\mu}(\alpha) $ (see Definition \ref{SymFun})
\end{theorem}
%For classes $\alpha_1,\ldots ,\alpha_r \in H^*(A)$ and partitions $\lambda_1,\ldots, \lambda_r$ we write $\alpha_1^{\lambda_1}\ldots\alpha_r^{\lambda_r}$ for the class $\p_{-\lambda_1}(\alpha_1)\ldots\p_{-\lambda_r}(\alpha_r)\vac$.

To obtain the multiplicative structure of $\H$, given by the cup-products, there is a description in \cite{LehnSorger} and \cite{LiQinWang} in terms of multiplication operators $\mathfrak{G}_k(a),\ a\in H^*(A)$ \cite[Def.~5.1]{LiQinWang}, related to chern characters. There is the following commutation relation: 
$$
[\mathfrak{G}_k(a),\kq_1(b)] = \frac{1}{k!}\ad(\mathfrak{d})^k(\kq_1(ab)),
$$
where the operator $\mathfrak{d}$ means multiplication with the first Chern class of the tautological sheaf.
We set $a^{(k)} := \mathfrak{G}_k(a) (1)$.

Next we focus on the structure of $H^2(A\hilb{n},\Z)$ for $n\geq 2$. It has rank $13$, and there is a basis consisting of:
\begin{itemize}
 \item $\frac{1}{(n-1)!}\p_{-1}(1)^{n-1}\p_{-1}(a_{ij})\vac,\ 1\leq i < j\leq 4$,
 \item $\frac{1}{(n-2)!}\p_{-1}(1)^{n-2}\p_{-1}(a_{i})\p_{-1}(a_{j})\vac,\ 1\leq i < j\leq 4$,
 \item $\frac{1}{2(n-2)!}\p_{-1}(1)^{n-2}\p_{-2}(1) \vac$. We denote this class by $\delta$.
\end{itemize}
It is clear that these classes form a basis of $H^2(A\hilb{n},\Q)$. By \cite[Thm.~4.6,Lemma~5.2]{QinWang}, they also form a basis for $H^2(A\hilb{n},\Z)$. TODO: refine this argument

The first $6$ classes give an injection $j : H^2(A,\Z)\rightarrow H^2(A\hilb{n},\Z)$. 

\section{Generalized Kummer varieties}
\begin{definition}
Let $A$ be a complex projective torus of dimension $2$ and $A\hilb{n}$, $n\geq 1$, the corresponding Hilbert scheme of points. Denote $\Sigma : A\hilb{n} \rightarrow A$ the summation morphism, a smooth submersion that factorizes via the Hilbert--Chow morphism $: A\hilb{n}\rightarrow\Sym^n(A)\rightarrow A$. Then the generalized Kummer $K^{n-1} A $ is defined as the fiber over $0$:
\begin{equation}\label{square}
\begin{CD}
K^{n-1}A @>\theta >> A\hilb{n}\\
@VVV @VV\Sigma V\\
\{0\} @> >> A
\end{CD}
\end{equation}
\end{definition}
Our first objective is to collect some information about this pullback diagram. 
We recall Theorem 2 of \cite{Spanier}.
\begin{theorem}\label{torsion}
The cohomology of $K_{2}(A)$ is torsion free. 
\end{theorem}
Or main reference is \cite{Beauville} where it is shown, that $K^{n-1}$ is an irreducible holomorphically symplectic manifold. So $H^2(\kum{A}{n-1},\Z)$ admits an integer-valued nondegenerated quadratic form (called Beauville--Bogomolov form) $q$ which gives $H^2(\kum{A}{n-1},\Z)$ the structure of a lattice isomorphic to $U^{\oplus 3}\oplus \left< -2n \right>$, for $n\geq 3$. We have the following formula for $\alpha\in H^2(\kum{A}{n-1},\Z)$:
\begin{equation} \label{fujiki}
\int_{\kum{A}{n-1}} \alpha^{2n-2} = n\frac{(2n-2)!}{2^{n-1}(n-1)!} q(a)^{n-1}
\end{equation}


The morphism $\theta$ induces a homomorphism of graded rings
\begin{equation}
\theta^* :H^*(A\hilb{n},\Z)\longrightarrow H^*(\kum{A}{n-1},\Z).
\end{equation}

\begin{proposition}Let $n\geq 3$.
\begin{enumerate} 
 \item $\theta^*$ maps $H^1(A\hilb{n},\Z)$ to zero.
 \item $\theta^*$ is surjective on $H^2(A\hilb{n},\Z)$ with kernel $\Lambda^2H^1(A\hilb{n},\Z)$.
\end{enumerate}
\end{proposition}
\begin{proof}
The first statement is clear since $H^1(\kum{A}{n-1})$ is always zero \cite[Thm.~3]{Beauville}. Furthermore, by \cite[Sect.~7]{Beauville}, $\theta^{\ast} : H^2(A\hilb{n},\C) \rightarrow H^2(\kum{A}{n-1},\C)$ is surjective. The second Betti numbers of $A\hilb{n}$ and $\kum{A}{n-1}$ are $13$ and $7$, respectively. It is clear that $\Lambda^2H^1(A\hilb{n},\Z)$ is contained in the kernel, and since the dimension of the kernel has to be $6$, it must be all.

It remains to show that $\theta^*$  is surjective for integral coefficients, too. We do it only for $n=3$. We use a formula in \cite[p. 8]{Britze}, namely:
\begin{equation} \int_{A\hilb{3}}j(a)^6 = \frac{5}{3} \int_A a^2 \int_{\kum{A}{2}} \theta^* j(a)^4
\end{equation}
for all $a\in H^2(A)$. One computes $\int_{A\hilb{3}}j(a)^6 = 15 \left(\int_A a^2\right)^3$. Comparing this with (\ref{fujiki}), we see that the sublattice given by the image of $\theta^*\circ j$ is unimodular. Secondly, we must show that $q(\theta^*\delta) = -6$. TODO: show this! 
Remark: $\theta^*\delta$ seems to be indivisible (because of (\ref{fujiki})), but every product with $\theta^*\delta$ is divisible by 3. Indeed, the value of (\ref{fujiki}) for $\alpha=\theta^*\delta$ is 324.
\end{proof}


\begin{proposition}
We have $a_i^{(0)}= \frac{1}{2}\kq_1(1)^2\kq_1(a_i)\vac$. The class of $\kum{A}{2}$ in $H^4(A\hilb{3},\Q)$ is given by
$$
%\prod_{i=1}^4 \left(\tfrac{1}{2}\pone(1)^2\pone(\alpha_i)\vac\right).
a_1^{(0)}\cdot a_2^{(0)}\cdot a_3^{(0)}\cdot a_4^{(0)}.
$$ 
Conjecture: This is true for all $n$, not only $n=3$.
\end{proposition}
\begin{proof}
We know that for all $i$ and all $\beta\in H^7(A\hilb{3})$, we have $\int_ {\kum{A}{2}}\theta^*(\alpha_i\cdot\beta) = \int_ {A\hilb{3}}\alpha_i\cdot\beta \cdot[\kum{A}{2}]= 0$ and
for a basis $(\gamma_i) $ of  $H^2(A\hilb{3})$,
$$
\int_ {A\hilb{3}}\gamma_i\cdot\gamma_j\cdot\gamma_k\cdot\gamma_l\cdot[\kum{A}{2}] =
 3\left(\left<\gamma_i,\gamma_j\right>\left<\gamma_k,\gamma_l\right>+\left<\gamma_i,\gamma_k\right>\left<\gamma_j,\gamma_l\right>+\left<\gamma_i,\gamma_l\right>\left<\gamma_j,\gamma_k\right>  \right)
$$
These equations admit a unique solution.
\end{proof}
\begin{remark}
This allows us to better understand the morphism $\theta^*$. Since the Poincar\'e pairing is nondegenerated, $[\kum{A}{n-1}]\cdot \alpha=0$ implies $\theta^* \alpha =0$.
\end{remark}

Now we focus on classes of cohomological degree 4.
\begin{proposition}
The classes $\theta^* \left(\p_{-2}(a_{ij})\pone(1)\vac\right) $ and $\theta^*\left( \p_{-2}(1)\pone(a_{ij})\vac\right) $ are linearly dependent.
\end{proposition}
\begin{proof}
We can compute the product of these two classes with $[ \kum{A}{2} ] $ in $H^*(A\hilb{3})$. The two results are linearly dependent. Is there a direct geometric proof?
\end{proof}

\begin{proposition}
$\theta^*\left(\p_{-3}(x)\vac\right) =0$ 
\end{proposition}
\begin{corollary}
$\theta^* \left(\p_{-2}(a_{ij})\pone(1)\vac\right) = \frac{1}{4}\theta^*\left( \p_{-2}(1)\pone(a_{ij})\vac\right) $
\end{corollary}
\begin{proof}
Let  $a_{ij}$ and $a_{kl}$ be complementary, \ie $a_{ij}a_{kl}=1$. We have $a_{kl}^{(1)} = -\frac{1}{2} \p_{-2}(a_{kl})\pone(1)\vac$. Then:
$$
\theta^*\left(a_{ij}^{(1)}\cdot a_{kl}^{(1)} \right) =
\theta^*\left(\p_{-3}(1)\vac + \frac{1}{2}\pone(x)^2\pone(1)\vac \right)
$$
But on the other hand, $\delta \cdot j(a) = \frac{1}{2} \p_{-2}(1)\pone(a_{ij})\vac+\p_{-2}(a_{ij})\pone(1)\vac$, and
$$
\theta^*\left(a_{kl}^{(1)} \cdot \delta \cdot j(a)\right) =
\theta^*\left(-3\p_{-3}(1)\vac  - 3\pone(x)^2\pone(1)\vac \right).
$$
\end{proof}
\begin{corollary}
$\theta^*\left( \delta \cdot j(a_{ij}) \right) = \theta^*\left( \frac{3}{4} \p_{-2}(1)\pone(a_{ij})\vac\right)$ is divisible by 3. \qed
\end{corollary}
\begin{proposition}
The classes $\theta^*\left(j(a_{ij})^2 - \frac{1}{3}j(a_{ij})\cdot \delta\right)$ are divisible by 2.
\end{proposition}
\begin{proof}
By \cite{QinWang}, the classes $\frac{1}{2} \pone(a_{ij})^2\pone(1)\vac - \frac{1}{2}\p_{-2}(a_{ij})\pone(1)\vac$ are integral in $H^4(A\hilb{n})$. But $j(a_{ij})^2= \pone(a_{ij})^2\pone(1)\vac $ and $\theta^*\left(\frac{1}{3}j(a_{ij})\cdot\delta\right) =\theta^*\left(\p_{-2}(a_{ij})\pone(1)\vac\right)$.
\end{proof}


\begin{proposition}\label{DelSum3}
The class $\theta^*\left(\delta^2+j(a_{12})\cdot j(a_{34})-j(a_{13})\cdot j(a_{24})+j(a_{14})\cdot j(a_{23})\right)$ is divisible by 3.
\end{proposition}
\begin{proof}
It is equal to $\theta^*\left(\p_{-3}(1)\vac -\frac{3}{2}\pone(x)\pone(1)^2\vac \right)$.
\end{proof}

\begin{proposition}\label{Pi'}
We have:
$$H^{4}(K_{2}(A),\Q)=\Sym^2 H^{2}(K_{2}(A),\Q)\oplus^{\bot} \Pi'\otimes\Q.$$
\end{proposition}
\begin{proof}
In Section 4 of \cite{HassettTschinkel}, we can find the following formula:
\begin{equation}
Z_{\tau}\cdot D_{1}\cdot D_{2}=2\cdot q(D_{1},D_{2}),
\label{ZT}
\end{equation}
for all $D_{1}$, $D_{2}$ in $H^{2}(K_{2}(A),\Z)$, $\tau\in A[3]$ and $q$ the Beauville-Bogomolov form.
It follows that $\Pi'\subset \Sym^2 H^{2}(K_{2}(A),\Z)^{\bot}$.
Since the cup-product is non-degenerated and by Proposition 4.3 of \cite{HassettTschinkel}, 
we have: 
\begin{align*}
\rk \left(\Sym^2 H^{2}(K_{2}(A),\Z) \oplus\Pi'\right)&=\rk \Sym^2 H^{2}(K_{2}(A),\Z) + \rk\Pi'\\
&=28+80\\
&= \rk H^{4}(K_{2}(A),\Z).
\end{align*}
It follows that $$H^{4}(K_{2}(A),\Q)=\Sym^2 H^{2}(K_{2}(A),\Q)\oplus^{\bot} \Pi'\otimes\Q.$$
\end{proof}

Next we look at the Chern classes of the tangent sheaves. Since the morphism $\Sigma$ from the defining pullback diagram (\ref{square}) is a submersion, the normal bundle of $\kum{A}{n-1}$ in $A\hilb{n}$ is trivial. Hence $c(\kum{A}{2}) = \theta^* c(A\hilb{3})$. Looking in \cite[Sect.~8]{Generating}, we find a general formula for Chern classes of Hilbert schemes of points on surfaces. So we deduce
\begin{align*}
c_2(A\hilb{3}) &= \left(\tfrac{3}{2} \mathfrak{q}_{*(1,1)}(1) \mathfrak{q}_1(1) -\tfrac{1}{3} \mathfrak{q}_3\right)\vac \\
 &= 10(1\hilb{\bullet}_{(4)})  -2(1\hilb{\bullet}_{(2)})^2 \\
c_4(A\hilb{3}) &= \tfrac{4}{3} \mathfrak{q}_{*(1,1,1)}( 1)\vac = 4 (1\hilb{\bullet}_{(4)})^2. 
\end{align*}

\begin{proposition}
We have:
$$c_{2}(K_{2}(A))=\theta^{*}\left(4j(a_{12})\cdot j(a_{34})-4j(a_{13})\cdot j(a_{24})+4j(a_{14})\cdot j(a_{23})-\frac{1}{3}\delta^{2}\right).$$
In particular $c_{2}(K_{2}(A))\in \Sym^2 H^{2}(K_{2}(A),\Z).$
\end{proposition}
\begin{proof}
We can write:
$$c_{2}(K_{2}(A))=a+b,$$
with $a\in \Sym^2 H^{2}(K_{2}(A),\Q)$ and $b\in \Pi'$.
First, we prove that $b=0$.
We have $c_{2}(K_{2}(A))\in \Pi'^{\bot}$ and also $a\in \Pi'^{\bot}$, it follows that 
$b\in \Pi'^{\bot}$. 
Since the cup-product is non-degenerated, it follows that $b$ is of torsion. 
Then by Theorem \ref{torsion}, $b=0$.

By (\ref{ZT}) and Proposition 5.1 of \cite{HassettTschinkel}, we can see that for all $D_{1}$ and $D_{2}$ in $H^{2}(K_{2}(A),\Z)$, we have:
$$c_{2}(K_{2}(A))\cdot D_{1}\cdot D_{2}=54\cdot q(D_{1},D_{2}),$$
where $q$ is the Beauville-Bogomolov form.
Then we can calculate that:
$$c_{2}(K_{2}(A))=\theta^{*}\left(4j(a_{12})\cdot j(a_{34})-4j(a_{13})\cdot j(a_{24})+4j(a_{14})\cdot j(a_{23})-\frac{1}{3}\delta^{2}\right).$$
\end{proof}
\begin{corollary}\label{DeltaSquare3}
The class $\theta^* \delta^2$ is divisible by 3.
\end{corollary}

\begin{proposition}
The element 
$$\theta^{*}\left(j(a_{12})\cdot j(a_{34})-j(a_{13})\cdot j(a_{24})+j(a_{14})\cdot j(a_{23})\right)$$ 
is divisible by 6. More precisely, it is equal to $6\,Y_p$ (see \cite{HassettTschinkel}).
\end{proposition}
\begin{proof}
Again by Section 4 of \cite{HassettTschinkel}, we have:
$$W=\frac{3}{8}(c_{2}(K_{2}(A))+3\theta^{*}(\delta)^2).$$
It follows:
\begin{equation}
W=\frac{3}{8}\theta^{*}\left(4j(a_{12})\cdot j(a_{34})-4j(a_{13})\cdot j(a_{24})+4j(a_{14})\cdot j(a_{23})+\frac{8}{3}\delta^{2}\right).
\label{W}
\end{equation}
It follows that 
$$\theta^{*}(j(a_{12})\cdot j(a_{34})-j(a_{13})\cdot j(a_{24})+j(a_{14})\cdot j(a_{23})).$$
is divisible by 2.
For the divisibility by 3, combine Proposition \ref{DelSum3} with Corollary \ref{DeltaSquare3}.
\end{proof}
\begin{remark}
We also have the following formulas:
\begin{align}
W &= \theta^* \left(\p_{-3}(1)\vac\right) \\
Y_p & = -\tfrac{1}{9}\, \theta^* \left( \pone(1) L_{-2}(1)\vac\right)
\end{align}
\end{remark}
Let us now look at cohomology classes of odd degree. Since $H^1(\kum{A}{2}) = H^7(\kum{A}{2}) =0$, we only need to consider the degrees 3 and 5.
\begin{proposition}
The map $\theta^* : H^*(A\hilb{3} ,\Q) \rightarrow H^*(\kum{A}{2},\Q)$ is surjective in degrees 3 and 5. 
If we set
\begin{align}
B_3 &:= \{ a_{\overline{i}}^{(0)},\; 1\leq i\leq 4\} \cup \{ a_{i}^{(1)},\; 1\leq i\leq 4\}  \\
B_5 &:= \{ a_{\overline{i}}^{(1)},\; 1\leq i\leq 4\} \cup \{ a_{i}^{(2)},\; 1\leq i\leq 4\} ,
\end{align}
then the images of $B_3$ and $B_5$ give bases of $H^3(\kum{A}{2},\Q)$ and $H^5(\kum{A}{2},\Q)$ that are orthogonal under the intersection pairing.
We have
\begin{align}
\int \theta^* \left( a_{\overline{ i}}^{(0)} \cdot  a_i^{(2)}  \right ) &= \pm \frac{3}{2} \\
\int \theta^* \left( a_i^{(1)} \cdot a_{\overline{ i}}^{(1)} \right ) &= \pm \frac{1}{2}.
\end{align}
Moreover, $a_{\overline{ i}}^{(0)}\cdot[\kum{A}{2}]$ and $\frac{2}{3} a_i^{(2)}\cdot[\kum{A}{2}]$ are integral classes. 
This implies (by Poincar\'e duality) that $\theta^* a_{\overline{ i}}^{(0)}$ and $\frac{2}{3} \theta^*a_i^{(2)} $ are integral. 

Question: Which of $\theta^*  a_i^{(1)} $ and $ \theta^*a_{\overline{ i}}^{(1)}$ is not integral?
\end{proposition}
 


\bibliographystyle{amsplain}
\begin{thebibliography}{10}

\bibitem{Beauville}
A.~Beauville, \emph{Variétés kähleriennes dont la première classe de Chern est nulle}, 
  J. Differential geometry, 18 (1983) 755-782

\bibitem{Generating}
S.~Boissi\`ere and M.~Nieper-Wi{\ss}kirchen, \emph{Generating series in the cohomology 
  of Hilbert schemes of points on surfaces}, LMS J.~of Computation and Mathematics 10 (2007), 254--270 .

\bibitem{BNS}
S.~Boissi\`ere, M.~Nieper-Wi{\ss}kirchen, and A.~Sarti, \emph{Smith theory and 
  Irreducible Holomorphic Symplectic Manifolds}, Journal of Topology 6 (2013), no.~2, 361–390.

\bibitem{Britze} 
M.~Britze, \emph{On the cohomology of generalized Kummer varieties}, (2003) 

\bibitem{HassettTschinkel}
B.~Hassett and Y.~Tschinkel, \emph{ Hodge theory and Lagrangian planes on 
  generalized Kummer fourfolds}, Moscow Math. Journal, 13, no. 1, 33-56, (2013) 
  
\bibitem{LehnSorger}
M.~Lehn and C.~Sorger, \emph{The cup product of {H}ilbert schemes for {$K3$}
  surfaces}, Invent. Math. \textbf{152} (2003), no.~2, 305--329.

\bibitem{LiQinWang}
W.~Li, Z.~Qin and W.~Wang, \emph{Vertex algebras and the cohomology ring structure of 
  Hilbert schemes of points on surfaces} (2002)

\bibitem{Markman}
E.~Markman, \emph{Integral generators for the cohomology ring of moduli spaces of
  sheaves over {P}oisson surfaces}, Adv. Math. \textbf{208} (2007), no.~2,
  622--646.

\bibitem{Markman2}
E.~Markman, \emph{Integral constraints on the monodromy group of the
  hyper{K}\"ahler resolution of a symmetric product of a {$K3$} surface},
  Internat. J. Math. \textbf{21} (2010), no.~2, 169--223.

\bibitem{Nakajima}
H.~Nakajima, \emph{Heisenberg algebra and {H}ilbert schemes of points on
  projective surfaces}, Ann. of Math. (2) \textbf{145} (1997), no.~2, 379--388.

\bibitem{QinWang}
Z.~Qin and W.~Wang, \emph{Integral operators and integral cohomology classes of
  {H}ilbert schemes}, Math. Ann. \textbf{331} (2005), no.~3, 669--692.

\bibitem{Spanier}
E.~Spanier, 
\newblock {\em The homology of Kummer manifolds},
\newblock Proc. Amer. Math. Soc.
\newblock 7, (1956), 155-160.

\end{thebibliography}
\end{document}
