\documentclass{amsart}

\usepackage{amsmath,amssymb,amsfonts,amscd}
\usepackage[all]{xy}
\usepackage{appendix,listings,hyperref}

\DeclareMathOperator{\rank}{rank}
\DeclareMathOperator{\trace}{tr}
\DeclareMathOperator{\Tor}{Tor}
\DeclareMathOperator{\Ext}{Ext}
\DeclareMathOperator{\Aut}{Aut}
\DeclareMathOperator{\End}{End}
\DeclareMathOperator{\id}{id}
\DeclareMathOperator{\Hom}{Hom}
\DeclareMathOperator{\im}{Im}
\DeclareMathOperator{\Ker}{Ker}
\DeclareMathOperator{\Sym}{Sym}
\DeclareMathOperator{\Hilb}{Hilb}
\DeclareMathOperator{\ch}{ch}


\newcommand{\hilb}[1]{^{[#1]}}
\newcommand{\ie}{{\it i.e. }}
\newcommand{\eg}{{\it e.g. }}
\newcommand{\loccit}{{\it loc. cit. }}
\newcommand{\vac}{|0\rangle}
\newcommand{\odd}{{\rm{odd}}}
\newcommand{\even}{{\rm{even}}}
\newcommand{\tors}{{\rm{tors}}}

\newcommand{\p}{\mathfrak{p}}
\newcommand{\pone}{ \mathfrak{p}_{ - 1} }

\newcommand{\coloneqq}{:=}
\newcommand{\kum}[2]{K_{ #2 }( #1 )}


%%%%%%%%%%%%%%%%%%%%%%%%%%%%%%

\newcommand{\C}{\mathbb{C}}
\renewcommand{\H}{\mathbb{H}}
\newcommand{\R}{\mathbb{R}}
\newcommand{\Q}{\mathbb{Q}}
\newcommand{\Z}{\mathbb{Z}}


%%%%%%%%%%%%%%%%%%%%%%%%%%%%%

\newcommand{\kS}{\mathfrak{S}}

\newcommand{\km}{\mathfrak{m}}
\newcommand{\kq}{\mathfrak{q}}

%%%%%%%%%%%%%%%%%%%%%%%%%%%%%%

\newcommand{\lra}{\longrightarrow}
\newcommand{\ra}{\rightarrow}

%%%%%%%%%%%%%%%%%%%%%%%%%%%%%

\theoremstyle{plain}
\newtheorem{theorem}{Theorem}[section]
\newtheorem{lemma}[theorem]{Lemma}
\newtheorem{proposition}[theorem]{Proposition}
\newtheorem{corollary}[theorem]{Corollary}
\theoremstyle{definition}
\newtheorem{definition}[theorem]{Definition}
\newtheorem{notation}[theorem]{Notation}
\theoremstyle{remark}
\newtheorem{remark}[theorem]{Remark}
\newtheorem{example}[theorem]{Example}


%%%%%%%%%%%%%%%%%%%%%%%%%%%%%

\begin{document}

\title{Integral cohomology of $K^2(A)$}


\author{Simon Kapfer}
\address{Simon Kapfer, Laboratoire de Math\'ematiques et Applications, UMR CNRS 6086, Universit\'e de Poitiers, T\'el\'eport 2, Boulevard Marie et Pierre Curie, F-86962 Futuroscope Chasseneuil}
\email{simon.kapfer@math.univ-poitiers.fr}
%\urladdr{http://www.math.uni-augsburg.de/alg/}


\date{\today}

%\keywords{}

\begin{abstract} 
What we know already
\end{abstract}

\maketitle

\section{Chomology of Hilbert schemes of points on a torus surface}
Let $A$ be a complex projective torus of dimension $2$. Its first cohomology $H^1(A,\Z)$ is freely generated by four elements $a_1,a_2,a_3,a_4,$ corresponding to the four different circles on the torus. The cohomology ring is isomorphic to the exterior algebra:
$$
H^*(A,\Z) = \Lambda^* H^1(A,\Z).
$$
We abbreviate for the products $a_i\cdot a_j =: a_{ij}$ and $a_i\cdot a_j\cdot a_k =: a_{ijk}$. We write $a_1\cdot a_2\cdot a_3 \cdot a_4 =:x$ for the class corresponding to a point on $A$. We choose the $a_i$ such that $\int_A x = 1$.
The bilinear form, given by $(a_{ij},a_{kl})\mapsto\int_A a_{ij}a_{kl}$ gives $H^2(A,\Z)$ the structure of a unimodular lattice, isomorphic to $U^{\oplus 3}$, three copies of the hyperbolic lattice. 

Let $A\hilb{n}$ the Hilbert scheme of $n$ points on the torus, \ie the moduli space of finite subschemes of $A$ of length $n$. Their rational cohomology can be described in terms of Nakajima's operators. First consider the direct sum
$$
\H := \bigoplus_{n=0}^{\infty} H^*(A\hilb{n},\Q)
$$
This space is bigraded by cohomological \emph{degree} and the \emph{weight}, which is given by the number of points $n$. The unit element in $H^0(A\hilb{0},\Q) \cong \Q$ is denoted by $\vac$, called the \emph{vacuum}.

There are linear operators $\p_m(\alpha)$, for each $m\in \Z,\ \alpha \in H^*(A,\Q)$, acting on $\H$ which have the following properties: They depend linearly on $\alpha$, and if $\alpha\in H^k(A,\Q)$ is homogeneous, the operator $\p_{-m}(\alpha)$ is bihomogeneous of degree $k+2(|m|-1)$ and weight $m$:
$$
\p_{-m}(\alpha) : H^l(A\hilb{n}) \rightarrow H^{l+k+2(|m|-1)}(A\hilb{n+m})
$$
They satisfy the following commutation relations for $\alpha\in H^k(A,\Q),\ \beta\in H^{k'}(A,\Q)$:
$$
\p_{m}(\alpha)\p_{m'}(\beta) - (-1)^{k\cdot k'}\p_{m'}(\beta)\p_{m}(\alpha) = -m\,\delta_{m,-m'} \int_A \alpha\cdot\beta.
$$
Every element in $\H$ can be decomposed uniquely as a linear combination of products of operators $\p_{m}(\alpha),\ m<0$, acting on the vacuum. 

For the study of integral cohomology we cite:
\begin{theorem} \cite{QinWang}
The following operators map integral classes in $\H$ to integral classes:
\begin{itemize}
\item $\p_{-\lambda}(\alpha)$ for $\alpha \in H^*(A,\Z)$ 
\item $\frac{1}{z_\lambda}\p_{-\lambda}(1)^n$ 
\item $\mathfrak{m}_{\lambda}(\alpha)$ for $\alpha \in H^2(A,\Z)$ 
\end{itemize}
Here, $\mathfrak{m}_{\lambda}$ is defined as $\mathfrak{m}_{\lambda}(\alpha):=\sum_{\mu} c_{\lambda\mu} \p_{-\mu}(\alpha) $ and $c_{\lambda\mu}$ are the coefficients of the transition matrix between monomial symmetric functions and power sum symmetric functions.
\end{theorem}
For a classes $\alpha_1,\ldots ,\alpha_r \in H^*(A)$ and a partitions $\lambda_1,\ldots, \lambda_r$ we write $\alpha_1^{\lambda_1}\ldots\alpha_r^{\lambda_r}$ for the class $\p_{-\lambda_1}(\alpha_1)\ldots\p_{-\lambda_r}(\alpha_r)\vac$.
\section{Generalized Kummer varieties}
\begin{definition}
Let $A$ be a complex projective torus of dimesion $2$ and $A\hilb{n}$ the corresponding Hilbert scheme of points. Denote $\Sigma : A\hilb{n} \rightarrow A$ the summation morphism. Then the generalized Kummer $K^{n-1} A $ is defined as the fiber over $0$:
\begin{equation}\label{square}
\begin{CD}
K^{n-1}A @>\theta >> A\hilb{n}\\
@VVV @VV\Sigma V\\
\{0\} @> >> A
\end{CD}
\end{equation}
\end{definition}
By \cite{Beauville}, $\theta^{\ast} : H^2(A\hilb{n}) \rightarrow H^2(K^{n-1}A)$ is surjective. We have injections $j : H^2(A)\rightarrow H^2(A\hilb{n})$ and $i = \theta^* j$. The cohomology $H^*(A\hilb{n})$ is described in terms of vertex operators in \cite{LehnSorger} and \cite{LiQinWang}.

We describe now the image of $\theta^*$ in the case $n=3$:
\begin{itemize}
\item We know $j(a)=\frac{1}{2}\p_{-1}(a)\p_{-1}(1)^2\vac$, because the two must be linearly dependent and
$$ \int_{A\hilb{3}}j(a)^6 = 15 q(a)^3, \quad \left(\frac{1}{2}\p_{-1}(a)\p_{-1}(1)^2\vac\right)^3 = 15 q(a)^3\p_{-1}(x)^3\vac.
$$
\item By \cite[p. 8]{Britze}, we have for $\alpha = j(a)=\frac{1}{2}\p_{-1}(a)\p_{-1}(1)^2\vac$: 
$$ \int_{A\hilb{3}}\alpha^6 = \frac{5}{3} q(a) \int_{K^2} \theta^* \alpha^4
$$
On the other hand, 
$$\alpha^4 = 3 q(a)^2\p_{-1}(x)^2\p_{-1}(0)\vac + 3q(a) \p_{-1}(x) \p_{-1}(a)^2\vac,$$
so if the image of both summands under $\int\theta^*$ is positive, then 
$$ \int\theta^*\p_{-1}(x)^2\p_{-1}(0)\vac = \int\theta^* \tfrac{1}{2}\p_{-1}(x) \p_{-1}(a)^2\vac=1 .$$ 

\end{itemize}
\begin{proposition}
The class of $K^2(A)$ in $H^4(A\hilb{3},\Q)$ is given by
$$
%\prod_{i=1}^4 \left(\tfrac{1}{2}\pone(1)^2\pone(\alpha_i)\vac\right).
a_1^{(1)}\cdot a_2^{(1)}\cdot a_3^{(1)}\cdot a_4^{(1)}.
$$ 
\end{proposition}
\begin{proof}
We know that for all $i$ and all $\beta\in H^7(A\hilb{3})$, we have $\int_ {K^2(A)}\theta^*(\alpha_i\cdot\beta) = \int_ {A\hilb{3}}\alpha_i\cdot\beta \cdot[K^2(A)]= 0$ and
for a basis $(\gamma_i) $ of  $H^2(A\hilb{3})$,
$$
\int_ {A\hilb{3}}\gamma_i\cdot\gamma_j\cdot\gamma_k\cdot\gamma_l\cdot[K^2(A)] =
 3\left(\left<\gamma_i,\gamma_j\right>\left<\gamma_k,\gamma_l\right>+\left<\gamma_i,\gamma_k\right>\left<\gamma_j,\gamma_l\right>+\left<\gamma_i,\gamma_l\right>\left<\gamma_j,\gamma_k\right>  \right)
$$
These equations admit a unique solution.
\end{proof}

Let $\{a_i\}_{i= 1 \ldots 6}$ be a hyperbolic basis of 
$H^2(A,\Z)$.
\begin{proposition}
The classes $\theta^* \left(\p_{-2}(a_i)\pone(1)\vac\right) $ and $\theta^*\left( \p_{-2}(1)\pone(a_i)\vac\right) $ are linearly dependent.
\end{proposition}
\begin{proposition}
$\theta^*\left(\p_{-3}(x)\vac\right) =0$ 
\end{proposition}
\begin{corollary}
$\theta^* \left(\p_{-2}(a_i)\pone(1)\vac\right) = \frac{1}{4}\theta^*\left( \p_{-2}(1)\pone(a_i)\vac\right) $
\end{corollary}
\begin{proof}
Let $a_j$ be complementary, \ie $a_ia_j=1$. Let $\ch_1(a_j) = -\frac{1}{2} \p_{-2}(a_j)\pone(1)\vac$ be the chern character in the vertex algebra description of $H^*(A\hilb{3})$. Then:
$$
\theta^*\left(-\frac{1}{2}\ch_1(a_j)\cdot\p_{-2}(a_i)\pone(1)\vac \right) =
\theta^*\left(\p_{-3}(1)\vac + \frac{1}{2}\pone(x)^2\pone(1)\vac \right)
$$
But on the other hand, $\delta \cdot j(a) = \frac{1}{2} \p_{-2}(1)\pone(a_i)\vac+\p_{-2}(a_i)\pone(1)\vac$, and
$$
\theta^*\left(\ch_1(a_j)\cdot \delta \cdot j(a)\right) =
\theta^*\left(-3\p_{-3}(1)\vac  - 3\pone(x)^2\pone(1)\vac \right).
$$
\end{proof}
\begin{corollary}
$\theta^*\left( \delta \cdot j(a_i) \right) = \theta^*\left( \frac{3}{4} \p_{-2}(1)\pone(a_i)\vac\right)$ is divisible by 3. \qed
\end{corollary}
\begin{proposition}
The classes $\theta^*\left(j(a_i)^2 - \frac{1}{3}j(a_i)\cdot \delta\right)$ are divisible by 2.
\end{proposition}
\begin{proof}
By \cite{QinWang}, the classes $\frac{1}{2} \pone(a_i)^2\pone(1)\vac - \frac{1}{2}\p_{-2}(a_i)\pone(1)\vac$ are integral in $H^4(A\hilb{n})$. But $j(a_i)^2= \pone(a_i)^2\pone(1)\vac $ and $\theta^*\left(\frac{1}{3}j(a_i)\cdot\delta\right) =\theta^*\left(\p_{-2}(a_i)\pone(1)\vac\right)$.
\end{proof}
\begin{proposition}
The class $\delta^2$ is divisible by 2.
\end{proposition}
\begin{proof}
By \cite[Prop.~4.1]{HassettTschinkel}, $\Sym^2H^2 \oplus \left(\Sym^2H^2\right)^\perp = H^4$. 
We want to show that $\delta^2\cdot \Sym^2H^2 = 2\Z$. We know a $\Q$--basis of $\Sym^2H^2$ with at most one class divisible by 2, given by $j(a_i)j(a_j)$, $\delta^2$ and the above proposition. By computation, $\int \delta^4$ is divisible by 4 and $\int \delta^2 j(a_i)j(a_j)$ and $\int \delta^3 j(a_i)$ are all divisible by 2.
So $\delta^2\cdot H^4 = 2\Z$ and therefore $\delta^2$ is divisible by $2$, since $ H^4 $ is unimodular.
\end{proof}

\begin{proposition}
The class $\theta^*\left(\delta^2 - j(a_1)\cdot j(a_2)- j(a_3)\cdot j(a_4)- j(a_5)\cdot j(a_6)\right)$ is divisible by 3.
\end{proposition}
\begin{proof}
It is equal to $\theta^*\left(\p_{-3}(1)\vac +\frac{3}{2}\pone(x)\pone(1)^2\vac \right)$.
\end{proof}

Next we look at the Chern classes of the tangent sheaves. Since the morphism $\Sigma$ from the defining pullback diagram (\ref{square}) is a submersion, the normal bundle of $\kum{A}{n-1}$ in $A\hilb{n}$ is trivial. Hence $c(\kum{A}{2}) = \theta^* c(A\hilb{3})$. Looking in \cite[Sect.~8]{Generating}, we find a general formula for Chern classes of Hilbert schemes of points on surfaces. So we deduce
\begin{align*}
c_2(A\hilb{3}) &= \left(\tfrac{3}{2} \mathfrak{q}_{(1,1)}(\Delta 1) \mathfrak{q}_1(1) -\tfrac{1}{3} \mathfrak{q}_3\right)\vac \\
 &= 10(1\hilb{\bullet}_{(4)})  -2(1\hilb{\bullet}_{(2)})^2 \\
c_4(A\hilb{3}) &= \tfrac{4}{3} \mathfrak{q}_{(1,1,1)}(\Delta 1)\vac = 4 (1\hilb{\bullet}_{(4)})^2. 
\end{align*}



\bibliographystyle{amsplain}
\bibliography{Kummer2Bib.tex}
\end{document}
