
\section{Odd cohomology of \texorpdfstring{$A\hilb{2}$}{the Hilbert scheme of two points}}\label{OddHilb2}
Let $A$ be a smooth compact surface with torsion free cohomology and $A\hilb{2}$ the Hilbert scheme of two points. 
It can be constructed as follows: Consider the direct product $A\times A$. Denote 
$$b: \widetilde{A\times A} \rightarrow A\times A $$ 
the blow-up along the diagonal $\Delta \cong A$ with exceptional divisor $E$.
Let $j: E\rightarrow \widetilde{A\times A}  $ be the embedding. 
%Since the normal bundle of $\Delta$ in $A\times A$ is trivial, we have:
%$$
%E \cong  \Delta\times \mathbb{P}^1.
%$$
The action of $\mathfrak{S}_2$ on $A\times A$ lifts to an action on $\widetilde{A\times A}$. 
We have the pushforward $j_*:H^*(E,\Z)\rightarrow H^*(\widetilde{A\times A} ,\Z) $.

The quotient by the action of $\mathfrak{S}_2$ is 
$ \pi:\widetilde{A\times A} \rightarrow A\hilb{2}$.
Now, $A\hilb{2}$ is a compact complex manifold with torsion-free cohomology,~\cite[Theorem~2.2]{Totaro}.
%By (\ref{classicexact}), there is an exact sequence
%$$
%0 \rightarrow \pi_*(H^k(\widetilde{A\times A,\Z})) \rightarrow H^k(A\hilb{2},\Z) \rightarrow \left(\frac{\Z}{2{\Z}}\right)^{\oplus\alpha_k}\rightarrow 0
%$$
%with $k\in \left\{1,...,8\right\}$.
In this section, we want to prove the following proposition.
\begin{proposition} \label{Alpha35}
Let $A$ be a smooth compact surface with torsion free cohomology.
Then
\begin{itemize}
\item[(i)]
$H^{3}(A\hilb{2},\Z)=\pi_*(b^{*}(H^{3}(A\times A,\Z)))\oplus \pi_*j_*b_{|E}^{*}(H^1(\Delta,\Z))$,
\item[(ii)]
$H^{5}(A\hilb{2},\Z)=\pi_*(b^{*}(H^{5}(A\times A,\Z)))\oplus \frac{1}{2}\pi_*j_*b_{|E}^{*}(H^3(\Delta,\Z))$.
\end{itemize}
%We have:
%$$\alpha_3=0\ \text{ and }\ \alpha_5=4.$$
\end{proposition}
We adopt the following notation.
\begin{notation} \label{TorusClasses}
We denote the generators of $H^1(A,\Z)$ by $a_i$, $1\leq i\leq k$ and their respective duals by $a_i^*\in H^3(A,\Z)$. 
We denote the generator of the top cohomology $H^4(A,\Z)$ by $x \defIs  a_1 \cdots a_{k}$.
A basis of $H^2(A,\Z)$ will be denoted by $(b_i)_{1\leq i \leq q}$.
\end{notation}
The rest of this section is dedicated to the proof of this proposition. 
This proposition is proved using techniques developed in \cite{Lol}, for another approach see \cite{Totaro}.
The proof is organized as follows. First we recall some notions on integral cohomology endowed with the action of an involution in Section \ref{IntegralTools}. 
Then Section \ref{Prelemma} is devoted to calculate the torsion of $H^{3}(A\hilb{2}\smallsetminus E,\Z)$ (Lemma~\ref{3}) using equivariant cohomology techniques. Then this knowledge allow us to deduce $\alpha_3=0$ using the exact sequence (\ref{exactutile}) and $\alpha_5=4$ using the unimodularity of the lattice $H^{3}\left(\widetilde{A\times A},\Z\right)\oplus H^{5}\left(\widetilde{A\times A},\Z\right)$.
\subsection{Integral cohomology under the action of an involution}\label{IntegralTools}
The main references of this subsection are \cite{Lol} and \cite{BNS}.

Let $G=\left\langle \iota\right\rangle$ be the group generated by an involution $\iota$ on a complex manifold $X$.
As denoted in \cite[Section 5]{BNS}, let $\mathcal{O}_{K}$ be the ring $\Z$ with the following $G$-module structure:
$\iota\cdot x=-x$ for $x\in \mathcal{O}_{K}$. For $a\in \Z$, we also denote by $(\mathcal{O}_{K},a)$ the module $\Z\oplus\Z$ whose $G$-module structure is defined by $\iota\cdot(x,k)=(-x+ka,k)$. We also denote by $N_{2}$ the $\mathbb{F}_{2}[G]$-module $(\mathcal{O}_{K},a)\otimes\mathbb{F}_{2}$.
We recall Definition-Proposition 2.2.2 of \cite{Lol}.
\begin{defipro}\label{defiprop}
Assume that $H^{*}(X,\Z)$ is torsion-free. Then for all $0\leq k \leq 2\dim X$, we have an isomorphism of $\Z[G]$-modules:
$$H^{k}(X,\Z)\simeq \bigoplus_{i=1}^{r}(\mathcal{O}_{K},a_{i})\oplus \mathcal{O}_{K}^{\oplus s}\oplus\Z^{\oplus t},$$
for some odd numbers $a_{i}$ and $(r,s,t)\in\mathbb{N}^3$.
We get the following isomorphism of $\mathbb{F}_{2}[G]$-modules:
$$H^{k}(X,\mathbb{F}_{2})\simeq N_{2}^{\oplus r}\oplus\mathbb{F}_{2}^{\oplus (s+t)}.$$
We denote $l_{2}^k(X):=r$, $l_{1,-}^k(X):=s$, $l_{1,+}^k(X):=t$, $\mathcal{N}_{2}:=N_{2}^{\oplus r}$ and $\mathcal{N}_{1}:=\mathbb{F}_{2}^{\oplus s+t}$.
\end{defipro}
\begin{rmk}
These invariants are uniquely determined by $G$, $X$ and $k$.
\end{rmk}
%We recall an adaptation of Proposition 5.1 and Corollary 5.8 of \cite{BNS} that can be found in Section 2.2 of \cite{Lol}.
\begin{prop}\cite[Sect.2~2]{Lol}\label{sarti}
Let $X$ be a compact complex manifold of dimension $n$ and $\iota$ an involution. Assume that $H^{*}(X,\Z)$ is torsion free.
We have:
\begin{itemize}
\item[(i)]
$\rk H^{k}(X,\Z)^{\iota}=l_{2}^k(X)+l_{1,+}^k(X).$
\item[(ii)]
We denote $\sigma:=\id+\iota^*$ and $S^k_\iota:= \Ker \sigma \cap H^{k}(X,\Z)$. 
We have $H^{k}(X,\Z)^{\iota}\cap S^k_\iota=0$ and
$$\frac{H^{k}(X,\Z)}{H^{k}(X,\Z)^{\iota}\oplus S^k_\iota}=\left(\frac{\Z}{2\Z}\right)^{\oplus l_2^k(X)}.$$
\end{itemize}
\end{prop}
\begin{rmk}\label{x+ix}
Note that the elements of $(\mathcal{O}_{K},a_{i})^{\iota}$ are written $x+\iota^{*}(x)$ with $x\in (\mathcal{O}_{K},a_{i})$.
\end{rmk}
Let $\pi: X\rightarrow X/G$ be the quotient map. 
We denote by $\pi^{*}$ and $\pi_{*}$ respectively the pull-back and the push-forward along $\pi$. We recall that 
\begin{equation}
\pi_*\circ \pi^*=2\id \text{ and } \pi^*\circ\pi_* =\id+\iota^*.
\label{pietpi}
\end{equation}
Assuming that $H^{k}(X,\Z)$ is torsion free, we obtain the exact sequence of Proposition 3.3.3 of \cite{Lol}, which will be useful in the next section.
\begin{equation}
\xymatrix{ 0\ar[r]&\pi_{*}(H^{k}(X,\Z))\ar[r] & H^{k}(X/G,\Z)/\tors\ar[r] & (\Z/2\Z)^{\alpha_{k}}\ar[r]& 0,}
\label{classicexact}
\end{equation}
with $\alpha_{k}\in \mathbb{N}$.
We also recall the commutativity behaviour of $\pi_*$ with the cup product.
\begin{prop}\cite[Lemma 3.3.7]{Lol}\label{commut}
Let $X$ be a compact complex manifold of dimension $n$ and $\iota$ an involution. Assume that $H^{*}(X,\Z)$ is torsion free.
Let $0\leq k \leq 2n$, $m$ an integer such that $km\leq 2n$, and let $(x_{i})_{1\leq i \leq m}$ be elements of $H^{k}(X,\Z)^{\iota}$.
Then $$\pi_{*}(x_{1})\cdot...\cdot \pi_{*}(x_{m})=2^{m-1}\pi_{*}(x_{1}\cdot...\cdot x_{m}).$$
\end{prop}
\subsection{Preliminary lemmas}\label{Prelemma}
%Fist we need to calculate the following invariant:
We denote $V=\widetilde{A\times A}\smallsetminus E$ and $U=V/\mathfrak S_{2}=A\hilb{2}\smallsetminus E$, where $\mathfrak{S}_{2}=\left\langle \sigma_{2}\right\rangle$. 
\begin{lemma}\label{1}
We have: $H^{k}(A\times A,\Z)=H^{k}(V,\Z)$
for all $k\leq 3$.
\end{lemma}
\begin{proof}
We have $V\cong A\times A\smallsetminus \Delta$,
so we get the following natural exact sequence:
$$\xymatrix{ \cdots\ar[r]&H^{k}(A\times A,V,\Z)\ar[r] & H^{k}(A\times A,\Z)\ar[r] & H^{k}(V,\Z)\ar[r]& \cdots}$$
Moreover, by Thom isomorphism $H^{k}(A\times A,V,\Z)=H^{k-4}(\Delta,\Z)=H^{k-4}(A,\Z)$.
Hence $H^{k}(A\times A,V,\Z)=0$ for all $k\leq 3$.
Hence $H^{k}(A\times A,\Z)=H^{k}(V,\Z)$
for all $k\leq 2$. It remains to consider the following exact sequence:
$$\xymatrix{ 0\ar[r]&H^{3}(A\times A,\Z)\ar[r]&H^{3}(V,\Z)\ar[r] & H^{4}(A\times A,V,\Z)\ar[r]^{\rho} & H^{4}(A\times A,\Z)}.$$
The map $\rho$ is given by $\Z \left[\Delta\right] \rightarrow H^{4}(A\times A,\Z)$.
Using Notation~\ref{TorusClasses}, the class $x\otimes 1$ is also in $H^{4}(A\times A,\Z)$ and intersects $\Delta$ in one point.
Hence the class of $\Delta$ in $H^{4}(A\times A,\Z)$ is not trivial and the map $\rho$ is injective.
%Moreover, we know by K�nneth formula that:
%$$H^{4}(A\times A,\Z)=H^{0}(A,\
It follows that 
\begin{equation*}
H^{3}(A\times A,\Z)=H^{3}(V,\Z).
\qedhere
\end{equation*}
\end{proof}
Now we will calculate the invariant $l_{1,-}^{2}(A\times A)$ and $l_{1,+}^{1}(A\times A)$ from Definition-Proposition~\ref{defiprop}.

\begin{lemma}\label{2}
We have: $l_{1,-}^{2}(A\times A)=l_{1,+}^{1}(A\times A)=0$.
\end{lemma}
\begin{proof}
By K�nneth formula we have:
$$H^{1}(A\times A,\Z)=H^{0}(A,\Z)\otimes H^{1}(A,\Z)\oplus H^{1}(A,\Z)\otimes H^{0}(A,\Z).$$
The elements of $H^{0}(A,\Z)\otimes H^{1}(A,\Z)$ and $H^{1}(A,\Z)\otimes H^{0}(A,\Z)$ are exchanged under the action of $\sigma_2$. It follows that $l_{2}^{1}(A\times A)=4$ and necessary $l_{1,-}^{1}(A\times A)=l_{1,+}^{1}(A\times A)=0$.
Using K�nneth again, we get: 
\begin{align*}
H^{2}(A\times A,\Z)&=H^{0}(A,\Z)\otimes H^{2}(A,\Z)\oplus H^{1}(A,\Z)\otimes H^{1}(A,\Z)\\
&\oplus H^{2}(A,\Z)\otimes H^{0}(A,\Z).
\end{align*}
As before, elements $x\otimes y\in H^{2}(A\times A,\Z)$ are sent to $y\otimes x$ by the action of $\sigma_2$. Such an element is fixed by the action of $\sigma_2$ if $x=y$. It follows:
$$l_{2}^{2}(A\times A)=6+6=12,$$
$$l_{1,+}^{2}(A\times A)=4,$$
and thus:
\begin{equation*}
l_{1,-}^{2}(A\times A)=0.
\qedhere
\end{equation*}
\end{proof}
\begin{lemma}\label{3}
The group $H^{3}(U,\Z)$ is torsion free.
\end{lemma}
\begin{proof}
Using the spectral sequence of equivariant cohomology, it follows from Proposition 3.2.5 of~\cite{Lol}, Lemma~\ref{1} and~\ref{2}.
\end{proof}
\subsection{Third cohomology group}\label{=}
%Now, we prove that $\alpha_3=0$.
By Theorem 7.31 of~\cite{Voisin}, we have:
\begin{equation}
H^{3}(\widetilde{A\times A},\Z)=b^{*}(H^{3}(A\times A,\Z))\oplus j_*b_{|E}^{*}(H^{1}(\Delta,\Z)).
\label{voisin1}
\end{equation}
It follows that $$H^{3}(A^{[2]},\Z)\supset \pi_{*}b^{*}(H^{3}(A\times A,\Z))\oplus \pi_{*}j_*b_{|E}^{*}(H^{1}(\Delta,\Z)).$$
We want to show that this inclusion is an equality. We will proceed as follows: We first prove that $\pi_{*}b^{*}(H^{3}(A\times A,\Z))$ is primitive. Then, in Lemma~\ref{primitive3}, we show that $\pi_{*}j_*b_{|E}^{*}(H^{1}(\Delta,\Z))$ is primitive and finally we remark that this implies that the direct sum $\pi_*b^{*}(H^{3}(A\times A,\Z))\oplus \pi_*j_*b_{|E}^{*}(H^{1}(\Delta,\Z))$ is primitive.  

It follows from the K\"unneth formula that
all elements in $H^{3}(A\times A,\Z)^{\mathfrak{S}_2}$ are written as $x+\sigma_{2}^{*}(x)$ with $x\in H^{3}(A\times A,\Z)$.
Since $\frac{1}{2}\pi_{*}(x+\sigma_{2}^{*}(x))=\pi_{*}(x)$, it follows that $\pi_{*}(b^*(H^{3}(A\times A,\Z)))$ is primitive in $H^{3}(A^{[2]},\Z)$.
Moreover by (\ref{voisin1}), we have the following values which will be used in Section~\ref{d5}:
\begin{equation}
l_2^3(\widetilde{A\times A})=\rk H^{3}(A\times A,\Z)^{\mathfrak{S}_2}=28.
\label{l1}
\end{equation}
and
$$l_{1,+}^3(\widetilde{A\times A})=\rk H^{1}(\Delta,\Z)^{\mathfrak{S}_2}=4,\ \text{ and }\ l_{1,-}^3(\widetilde{A\times A})=0.$$
\begin{lemma}\label{primitive3}
The group $\pi_{*}j_*b_{|E}^{*}(H^{1}(\Delta,\Z))$ is primitive in $H^{3}(A^{[2]},\Z)$.
\end{lemma}
\begin{proof}
We consider the following commutative diagram:
\begin{equation}
\xymatrix@C=10pt{
\ar[d]^{d\pi^{*}} H^{3}(\mathscr{N}_{A^{[2]}/\pi(E)},\mathscr{N}_{A^{[2]}/\pi(E)}\smallsetminus0,\Z)=H^{3}(A^{[2]},U,\Z)
\ar[r]^-{g}&
H^{3}(A^{[2]},\Z)\ar[d]_{\pi^{*}} \\
H^{3}(\mathscr{N}_{\widetilde{A\times A}/E},\mathscr{N}_{\widetilde{A\times A}/E}\smallsetminus0,\Z)=H^{3}(\widetilde{A\times A},V,\Z)
\ar[r]^-{h}&H^{3}(\widetilde{A\times A},\Z),
}
\label{ThomII}
\end{equation}
where $\mathscr{N}_{A^{[2]}/E}$ and $\mathscr{N}_{\widetilde{A\times A}/E}$ are the normal bundles of $\pi(E)$ in $A^{[2]}$ and of $E$ in $\widetilde{A\times A}$, respectively.
By the proof of Theorem 7.31 of~\cite{Voisin}, the map $h$ is injective with image in $H^{3}(\widetilde{A\times A},\Z)$ given by $j_*b_{|E}^{*}(H^{1}(\Delta,\Z))$. Hence Diagram (\ref{ThomII}) shows that $g$ is also injective and has image $\pi_{*}j_*b_{|E}^{*}(H^{1}(\Delta,\Z))$ in $H^{3}(A^{[2]},\Z)$.
We obtain:
\begin{equation}
\xymatrix{ 0\ar[r]&H^{3}(A^{[2]},U,\Z)\ar[r]^g&H^{3}(A^{[2]},\Z)\ar[r]&H^{3}(U,\Z)}.
\label{exactutile}
\end{equation}
However, by Lemma~\ref{3}, $H^{3}(U,\Z)$ is torsion free; it follows that $\pi_{*}j_*b_{|E}^{*}(H^{1}(\Delta,\Z))$ is primitive in $H^{3}(A^{[2]},\Z)$.
\end{proof}
Now it remains to prove that $\pi_{*}b^{*}(H^{3}(A\times A,\Z))\oplus \pi_{*}j_*b_{|E}^{*}(H^{1}(\Delta,\Z))$ is primitive in $H^{3}(A^{[2]},\Z)$.
This comes from the fact that all elements in $\pi_{*}b^{*}(H^{3}(A\times A,\Z)^{\mathfrak{S}_2})$ are divisible by 2, so the relations (\ref{pietpi}) on $\pi_*$ and $\pi^*$ impose the above sum to be primitive.  

More detailed, let $x\in \pi_*b^{*}(H^{3}(A\times A,\Z))$ and $y\in \pi_*j_*b_{|E}^{*}(H^{1}(\Delta,\Z))$. It is enough to show that if $\frac{x+y}{2}\in H^{3}(A^{[2]},\Z)$, then $\frac{x}{2}\in H^{3}(A^{[2]},\Z)$ and $\frac{y}{2}\in H^{3}(A^{[2]},\Z)$.
As we have seen, we can write $x=\frac{1}{2}\pi_*(z+\sigma_{2}^{*}(z))$, with $z\in b^{*}(H^{3}(A\times A,\Z))$ and $y=\pi_*(y')$, with $y'\in j_*b_{|E}^{*}(H^{1}(\Delta,\Z))$. If $$\frac{\frac{1}{2}\pi_*(z+\sigma_{2}^{*}(z))+\pi_*(y)}{2}\in H^{3}(A^{[2]},\Z)$$ then taking the image by $\pi^{*}$ of this element, we obtain $$\frac{z+\sigma_{2}^{*}(z)}{2}+y'\in H^{3}(\widetilde{A\times A},\Z).$$ Hence $\frac{z+\sigma_{2}^{*}(z)}{2}\in b^{*}(H^{3}(A\times A,\Z))^{\mathfrak{S}_2}$. Hence there is $z'\in b^{*}(H^{3}(A\times A,\Z))$ such that 
$$\frac{z+\sigma_{2}^{*}(z)}{2}=z'+\sigma_{2}^{*}(z').$$
So $x$ is divisible by 2 and then also $y$.  

This finishes the proof of (i) of Proposition~\ref{Alpha35}.
\subsection{The fifth cohomology group}\label{d5}
Now we prove (ii) of Proposition~\ref{Alpha35}.
We will need two basic properties from lattice theory that we recall here and can be found for example in
Chapter 8.2.1 of \cite{Dolgachev}.

Let $M$ be a lattice. Let $L\subset M$ be a sublattice of the same rank. 
Then 
\begin{equation}
|M\DP L|=\sqrt{\frac{\discr L}{\discr M}}.
\label{squareDiscr}
\end{equation}
If $M$ is unimodular and $L\subset M$ is a primitive embedding, then 
\begin{equation}
\discr L = \discr L^\perp.
\label{discrOrthPrim}
\end{equation}

By Theorem 7.31 of~\cite{Voisin}, we have:
\begin{equation}
H^{5}(\widetilde{A\times A},\Z)=b^*(H^{5}(A\times A,\Z))\oplus j_*b_{|E}^{*}(H^{3}(\Delta,\Z)).
\label{voisin2}
\end{equation}
It follows that $$H^{5}(A^{[2]},\Z)\supset \pi_{*}(b^*(H^{5}(A\times A,\Z)))\oplus \pi_{*}j_*b_{|E}^{*}(H^{3}(\Delta,\Z)).$$
As before, by looking at the K�nneth formula, $\pi_{*}(b^*(H^{5}(A\times A,\Z)))$ is primitive in $H^{5}(A^{[2]},\Z)$.
Moreover, by (\ref{voisin2}):
\begin{equation}
l_2^5(\widetilde{A\times A})=\rk H^{5}(A\times A,\Z)^{\mathfrak{S}_2}=28,
\label{l2}
\end{equation}
and 
$$l_{1,+}^5(\widetilde{A\times A})=\rk H^{3}(\Delta,\Z)^{\mathfrak{S}_2}=4,\ \text{ and }\ l_{1,-}^5(\widetilde{A\times A})=0.$$
\begin{lemma}
The lattice $\pi_{*}(H^{3}(\widetilde{A\times A},\Z)\oplus H^{5}(\widetilde{A\times A},\Z))$ has discriminant $2^8$.
\end{lemma}
\begin{proof}
By Proposition~\ref{sarti} (ii):
\footnotesize
\begin{gather*}
\frac{H^{3}(\widetilde{A\times A},\Z)\oplus H^{5}(\widetilde{A\times A},\Z)}{H^{3}(\widetilde{A\times A},\Z)^{\mathfrak{S}_2}\oplus H^{5}(\widetilde{A\times A},\Z)^{\mathfrak{S}_2}\oplus \left(H^{3}(\widetilde{A\times A},\Z)^{\mathfrak{S}_2}\oplus H^{5}(\widetilde{A\times A},\Z)^{\mathfrak{S}_2}\right)^\bot}\hspace{30pt}
\\\hspace{92pt}= \left(\Z/2\Z\right)^{l_2^3(\widetilde{A\times A})+l_2^5(\widetilde{A\times A})}.
\end{gather*}
\normalsize
Since $H^{3}(\widetilde{A\times A},\Z)\oplus H^{5}(\widetilde{A\times A},\Z)$ is a unimodular lattice, 
it follows from (\ref{discrOrthPrim}) and (\ref{squareDiscr}) that 
$$\discr \left[H^{3}(\widetilde{A\times A},\Z)^{\mathfrak{S}_2}\oplus H^{5}(\widetilde{A\times A},\Z)^{\mathfrak{S}_2}\right]=2^{l_2^3(\widetilde{A\times A})+l_2^5(\widetilde{A\times A})}.$$
Then by Proposition~\ref{commut},
\begin{gather*}
\discr \pi_{*}(H^{3}(\widetilde{A\times A},\Z)^{\mathfrak{S}_2}\oplus H^{5}(\widetilde{A\times A},\Z)^{\mathfrak{S}_2})\hspace{55pt}\\
\hspace{62pt}=2^{l_2^3(\widetilde{A\times A})+l_2^5(\widetilde{A\times A})+\rk \left[H^{3}(\widetilde{A\times A},\Z)^{\mathfrak{S}_2}\oplus H^{5}(\widetilde{A\times A},\Z)^{\mathfrak{S}_2}\right]}.
\end{gather*}
Then by Proposition~\ref{sarti} (i):
\begin{equation}\label{odddiscr}
\begin{array}{rl}
\discr \pi_{*}(H^{3}(\widetilde{A\times A},\Z)^{\mathfrak{S}_2}\!\!\!\!&\oplus\; H^{5}(\widetilde{A\times A},\Z)^{\mathfrak{S}_2})\hspace{55pt}\\
\hspace{70pt} =\!\! & 2^{2\left( l_2^3(\widetilde{A\times A})+ l_2^5(\widetilde{A\times A})\right)+l_{1,+}^3(\widetilde{A\times A})+l_{1,+}^5(\widetilde{A\times A})}.
\end{array}
\end{equation}
By Remark~\ref{x+ix} and since $\pi_*(x+\iota^*(x))=2\pi_*(x)$, we have:
<<<<<<< HEAD
$$\frac{\pi_*(H^3(\widetilde{A\times A},\Z)\oplus H^5(\widetilde{A\times A},\Z))}{\pi_*(H^3(\widetilde{A\times A},\Z)^{\mathfrak{S}_2}\oplus H^5(\widetilde{A\times A},\Z)^{\mathfrak{S}_2})}=\left(\frac{\Z}{2\Z}\right)^{\oplus \left(l_2^3(\widetilde{A\times A})+l_2^5(\widetilde{A\times A})\right)}.$$
Then:
=======
$$\frac{\pi_*(H^3(\widetilde{A\times A},\Z)\oplus H^5(\widetilde{A\times A},\Z))}{\pi_*(H^3(\widetilde{A\times A},\Z)^{\mathfrak{S}_2}\oplus H^5(\widetilde{A\times A},\Z)^{\mathfrak{S}_2}}=\left(\frac{\Z}{2\Z}\right)^{\oplus l_2^3(\widetilde{A\times A})+l_2^5(\widetilde{A\times A})}.$$
Then by (\ref{odddiscr}):
>>>>>>> 8a708dd2acd66b5e22d0ab0c0553853683713ace
\begin{equation*}
\discr \pi_{*}(H^{3}(\widetilde{A\times A},\Z)\oplus H^{5}(\widetilde{A\times A},\Z))=2^{l_{1,+}^3(\widetilde{A\times A})+l_{1,+}^5(\widetilde{A\times A})}=2^8.
\qedhere
\end{equation*}
\end{proof}
The lattice $H^{3}(A^{[2]},\Z)\oplus H^{5}(A^{[2]},\Z)$ is unimodular. Hence by (\ref{squareDiscr}):
$$\frac{H^{3}(A^{[2]},\Z)\oplus H^{5}(A^{[2]},\Z)}{\pi_{*}(H^{3}(\widetilde{A\times A},\Z)\oplus H^{5}(\widetilde{A\times A},\Z))}=\left(\frac{\Z}{2\Z}\right)^{\oplus 4}.$$
However, from the last section, we know that $\pi_{*}(H^{3}(\widetilde{A\times A},\Z))=H^{3}(A^{[2]},\Z)$.
It follows that
$$
\frac{H^{5}(A^{[2]},\Z)}{\pi_{*}(H^{5}(\widetilde{A\times A},\Z))}=\left(\frac{\Z}{2\Z}\right)^{\oplus 4}.
$$
Then by the same argument as used in the end of Section~\ref{=}, we can see that the elements in $\frac{H^{5}(A^{[2]},\Z)}{\pi_{*}(H^{5}(\widetilde{A\times A},\Z))}$ are given by $\frac{1}{2}\pi_{*}j_*b_{|E}^{*}(H^{3}(\Delta,\Z))$.