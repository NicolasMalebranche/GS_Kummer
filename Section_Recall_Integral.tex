\section{Recall on the theory of integral cohomology of quotients}\label{IntegralTools}
The main references of this section are \cite{Lol} and \cite{BNS}.

Let $G=\left\langle \iota\right\rangle$ be the group generated by an involution $\iota$ on a complex manifold $X$.
As denoted in \cite[Section 5]{BNS}, let $\mathcal{O}_{K}$ be the ring $\Z$ with the following $G$-module structure:
$\iota\cdot x=-x$ for $x\in \mathcal{O}_{K}$. For $a\in \Z$, we also denote by $(\mathcal{O}_{K},a)$ the module $\Z\oplus\Z$ whose $G$-module structure is defined by $\iota\cdot(x,k)=(-x+ka,k)$. We also denote by $N_{2}$ the $\mathbb{F}_{2}[G]$-module $(\mathcal{O}_{K},a)\otimes\mathbb{F}_{2}$.
We recall Definition-Proposition 2.2.2 of \cite{Lol}.
\begin{defipro}\label{defiprop}
Assume that $H^{*}(X,\Z)$ is torsion-free. Then for all $0\leq k \leq 2\dim X$, we have an isomorphism of $\Z[G]$-modules:
$$H^{k}(X,\Z)\simeq \bigoplus_{i=1}^{r}(\mathcal{O}_{K},a_{i})\oplus \mathcal{O}_{K}^{\oplus s}\oplus\Z^{\oplus t},$$
for some $a_{i}\notin 2\Z$ and $(r,s,t)\in\mathbb{N}^3$.
We get the following isomorphism of $\mathbb{F}_{2}[G]$-modules:
$$H^{k}(X,\mathbb{F}_{2})\simeq N_{2}^{\oplus r}\oplus\mathbb{F}_{2}^{\oplus (s+t)}.$$
We denote $l_{2}^k(X):=r$, $l_{1,-}^k(X):=s$, $l_{1,+}^k(X):=t$, $\mathcal{N}_{2}:=N_{2}^{\oplus r}$ and $\mathcal{N}_{1}:=\mathbb{F}_{2}^{\oplus s+t}$.
\end{defipro}
\begin{rmk}
These invariants are uniquely determined by $G$, $X$ and $k$.
\end{rmk}
We recall an adaptation of Proposition 5.1 and Corollary 5.8 of \cite{BNS} that can be found in Section 2.2 of \cite{Lol}.
\begin{prop}\label{sarti}
Let $X$ be a compact complex manifold of dimension $n$ and $\iota$ an involution. Assume that $H^{*}(X,\Z)$ is torsion free.
We have:
\begin{itemize}
\item[(i)]
$\rk H^{k}(X,\Z)^{\iota}=l_{2}^k(X)+l_{1,+}^k(X).$
\item[(ii)]
We denote $\sigma:=\id+\iota^*$ and $S^k_\iota:= \Ker \sigma \cap H^{k}(X,\Z)$. 
We have $H^{k}(X,\Z)^{\iota}\cap S^k_\iota=0$ and
$$\frac{H^{k}(X,\Z)}{H^{k}(X,\Z)^{\iota}\oplus S^k_\iota}=(\Z/2\Z)^{l_2^k(X)}.$$
\end{itemize}
\end{prop}
\begin{rmk}\label{x+ix}
Note that the elements of $(\mathcal{O}_{K},a_{i})^{\iota}$ are written $x+\iota^{*}(x)$ with $x\in (\mathcal{O}_{K},a_{i})$.
\end{rmk}
Let $\pi: X\rightarrow X/G$ be the quotient map. 
We denote by $\pi^{*}$ and $\pi_{*}$ respectively the pull-back and the push-forward along $\pi$. We recall that 
\begin{equation}
\pi_*\circ \pi^*=2\id \text{ and } \pi^*\circ\pi_* =\id+\iota^*.
\label{pietpi}
\end{equation}
Assuming that $H^{k}(X,\Z)$ is torsion free, we obtain the exact sequence of Proposition 3.3.3 of \cite{Lol}, which will be useful in the next section.
\begin{equation}
\xymatrix{ 0\ar[r]&\pi_{*}(H^{k}(X,\Z))\ar[r] & H^{k}(X/G,\Z)/\tors\ar[r] & (\Z/2\Z)^{\alpha_{k}}\ar[r]& 0,}
\label{classicexact}
\end{equation}
with $\alpha_{k}\in \mathbb{N}$.
We also recall the commutativity behaviour of $\pi_*$ with the cup product.
\begin{prop}\cite[Lemma 3.3.7]{Lol}\label{commut}
Let $X$ be a compact complex manifold of dimension $n$ and $\iota$ an involution. Assume that $H^{*}(X,\Z)$ is torsion free.
Let $0\leq k \leq 2n$, $m$ an integer such that $km\leq 2n$, and let $(x_{i})_{1\leq i \leq m}$ be elements of $H^{k}(X,\Z)^{\iota}$.
Then $$\pi_{*}(x_{1})\cdot...\cdot \pi_{*}(x_{m})=2^{m-1}\pi_{*}(x_{1}\cdot...\cdot x_{m}).$$
\end{prop}
We also recall Definition 3.3.4 of \cite{Lol}. 
\begin{defi}
Let $X$ be a compact complex manifold and $\iota$ be an involution. 
Let $0\leq k\leq 2n$, and assume that $H^{k}(X,\Z)$ is torsion free. 
Then if the map $\pi_{*}:H^{k}(X,\Z)\rightarrow H^{k}(X/G,\Z)/\tors$ is surjective, we say that $(X,\iota)$ is \emph{$H^{k}$-normal}.
\end{defi}
\begin{rmk}\label{Hnormal}
The $H^{k}$-normal property is equivalent to the following property.

For $x\in H^{k}(X,\Z)^{\iota}$, $\pi_{*}(x)$ is divisible by 2 if and only if there exists $y\in H^{k}(X,\Z)$ such that 
$x=y+\iota^{*}(y)$.
\end{rmk}
We also need to recall Definition 3.5.1 of \cite{Lol} about fixed loci.
\begin{defi}\label{negligible}
Let $X$ be a compact complex manifold of dimension $n$ and $G$ an automorphism group of prime order $p$. 
\begin{itemize}
\item[1)]
We will say that $\Fix G$ is negligible if the following conditions are verified:
\begin{itemize}
\item[$\bullet$]
$H^{*}(\Fix G,\Z)$ is torsion-free.
\item[$\bullet$]
$\codim \Fix G\geq \frac{n}{2}+1$.
\end{itemize}
\item[2)]
We will say that $\Fix G$ is almost negligible if the following conditions are verified:
\begin{itemize}
\item[$\bullet$]
$H^{*}(\Fix G,\Z)$ is torsion-free.
\item[$\bullet$]
$n$ is even and $n\geq 4$.
\item[$\bullet$]
$\codim \Fix G =\frac{n}{2}$, and the purely $\frac{n}{2}$-dimensional part of $\Fix G$ is connected and simply connected. We denote the $\frac{n}{2}$-dimensional component by $Z$.
\item[$\bullet$]
The cocycle $\left[Z\right]$ associated to $Z$ is primitive in $H^{n}(X,\Z)$.
\end{itemize}
\end{itemize}
\end{defi}
Now, we are ready to provide Theorem 2.65 of \cite{Lol} which we will be one of the main tools in Part \ref{quotient}.
\begin{thm}\label{utile'}
Let $G=\left\langle \varphi\right\rangle$ be a group of prime order $p=2$ acting by automorphisms on a K�hler manifold $X$ of dimension $2n$. 
We assume:
\begin{itemize}
\item[i)]
$H^{*}(X,\Z)$ is torsion-free,
\item[ii)]
$\Fix G$ is negligible or almost negligible,
\item[iii)]
$l_{1,-}^{2k}(X)=0$ for all $1\leq k \leq n$, and
\item[iv)]
$l_{1,+}^{2k+1}(X)=0$ for all $0\leq k \leq n-1$, when $n>1$.
\item[v)]

$l_{1,+}^{2n}(X)+2\left[\sum_{i=0}^{n-1}l_{1,-}^{2i+1}(X)+\sum_{i=0}^{n-1}l_{1,+}^{2i}(X)\right]
= \sum_{k=0}^{\dim \Fix G}h^{2k}(\Fix G,\Z).$

\end{itemize}
Then $(X,G)$ is $H^{2n}$-normal.
\end{thm}
We will also need a proposition from Section 7 of \cite{BNS} about Smith theory. Let $X$ be a topological space and let $G=\left\langle \iota\right\rangle$ be an involution acting on $X$. 
Let $\sigma:=1+\iota\in \mathbb{F}_{2}[G]$. We consider the chain complex $C_{*}(X)$ of $X$ with coefficients in $\mathbb{F}_{2}$ and its subcomplexes $\sigma C_{*}(X)$. We denote also $X^{G}$ the fixed locus of the action of $G$ on $X$. 
\begin{prop}\label{SmithProp}
\begin{itemize}
\item[(1)] (\cite{Bredon}, Theorem 3.1). There is an exact sequence of complexes:
$$\xymatrix@C=20pt{0\ar[r] &\sigma C_{*}(X)\oplus C_{*}(X^{G})\ar[r]^{\ \ \ \ \ \ f}&C_{*}(X) \ar[r]^{\sigma}&\sigma C_{*}(X) \ar[r]&0
},$$ where $f$ denotes the sum of the inclusions.
\item[(2)] (\cite{Bredon}, (3.4) p.124). There is an isomorphism of complexes:
$$\sigma C_{*}(X)\simeq C_{*}(X/G,X^{G}),$$
where $X^{G}$ is identified with its image in $X/G$.
\end{itemize}
\end{prop}