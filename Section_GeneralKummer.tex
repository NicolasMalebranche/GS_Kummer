\section[Generalized Kummer varieties and the morphism to the Hilbert scheme]{Cohomology of generalized Kummer varieties via Hilbert scheme cohomology %
\sectionmark{Generalized Kummer varieties}}
\sectionmark{Generalized Kummer varieties}
\label{Section_GeneralKummer}
\begin{definition}
Let $A$ be a complex projective torus of dimension $2$ and $A\hilb{n}$, $n\geq 1$, the corresponding Hilbert scheme of points. Denote $\Sigma : A\hilb{n} \rightarrow A$ the summation morphism, a smooth submersion that factorizes via (\ref{HilbertChow}) the Hilbert--Chow morphism $A\hilb{n}\stackrel{\rho}{\rightarrow}\Sym^n(A)\stackrel{\sigma}{\rightarrow} A$. Then the generalized Kummer variety $\kum{A}{n-1}$ is defined as the fiber over $0$:
\begin{equation}\label{square}
\begin{CD}
\kum{A}{n-1} @>\theta >> A\hilb{n}\\
@VVV @VV\Sigma V\\
\{0\} @> >> A
\end{CD}
\end{equation}
\end{definition}
\begin{theorem}~\cite[Theorem 2]{Spanier}\label{torsion}
The cohomology of the generalized Kummer, $H^*(\kum{A}{n-1},\Z)$, is torsion free. 
\end{theorem}
Our first objective is to collect some information about the pullback diagram~(\ref{square}). 
We make use of Notation~\ref{TorusClasses}.

\begin{proposition}\label{KummerClass}
Set $\alpha_i \defIs  \frac{1}{(n-1)!}\kq_{1}(1)^{n-1}\kq_1(a_i)\vac = \G_0(a_i)1$. The class of %the Poincar\'e dual of 
$\kum{A}{n-1}$ in $H^4(A\hilb{n},\Z)$ is given by
$$
%\prod_{i=1}^4 \left(\tfrac{1}{2}\pone(1)^2\pone(\alpha_i)\vac\right).
[\kum{A}{n-1}]=\alpha_1\cdot\alpha_2\cdot\alpha_3\cdot\alpha_4.
$$ 
\end{proposition}
\begin{proof}
Since the generalized Kummer variety is the fiber over a point, its 
%Poincar\'e dual 
class must be the pullback of $x\in H^4(A)$ under $\Sigma$. But $\Sigma^* (x) = \Sigma^*(a_1)\cdot \Sigma^*(a_2)\cdot \Sigma^*(a_3)\cdot \Sigma^*(a_4)$, so we have to verify that $\Sigma^* (a_i) = \alpha_i$. To do this, we want to use the decomposition $\Sigma = \sigma\rho$.
The pullback along $\sigma$ of a class $a\in H^1(A,\Q)$ on $H^1(\Sym^n(A),\Q)$ 
%$\cong H^1(A^n,\Q)^{\mathfrak{S}_n}$ 
is given by $a\otimes 1\otimes \cdots\otimes 1 + \ldots + 1\otimes \cdots\otimes 1\otimes a$. It follows from (\ref{q1primitive}) that $\Sigma^* (a_i) = \frac{1}{(n-1)!}\kq_{1}(1)^{n-1}\kq_1(a_i)\vac $.
\end{proof}
The morphism $\theta$ induces a homomorphism of graded rings
\begin{equation}
\theta^* :H^*(A\hilb{n})\longrightarrow H^*(\kum{A}{n-1})
\end{equation}
and by the projection formula, we have
\begin{equation}
\theta_*\theta^*(\alpha)  = [\kum{A}{n-1}]\cdot\alpha.
\end{equation}

\begin{lemma}\label{petitlemmeenplus}
 Let $\beta\in H^*(K_{n-1}(A),\Q)$. Then there is a class $B\in H^{*}(A\hilb{n},\Q)$ such that 
 $$\theta_*(\beta)=\frac{1}{n^4}B\cdot [\kum{A}{n-1}].$$
\end{lemma}
\begin{proof}
For a point $a\in A$, we denote by $t_a$ the morphism on $A\hilb{n}$ induced by the translation by $a$.
Then we consider the morphism $\Theta :\kum{A}{n-1}\times A \longrightarrow A\hilb{n}$ defined by $\Theta(\xi,a)=t_a(\theta(\xi))$. It fits in a pullback diagram
\begin{equation}
\begin{CD}
\kum{A}{n-1}\times A @>\Theta >> A\hilb{n}\\
@VV\pr_2V @VV\Sigma V\\
A @> n\cdot >> A
\end{CD}
\end{equation}
that realizes $\kum{A}{n-1}\times A$ as a $n^4$-fold covering of $A\hilb{n}$ over $A$.
Now, for $\beta\in H^*(K_{n-1}(A),\Q)$ set
$$
B:=\Theta_*(\beta\otimes 1).
$$
Then the projection formula gives
\begin{align*}
B\cdot [K_{n-1}(A)]&= \Theta_*\left(\beta\otimes 1\cdot \Theta^*[\kum{A}{n-1}]\right) \\
&=n^4 \Theta_*\left((\beta\otimes 1)\cdot  (1\otimes x)\right)\\
&=n^4 \Theta_*(\beta \otimes x)\\
&=n^4\theta_*(\beta).
\qedhere
\end{align*}
\end{proof}

\begin{proposition}\label{annihilator}
The kernel of $\theta^*$ is equal to the annihilator of $[\kum{A}{n-1}]$.
\end{proposition}
\begin{proof}
Assume $\alpha\in \ker(\theta^*)$. Then we have
$
[\kum{A}{n-1}]\cdot \alpha = \theta_*\theta^*(\alpha) = 0
$. 
Conversely, if $\alpha\notin \ker(\theta^*)$,
let $\beta\in H^*(\kum{A}{n-1},\Q)$ be the Poincar\'e dual of $\theta^*(\alpha)$, so $\beta\cdot \theta^*(\alpha)\neq 0$.
Then by projection formula:
$
\theta_*(\beta)\cdot \alpha\neq 0.
$
By Lemma~\ref{petitlemmeenplus}, there exists $B\in H^*(A\hilb{n},\Q)$ such that 
$B\cdot [\kum{A}{n-1}]\cdot \alpha\neq 0$. It follows that $ [\kum{A}{n-1}]\cdot \alpha\neq 0$.
\end{proof}

\begin{corollary} \label{KummerEquality}
$\theta^*(\alpha) = \theta^*(\beta)$ if and only if $[\kum{A}{n-1}]\cdot \alpha = [\kum{A}{n-1}]\cdot \beta$. 
\qed
\end{corollary}

\begin{proposition}\label{Annihideal}
The annihilator of $[\kum{A}{n-1}]$ in $H^*(A\hilb{n},\Q)$ is the ideal generated by $H^1(A\hilb{n})$. 
\end{proposition}
First, we need to recall some material on super algebras (see for instance \cite{DeligneMorgan}).
\begin{definition}
Let $V$ be a super vector space and $n\geq 0$. Then the supersymmetric power $\SSym^n(V)$ of $V$ is a super vector space, given by
\begin{gather*}
\SSym^n(V) = \bigoplus_{p+q=n} \Sym^p(V^{+}) \!\otimes\! \Lambda^q(V^{-}), \\
\SSym^n(V)^{+}\! =\! \bigoplus_{\substack{p+q=n \\ q\text{ even} }} \Sym^p(V^{+}) \!\otimes\! \Lambda^q(V^{-}), \ \ 
\SSym^n(V)^{-}\! =\! \bigoplus_{\substack{p+q=n \\ q\text{ odd} }} \Sym^p(V^{+}) \!\otimes\! \Lambda^q(V^{-}).
\end{gather*}
\end{definition}
\begin{remark}
The supersymmetric power $\SSym^n (V)$ can be realized as a quotient of $V^{\otimes n}$ by an action of the symmetric group $\mathfrak S_n$. This action can be described as follows: If $\tau\in \mathfrak S_n$ is a transposition that exchanges two numbers $i<j$, then $\tau$ permutes the corresponding tensor factors in $v_1\otimes  \cdots\otimes v_n$ introducing a sign
$(-1)^{|v_i||v_j|+(|v_i|+|v_j|)\sum_{i<k<j} |v_k|}$.
\end{remark}

Now let $U$ be a vector space over $\Q$ and look at the exterior algebra $H\defIs  \Lambda^* U$. 
Since $H$ is a super vector space, we can construct the supersymmetric power $S^n\defIs  \SSym^n( H)$.
We may identify $S^n$ with the space of $\mathfrak S_n$-invariants in $H^{\otimes n}$ by means of the linear projection operator
$$
\pr : H^{\otimes n} \longrightarrow S^n , \quad \pr = \frac{1}{n!}\sum_{\pi \in\mathfrak S_n} \pi.
$$
The multiplication in $H^{\otimes n}$ induces a multiplication on the subspace of invariants, which makes $S^n$ a supercommutative algebra.

Since $H$ is generated as an algebra by $U=\Lambda^1(U)\subset H$, we may define a homomorphism of algebras:
$$ s : H \longrightarrow S^n, \quad s(u) = \pr( u \otimes 1\otimes\cdots\otimes 1)\text{ for }u\in U, $$
so $S^n$ becomes an algebra over $H$.
\begin{lemma}
\label{SuperFree}
The morphism $s$ turns $S^n$ into a free module over $H$, for $n\geq 1$.
\end{lemma}
\begin{proof}
We start with the tensor power $H^{\otimes n}$ and the ring homomorphism 
$$
\iota : H \longrightarrow H^{\otimes n},\quad h\longmapsto h\otimes 1\otimes\cdots\otimes 1
$$
that makes $H^{\otimes n}$ a free $H$-module. Note that $\pr \iota \neq s$, since $\pr$ is not a ring homomorphism.
(For example, $\pr\iota(h)\neq s(h)$ for any nonzero $h\in\Lambda^2(U)$.)
We therefore modify the $H$-module structure of $H^{\otimes n}$:

For some $u\in U$, denote $u^{(i)} \defIs  1^{\otimes i-1}\otimes u\otimes 1^{\otimes n-i+1} \in H^{\otimes n}$. Then $H^{\otimes n}$ is generated as a $k$-algebra by the elements $\{u^{(i)}\,,\,u\in U\}$. Now consider the ring automorphism
$$
\sigma : H^{\otimes n} \longrightarrow H^{\otimes n}, \quad u^{(1)} \longmapsto u^{(1)} +u^{(2)} + \ldots + u^{(n)}, \quad
u^{(i)} \longmapsto u^{(i)} \text{ for } i>1.
$$
Then we have $\sigma\iota = s$ on $S^n$. On the other hand, if $\{b_i\}$ is a $k$-basis of $V$, then $\{b_i^{(j)},\,j>1\} $ is a $\iota$-basis for $H^{\otimes n}$, and $\{\sigma(b_i^{(j)})\}$ is a $\sigma\iota$-basis for $H^{\otimes n}$.
Now if we project the basis elements, we get a set $\{\pr(\sigma(b_i^{(j)}))\}$ that spans $S^n$. Eliminating linear dependent vectors (this is possible over the rationals), we get a $s$-basis of $S^n$.
\end{proof}


\begin{proof}[Proof of Proposition \ref{Annihideal}]
Set $H=H^*(A,\Q)\cong \Lambda^*(H^1(A,\Q))$ and consider the exact sequence of $H$-modules
$$
0 \longrightarrow 
%H^{\geq 1}(A,\Q)  
J
\longrightarrow H \stackrel{x\cdot}{\longrightarrow} H.
$$
It is clear that $J$ is the ideal in $H$ generated by $H^{1}(A,\Q)$. 
Now denote $J^{(n)}$ the ideal generated by $H^1(\Sym^n(A),\Q)$ in $H^*(\Sym^n(A),\Q)\cong\SSym^n(H)$.
By the freeness result of Lemma~\ref{SuperFree}, tensoring with $\SSym^n(H)$ yields another exact sequence of $H$-modules
$$
0 \longrightarrow {J}^{(n)} \longrightarrow \SSym^n(H) \xrightarrow{\sigma(x)\cdot} \SSym^n(H).
$$
Now let $\mathfrak{H}$ be the operator algebra spanned by products of $\mathfrak d$ and $\q_1(a)$ for $a\in H^*(A)$. Let $\mathfrak C$ be the graded commutative subalgebra of $\mathfrak H$ generated by $\q_1(a)$ for $a\in H^*(A)$. The action of $\mathfrak H$ on $\vac$ gives $\H$ and the action of $\mathfrak C$ on $\vac$ gives $\rho^*(H^*(\Sym^n(A),\Q))\cong \SSym^n(H)$.
By sending $\mathfrak d$ to the identity, we define a linear map $c : \mathfrak H \rightarrow \mathfrak C$. 
Denote $J\hilb{n}$ the ideal generated by $H^1(A\hilb{n},\Q)$ in $H^*(A \hilb n,\Q)$. We claim that for every $\mathfrak y\in \mathfrak H$:
$$
\mathfrak y\vac \in J\hilb{n} \Leftrightarrow c(\mathfrak y)\vac \in J\hilb{n}.
$$
To see this, we remark that $H^1(A \hilb n,\Q) \cong H^1(A ,\Q)  $ and the multiplication with a class in $H^1(A \hilb n,\Q) $ is given by the operator $\mathfrak G_0(a)$ for some $a\in H^1(A ,\Q)$. Due to the fact that $\mathfrak d$ is also a multiplication operator (of degree 2), $\mathfrak G_0(a)$ commutes with $\mathfrak d$. It follows that for $\mathfrak y =\mathfrak G_0(a) \mathfrak r$ we have $c(\mathfrak y) = \mathfrak G_0(a) c(\mathfrak r)$.

Now denote $\mathfrak k$ the multiplication operator with the class $[\kum{A}{n-1}]$. We have:
$
[\mathfrak k, \mathfrak d] = 0.
$
Now let $y\in H^*(A\hilb{n},\Q)$ be a class in the annihilator of $[\kum{A}{n-1}]$. We can write $y= \mathfrak y\vac$ for a $\mathfrak y\in\mathfrak H$. Choose $\tilde y \in \SSym^n (H)$ in a way that $\rho^*(\tilde y) = c(\mathfrak y) \vac$. Then we have:
$$
0=[\kum{A}{n-1}]\cdot y = \mathfrak k\, \mathfrak y \vac =  \mathfrak k \,c(\mathfrak y)\vac = \rho^*(\sigma^*(x) \cdot \tilde y).
$$
Since $\rho^*$ is injective, $\tilde y$ is in the annihilator of $\sigma^*(x)$, so $\tilde y \in J^{(n)}$. It follows that $c(\mathfrak y)\vac$ and $y$ are in the ideal generated by $H^1(A\hilb{n},\Q)$.
\end{proof}

\begin{theorem}\cite[Th\'eor\`eme 4]{Beauville}
$\kum{A}{n-1}$ is a irreducible holomorphically symplectic manifold. In particular, it is simply connected and the canonical bundle is trivial.
\end{theorem}
This implies that $H^2(\kum{A}{n-1},\Z)$ admits an integer-valued nondegenerated symmetric bilinear form (the Beauville--Bogomolov form) $B$ which gives $H^2(\kum{A}{n-1},\Z)$ the structure of a lattice. Looking, for instance, in the useful table from the introduction of~\cite{Rapagnetta}, we know that this lattice is
isomorphic to $U^{\oplus 3}\oplus \left< -2n \right>$, for $n\geq 3$. 
We have the Fujiki formula for $\alpha\in H^2(\kum{A}{n-1},\Z)$:
\begin{equation} \label{fujiki}
%\int_{\kum{A}{n-1}} \alpha^{2n-2} = n\frac{(2n-2)!}{2^{n-1}(n\! - \! 1)!} q(a)^{n-1}
\int_{\kum{A}{n-1}} \alpha^{2n-2} = n\cdot(2n-3)!!\cdot B(\alpha,\alpha)^{n-1}
\end{equation}

\begin{proposition}\label{H2Sur} Assume $n\geq 3$. Then
$\theta^*$ is surjective on $H^2(A\hilb{n},\Z)$.
\end{proposition}
\begin{proof}
By~\cite[Sect.~7]{Beauville}, $\theta^{\ast} : H^2(A\hilb{n},\C) \rightarrow H^2(\kum{A}{n-1},\C)$ is surjective. 
But by Proposition 1 of~\cite{Britze}, the lattice structure of $\im \theta^*$ is the same as of $H^2(\kum{A}{n-1})$, so the image of $H^2(A\hilb{n},\Z)$ must be primitive. The result follows.
\end{proof}
\begin{notation}\label{BasisH2KA}
 We have seen that, for $n\geq 3$,
 $$
 H^2(\kum{A}{n-1},\Z) \cong H^2(A,\Z) \oplus\left<\theta^*(\delta)\right>.
 $$
We denote the injection $ : H^2(A,\Z) \rightarrow H^2(\kum{A}{n-1},\Z)$ by $j$. It can be described by 
$$
j : a \longmapsto \frac{1}{(n-1)!}\theta^*\left(\q_1(a)\q_1(1)^{n-1}\vac\right).
$$ 
Further, we set $e:=\theta^*(\delta)$. Using Notation~\ref{TorusClasses}, we give the following names for classes in $H^2(\kum{A}{n-1},\Z)$:
\begin{align*}
u_1 &:= j(a_1 a_2), & v_1 &:= j(a_1 a_3), & w_1 &:= j(a_1 a_4), \\ 
u_2 &:= j(a_3 a_4), & v_2 &:= j(a_4 a_2), & w_2 &:= j(a_2 a_3),
\end{align*}
These elements form a basis of $H^2(\kum{A}{n-1},\Z)$ with the following intersection relations under the Beauville-Bogomolov form:
\begin{align*}
B(u_1,u_2) &= 1, & B(v_1,v_2) &= 1, & B(w_1,w_2) &= 1,  &
B(e,e)&= -2n,
\end{align*}
and all other pairs of basis elements are orthogonal.
\end{notation}

