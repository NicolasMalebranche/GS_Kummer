\section{Cohomology of the Generalized Kummer fourfold}
Now we come to the special case $n=3$, so we study $\kum{A}{2}$, the Generalized Kummer fourfolds.
\begin{proposition}
The Betti numbers of $\kum{A}{2}$ are:
$
1,\,0,\,7,\,8,\,108,\,8,\,7,\,0,\,1.
$
\end{proposition}
\begin{proof}
This follows from G\"ottsche's formula \cite[page 49]{Gottsche}.
\end{proof}


\begin{notation} We give the following names for classes in $H^2(\kum{A}{2},\Z)$:
\begin{align*}
u_1 &:= j(a_1 a_2), & v_1 &:= j(a_1 a_3), & w_1 &:= j(a_1 a_4), \\ 
u_2 &:= j(a_3 a_4), & v_2 &:= j(a_4 a_2), & w_2 &:= j(a_2 a_3),
\end{align*}
Further, we set $e:=\theta^*(\delta)$.
These elements form a basis of $H^2(\kum{A}{2},\Z)$ with the following intersection relations under the Beauville-Bogomolov form:
\begin{align*}
q(u_1,u_2) &= 1, & q(v_1,v_2) &= 1, & q(w_1,w_2) &= 1,  &
q(e,e)&= -6,
\end{align*}
and all other pairs of basis elements are orthogonal.
\end{notation}
By means of the morphism $\theta^*$, we may express part of the cohomology of $\kum{A}{2}$ in terms of Hilbert scheme cohomology. We have seen in Proposition \ref{H2Sur} that $\theta^*$ is surjective for degree $2$ and (by duality) also in degree $6$. 
The next proposition shows that this also holds true for odd degrees.
\begin{proposition}
A basis of $H^3(A\hilb{3},\Z)$ is given by:
\begin{gather}
\label{A3_1}
\frac{1}{2}\theta^*\left( \q_1(1)^2\q_1(a^*_i)\vac \right), \\
\label{A3_2}
\theta^*\left(\q_2(a_i)\vac\right).
\end{gather}
and a basis of $H^5(A\hilb{3},\Z)$ is given by:
\begin{gather}
\label{A5_1}
\frac{1}{2}\theta^*\left( \q_1(1)\q_2(a^*_i)\vac \right), \\
\label{A5_2}
\frac{2}{3} \theta^*\left( \G_2(a_i)1 \right).
\end{gather}
\end{proposition}
\begin{proof}
We claim that
The classes (\ref{A3_1}) are Poincar\'e dual to (\ref{A5_1}) and the classes (\ref{A3_2}) are Poincar\'e dual to (\ref{A5_2}), so it remains to show, that all of them are integral.
By \cite{QinWang}, (\ref{A3_1}) and (\ref{A3_2}) are integral. By Proposition \ref{A2Basis}, $\frac{1}{2}\q_2(a^*_i)\vac$ is integral. If the operator $ \q_1(1)$ is applied, we get again an integral class, by \cite[Lemma 3.3]{QinWang}.

Further, $2\G_2(a_i)1$ is integral and $[\kum{A}{2}]\cdot 2\G_2(a_i)1$ is divisible by $3$. By Proposition \ref{IntegralityCheck}, $\frac{2}{3} \theta^*\left( \G_2(a_i)1 \right)$ is integral.
\end{proof}



Let us summarize our results on $\theta^*$:
\begin{theorem}
The homomorphism $\theta^* : H^*(A\hilb{3},\Q)\rightarrow H^*(\kum{A}{2},\Q)$ of graded rings is surjective in every degree except $4$. Moreover, the image of $H^4(A\hilb{3},\Q)$ is equal to $\Sym^2(H^2(\kum{A}{2},\Q))$. 
The kernel of $\theta^*$ is the ideal generated by $H^1(A\hilb{3},\Q)$.
%Proof: look at the ranks of H^*(A\hilb{3}):
%Rank H1*H4 = 188
%Rank H1*H5 = 239
%Rank H1*H6 = 196
%Rank H1*H7 = 102
%Rank H1*H8 = 40
\end{theorem}