\subsection{Odd Cohomology of the generalized Kummer fourfold}\label{oddcohoK2}

By means of the morphism $\theta^*$, we may express part of the cohomology of $\kum{A}{2}$ in terms of Hilbert scheme cohomology. We have seen in Proposition~\ref{H2Sur} that $\theta^*$ is surjective for degree $2$ and (by duality) also in degree $6$. 
The next proposition shows that this also holds true for odd degrees.
\begin{proposition}\label{oddcohomology}
A basis of $H^3(\X,\Z)$ is given by:
\begin{gather}
\label{A3_1}
\frac{1}{2}\theta^*\Big( \q_1(a^*_i)\q_1(1)^2\vac \Big), \\
\label{A3_2}
\frac{1}{2}\theta^*\Big(\q_2(a_i)\q_1(1)\vac\Big).
\end{gather}
and a dual basis of $H^5(\X,\Z)$ is given by:
\begin{gather}
\label{A5_1}
 \theta^*\Big( \q_1(a_ia_j)\q_1(a_j^*)\q_1(1) \vac \Big) \ \text{for any } j\neq i, \\
\label{A5_2}
\theta^*\Big( \q_2(a^*_i)\q_1(1)\vac \Big).
\end{gather}
\end{proposition}
\begin{proof}
The classes (\ref{A3_1}) are Poincar\'e dual to (\ref{A5_1}) and the classes (\ref{A3_2}) are Poincar\'e dual to (\ref{A5_2}) by direct computation:
\begin{gather*}
\frac{1}{2}\theta^*\Big( \q_1(a^*_i)\q_1(1)^2\vac \Big)\cdot \theta^*\Big( \q_1(a_ia_j)\q_1(a_j^*)\q_1(1) \vac \Big) \hspace{80pt}
\\ = \frac{1}{2} \theta^*\Big(  \G_0(a^*_i) \q_1(a_ia_j)\q_1(a_j^*)\q_1(1) \vac \Big)\\
 =  \frac{1}{2}[\X]\cdot \q_1(a_ia_j)\q_1(a_j^*)\q_1(a_i^*) = 1, \\
 \frac{1}{2}\theta^*\Big(\q_2(a_i)\q_1(1)\vac\Big)\cdot \theta^*\Big( \q_2(a^*_i)\q_1(1)\vac \Big) =\theta^*\Big( \G_1(a_i)\q_2(a^*_i)\q_1(1)\vac  \Big) \\
 = [\X]\cdot \left( 2\q_3(x) - \q_1(x)^2\q_1(1) \right) \vac = 0-1=-1.
\end{gather*}
It remains to show that all classes are integral.
It is clear from Lemma~\ref{IntegralOperators} that (\ref{A3_1}) is integral, while the integrality of (\ref{A5_1}) and (\ref{A5_2}) is obvious. By Proposition~\ref{A2Basis}, $\frac{1}{2}\q_2(a_i)\vac$ is integral as well. If the operator $ \q_1(1)$ is applied, we get again an integral class.
\end{proof}
%\textcolor{green}{Corollary~\ref{actionH3} to put after Proposition 3.18~\ref{} saying that it will be use in Part 2}
\begin{cor}\label{actionH3}
Let $A$ be an abelian surface and let $g$ be an automorphism of $A$. Let $g^{[[3]]}$ be the automorphism induced by $g$ on $K_2(A)$.
By Proposition~\ref{oddcohomology}, $H^3(K_2(A),\Z)\cong H^1(A,\Z)\oplus H^3(A,\Z)$ and the action of $g^{[[3]]}$ on $H^3(K_2(A),\Z)$ is given by the action of $g$ on $H^1(A,\Z)\oplus H^3(A,\Z)$.
\end{cor}
\begin{proof}
By Proposition \ref{oddcohomology}, we have an isomorphism:
$$f:H^1(A,\Z)\oplus H^3(A,\Z)\rightarrow H^3(K_2(A),\Z),$$
given by $f(a_i)=\frac{1}{2}\theta^*\Big(\q_2(a_i)\q_1(1)\vac\Big)$ and $f(a_i^*)=\frac{1}{2}\theta^*\Big( \q_1(a^*_i)\q_1(1)^2\vac \Big)$, for all $i\in\{1,...,4\}$.
We want to prove that $f\circ g^*=g^{[[3]]*}\circ f$. 
To do so, it is enough to show that $f\circ g^*(a_i)=g^{[[3]]*}\circ f(a_i)$ and $f\circ g^*(a_i^*)=g^{[[3]]*}\circ f(a_i^*)$ for all $i\in\{1,...,4\}$.
Let $g^{[3]}$ be the morphism induced by $g$ on $A^{[3]}$. By definition of $g^{[[3]]}$:
\begin{equation}
g^{[[3]]}\circ \theta=\theta\circ g^{[3]}.
\label{commuteThetag}
\end{equation}
Then by (\ref{commuteThetag}) and by defintion of $g^{[3]}$:
\begin{align*}
 g^{[[3]]*}\circ f(a_i)&=g^{[[3]]*}\circ\theta^*\Big(\q_2(a_i)\q_1(1)\vac\Big)\\
 &=\theta^*\circ g^{[3]*}\Big(\q_2(a_i)\q_1(1)\vac\Big)\\
 &=\theta^*\Big(\q_2(g^*(a_i))\q_1(1)\vac\Big)\\
 &=f\circ g^*(a_i).
\end{align*}
We prove $f\circ g^*(a_i^*)=g^{[[3]]*}\circ f(a_i^*)$ with the same method.


%Let $g^{[3]}$ be the automorphism on $A^{[3]}$ induced by $g$.
%We have for the pullbacks $g^{[3]*}(\q_2(a_i)\q_1(1)\vac)\!=\!\q_2(g^{*}a_i)\q_1(1)\vac$ and $g^{[3]*}(\q_1(a^*_i)\q_1(1)^2\vac)\!=\!\q_1(g^{*}a^*_i)\q_1(1)^2\vac$.
%Moreover, we have by definition $g^{[[3]]*}\circ \theta^{*}=\theta^{*}\circ g^{[3]*}$.
%The result follows from Proposition~\ref{oddcohomology}.
\end{proof}
