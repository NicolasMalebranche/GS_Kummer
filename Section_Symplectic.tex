\subsection{Actions of the symplectic group over finite fields}\label{Section_Symplectic}
The aim of this subsection is to provide some special computations used in Section~\ref{integralbasisH4}.

Let $V$ be a symplectic vector space of dimension $n\in 2\mathbb{N}$ over a field $k$ with a nondegenerate symplectic form $\omega : \Lambda^2 V \rightarrow k$. A line is a one-dimensional subspace of $V$ through the origin, a plane is a two-dimensional subspace of $V$. A plane $P\subset V$ is called isotropic, if $\omega (x,y)=0$ for any $x,y\in P$, otherwise non-isotropic.  The symplectic group $\Sp V$ is the set of all linear maps $\phi : V\rightarrow V$ with the property $\omega(\phi(x),\phi(y)) = \omega(x,y)$ for all $ x,y\in V$.
\begin{proposition}\label{transitively}
The symplectic group $\Sp V$ acts transitively on the set of non-isotropic planes as well as on the set of isotropic planes.
\end{proposition}
\begin{proof}
Given two planes $P_1$ and $P_2$, we may choose vectors $v_1,v_2,w_1,w_2$ such that $v_1,v_2$ span $P_1$ and $w_1,w_2$ span $P_2$ and $\omega(v_1,v_2) =\omega(w_1,w_2)$. We complete $\{v_1,v_2\}$ as well as $\{w_1,w_2\}$ to a symplectic basis of $V$.
Then define $\phi(v_1)=w_1$ and $\phi(v_2)=w_2$. 
It is now easy to see that the definition of $\phi$ can be extended to the remaining basis elements to give a symplectic morphism.
\end{proof}
\begin{remark} \label{simplePlanes}
The set of planes in $V$ can be identified with the simple tensors in $\Lambda^2V$ up to multiples. Indeed, given a simple tensor $v\wedge w \in \Lambda^2 V$, the span of $v$ and $w$ yields the corresponding plane. Conversely, any two spanning vectors $v$ and $w$ of a plane give the same element $v\wedge w$ (up to multiples).
\end{remark}
%\begin{proposition}
%If $\phi\in\Sp V$ acts through multiplication of a scalar, $\phi(v) = \alpha v$, then $\alpha = \pm 1$ (this is immediate from the definition). Moreover, if $\phi(v)\wedge \phi(w) = \alpha v\wedge w$, then $\alpha=1$.
%\end{proposition}
%\begin{proof}
%We may assume that $V$ is two-dimensional, generated by $v$ and $w$. Our condition on $\phi$ reads then $\det\phi = \alpha$. But the condition on %$\phi$ being symplectic is $\det\phi = 1$, because on a two-dimensional vector space there is only one symplectic form up to scalar multiple. 
%\end{proof}
From now on, we assume that $k$ is finite of cardinality $q$.
\begin{remark} \label{PlaneTriple}
 If $k$ is the field with two elements, then the set of planes in $V$ can be identified with the set $\{\{x,y,z\}\;|\;x,y,z\in V\backslash\{0\},\,x+y+z=0\}$. Observe that for such a $\{x,y,z\}$, $\omega(x,y)=\omega(x,y)=\omega(y,x)$ and this value gives the criterion for isotropy.
\end{remark}
\begin{proposition}\label{OrbitesSp}
\begin{align}
&\text{The number of lines in $V$ is }\frac{q^n-1}{q-1}, \\
&\text{the number of planes in $V$ is }\frac{(q^n-1)(q^{n-1}-1)}{(q^2-1)(q-1)}, \\
&\text{the number of isotropic planes in $V$ is }\frac{(q^n-1)(q^{n-2}-1)}{(q^2-1)(q-1)}, \\
&\text{the number of non-isotropic planes in $V$ is }\frac{q^{n-2}(q^n-1)}{q^2-1}.
\end{align}
\end{proposition}
\begin{proof}
A line $\ell$ in $V$ is determined by a nonzero vector. There are $q^n - 1$ nonzero vectors in $V$ and $q-1$ nonzero vectors in $\ell$. A plane $P$ is determined by a line $\ell_1 \subset V$ and a unique second line $\ell_2\in V/\ell_1$. We have $\frac{q^2-1}{q-1}$ choices for $\ell_1$ in $P$. The number of planes is therefore
$$
\frac{ \frac{q^n-1}{q-1} \cdot\frac{q^{n-1}-1}{q-1}}{\frac{q^2-1}{q-1} } = \frac{(q^n-1)(q^{n-1}-1)}{(q^2-1)(q-1)}.
$$
For an isotropic plane we have to choose the second line from $\ell_1^\perp/\ell_1$. This is a space of dimension $n-2$, hence the formula. The number of non-isotropic planes is the difference of the two previous numbers.
\end{proof}

We want to study the free $k$-module $k[V]$ with basis $\{X_i \,|\, i\in V\}$. It carries a natural $k$-algebra structure, given by
$X_i X_j \defIs  X_{i+j}$ with unit $1=X_0$. This algebra is local with maximal ideal $\mathfrak m$ generated by all elements of the form $(X_i-1)$.

We introduce an action of $\Sp (4,k)$ on $k[V]$ by setting $\phi(X_i) = X_{\phi(i)}$. Furthermore, the underlying additive group of $V$ acts on $k[V]$ by $v( X_i) = X_{i+v} =X_iX_v$. 
\begin{definition}
For a line $\ell\subset V$ define $S_\ell \defIs  \sum_{i\in\ell} X_i$. For a vector $0\neq v\in \ell$ we set $S_v\defIs S_\ell$.
\end{definition}
\begin{lemma}\label{SympLemma}
Let $P\subset V$ be a plane and $\ell_1,\ell_2\subset P$ two different lines spanning $P$. Then we have
$$
S_{\ell_1}S_{\ell_2}=\sum_{i\in P}X_i = \sum_{\ell\subset P}S_\ell.
$$
\end{lemma}
\begin{proof}
The first equality is clear. For the second equality observe that every point $i\in P$ is contained in one line, if we count modulo $q$.
\end{proof}

\begin{definition} \label{SymplecticIdeal}
We define two subsets of $k[V]$:
\begin{gather*}
M \defIs  \left\{\sum_{i\in P}X_i \,|\, P\subset V \text{ plane}\right\}, \\
N  \defIs  \left\{\sum_{i\in P}X_i \,|\, P\subset V \text{ non-isotropic plane}\right\}.
\end{gather*}
Let $(M)$ and $(N)$ be the ideals generated by $M$ and $N$, respectively. 
Further, let $D$ be the linear span of $\{v(b) - b \,|\, b\in N, v\in V \}$. Then $D$ is in fact an ideal, namely the product of ideals $\mathfrak m\cdot (N)$.
\end{definition}
\begin{proposition}
We have $(M)=(N)$.
\end{proposition}
\begin{proof}
We have to show that $\sum_{i\in P}X_i \in (N)$ for an isotropic plane $P$. Let $v,w$ be two spanning vectors of $P$ and $u$ a vector with $\omega(u,v)\neq 0$. Denote $P'$ the non-isotropic plane spanned by $u $ and $v$. By Lemma~\ref{SympLemma}, we have
$$
S_uS_vS_w = \sum_{\ell\subset P'} S_{\ell}S_w= \left(S_v+\sum_{\lambda\in k}S_{u + \lambda v}\right)S_w.
$$
Now $w$ spans a non-isotropic plane with every line in $P'$, except one, namely the line that contains $v$. So it follows that
$$
\sum_{i\in P}X_i = S_vS_w = S_uS_vS_w  - \sum_{\lambda\in k} S_{u + \lambda v}S_w, 
$$
and we see that the right hand side is an element of $(N)$.
\end{proof}
For the rest of this section, we assume $\dim_k V=4$. 
\begin{proposition} \label{SymplecticIdealsDimension}
The following table illustrates the dimensions of $(N)$ and $D$ for some $k$.
\vspace{2mm}
\begin{center}
\begin{tabular}{c||c|c|c}
 $k$ & $\dim_k(N)$ & $\dim_k D$ \\
\hline
$\mathbb F_2$   & 11 &  5  \\
$\mathbb F_3$  & 50 & 31  \\
$\mathbb F_5$  &355 &270
\end{tabular}
\end{center}
\end{proposition}
Since this is computed numerically using a naive approach, we do not give a formal proof.
\begin{rmk}\label{c2}
We remark that $X\defIs \sum_{i\in V}X_i\in D$. Indeed, let $P$, $P'$ be two non-isotropic planes with $P \cap P' = 0$. Then $X = \left( \sum_{i\in P}X_i\right)\left(  \sum_{i\in P'}X_i \right)$ and both factors are contained in $(N)\subset \mathfrak m$, so $X\in \mathfrak m \cdot (N) = D$.
\end{rmk}
Let us now consider some special orthogonal sums. Take two vectors $v,w\in V$ with $\omega(v,w)=1$ and set $x\defIs  (v\wedge w)^2\in \Sym^2 (\Lambda^2V)$. Denote $P$ the plane spanned by $v$ and $w$ and set $y\defIs  \sum_{i\in P}X_i\in  k[V]$.

We consider now the action of $\Sp V$ on $\Sym^2 (\Lambda^2V)\oplus k[V]$. 
Denote $O$ the vector space spanned by the elements $\phi(x)\oplus \phi(z),$ for $\phi \in \Sp V,$ $z \in (y)$
and $U$ the vector space spanned by the elements $\phi(x),$ for $\phi \in \Sp V$.
\begin{proposition}\label{CombinedSymplectic}
Then we have by numerical computation:
\vspace{2mm}
\begin{center}
\begin{tabular}{c||c|c}
 $k$ & $\dim_k O$  & $\dim_k U$\\
\hline
$\mathbb F_2$ & 11 & 6 \\
$\mathbb F_3$ & 51  & 20 \\
$\mathbb F_5$ & 375  & 20 
\end{tabular}
\end{center}
\end{proposition}
Now we prove the following lemma that we will need for a divisibility argument in Section~\ref{integralbasisH4}.
\begin{lemme}\label{cleffinclassesdiv}
We assume that $k=\mathbb F_3$. Let $\pr_1: \Sym^2 (\Lambda^2V)\oplus k[V]\rightarrow \Sym^2 (\Lambda^2V)$ and $\pr_2: \Sym^2 (\Lambda^2V)\oplus k[V]\rightarrow k[V]$ be the projections. 
We have: 
\begin{itemize}
\item[(i)]
$\dim \ker\pr_{2|O}=1$.
\item[(ii)]
$\dim \ker\pr_{1|O}=31$.
%and $\ker\pr_{1|O} = D$.
\end{itemize}
\end{lemme}
\begin{proof}
We have $\pr_{1}(O)=U$ and $\pr_{2}(O)=(N)$. Using the dimension tables from Propositions~\ref{CombinedSymplectic} and~\ref{cleffinclassesdiv}, we get
\begin{gather*}
\dim \ker\pr_{1|O} = \dim O - \dim U  = 31,\\
\dim \ker\pr_{2|O} = \dim O - \dim (N) = 1.
%\qedhere
\end{gather*}

\end{proof}