
 \section{Middle cohomology}\label{Middle}
The middle cohomology $H^4(\X,\Z)$ has been studied by Hassett and Tschinkel in \cite{Hassett}. We first recall some of their results,
then we proceed by using $\theta^*$ to give a partial description of $H^4(\X,\Z)$ in terms of the well-understood cohomology of $A\hilb{3}$. 
Finally, we find a basis of $H^4(\X,\Z)$ using the action of the image of monodromy representation.
\subsection{Notation and Hassett--Tschinkel's formulas}
\begin{notation}\label{TheZs}
For each $\tau \in A$, denote $W_\tau$ the Brian\c con subscheme of $A\hilb{3}$ consisting of the elements supported entirely at the point $\tau$. If $\tau\in A[3]$ is a point of three-torsion, $W_\tau$ is actually a subscheme of $\X$. We will also use the symbol $W_\tau$ for the corresponding class in $H^4(\X,\Z)$. Further, set 
$$
W := \sum_{\tau\in A[3]} W_\tau.
$$
For $p\in A$, denote $Y_p$ the locus of all $\{x,y,p\}$ in $\X$. The corresponding class $Y_p \in H^4(\X,\Z)$ is independent of the choice of the point $p$. Then set $Z_\tau := Y_p - W_\tau$ and denote $\Pi$ the lattice generated by all $Z_\tau$, $\tau \in A[3]$.
\end{notation}
\begin{proposition}
Denote by $\Sym := \Sym^2\left(H^2\left(\X,\Z\right)\right) \subset H^4\left(\X,\Z\right) $ the span of products of integral classes in degree $2$.
Then 
$$
\Sym+\Pi \ \subset\  H^4\left(\X,\Z\right)
$$
is a sublattice  of full rank.  
\end{proposition}
\begin{proof}
This follows from \cite[Proposition 4.3]{Hassett}.
\end{proof}

In Section 4 of \cite{Hassett}, one finds the following formula:
\begin{equation}
Z_{\tau}\cdot D_{1}\cdot D_{2}=2\cdot B(D_{1},D_{2}),
\label{ZT}
\end{equation}
for all $D_{1}$, $D_{2}$ in $H^{2}(\X,\Z)$, $\tau\in A[3]$ and $B$ the Beauville-Bogomolov form on $\X$.

\begin{definition}\label{defiPi}
We define $\Pi' := \Pi \cap \Sym^{\perp}$. It follows from (\ref{ZT}) that $\Pi'$ can be described as the span of all classes of the form $Z_\tau -Z_0$ or alternatively as the set of all
$
\sum_\tau \alpha_\tau Z_\tau $, such that $ \sum_\tau \alpha_\tau =0$.
Note that in \cite{Hassett} the symbol $\Pi'$ denotes something different.
\end{definition}
\begin{remark}
Since $\rk \Sym = 28$ and $\rk \Pi' = 80$, the lattice $\Sym\oplus\Pi'\subset H^4\left(\X,\Z\right)$ has full rank.
\end{remark}

\begin{proposition}
The class $W$ can be written with the help of the square of half the diagonal as
\begin{align} 
W &= \theta^*\Big( \q_3(1)\vac \Big) \\
\label{WFormula}
&= 9 Y_p + e^2.
\end{align}
The second Chern class is non-divisible and given by 
\begin{align}
\label{sumZ}
\cc &= \frac{1}{3}\sum_{\tau\in A[3] } Z_\tau \\
\label{ChernY}
&= \frac{1}{3} \Big(72Y_p-e^2 \Big). 
\end{align}
\end{proposition}
\begin{proof} 
In Section 4 of \cite{Hassett} one finds the equations
\begin{gather}
W = \frac{3}{8}\Big(\cc + 3e^2\Big),\label{WW} \\
Y_p = \frac{1}{72}\Big(3\cc + e^2\Big),
\end{gather}
from which we deduce (\ref{WFormula}) and (\ref{ChernY}).
Equation (\ref{sumZ}) and the non-divisibility are from \cite[Proposition 5.1]{Hassett}.
\end{proof}
\subsection{Properties of $\Sym^2H^2(K_2(A),\Z)$ in $H^4(K_2(A),\Z)$}
\begin{proposition}\label{ImSym}
The image of $H^4(A\hilb{3},\Q)$ under the morphism $\theta^*$ is equal to $\Sym^2H^2(\X,\Q)$.
\end{proposition}
\begin{proof}
We start by giving a set of generators of $H^4(A\hilb{n},\Q)$. Theorem 5.30 of \cite{LiQinWang} ensures that it is possible to do this in terms of multiplication operators. To enumerate elements of $H^*(A,\Q)$, we follow Notation \ref{TorusClasses}. Basis elements of $H^2(A,\Q)$ will be denoted by $b_i$ for $1\leq i\leq 6$. Then our set of generators is given by:
\begin{center}
\begin{tabular}{l|c}
multiplication operator & number of classes \\ \hline
$\G_0(a_1)\G_0(a_2)\G_0(a_3)\G_0(a_4) $  & $1$ \\
$\G_0(a_i)\G_0(a_j)\G_0(b_k)$ for $i<j$  & $\binom{4}{2}\cdot 6 = 36$ \\
$\G_0(a_i)\G_0(a_j^*)$ & $4\cdot 4 = 16$ \\
$\G_0(b_i)\G_0(b_j)$ for $i\leq j$& $\binom{6+1}{2}= 21$ \\
$\G_0(x)$ &  $1$  \\ \hline
$\G_0(a_i)\G_0(a_j)\G_1(1)$ for $i<j$ & $\binom{4}{2} = 6$ \\
$\G_0(a_i)\G_1(a_j)$ & $4\cdot 4=16$ \\
$\G_0(b_i)\G_1(1)$ & $6$ \\
$\G_1(b_i)$ & $6$ \\
$\G_1(1)^2 $ & $1$ \\ \hline
$\G_2(1) $ &$1$
\end{tabular}
\end{center}
Any multiplication operator of degree $4$ can be written as a linear combination of these $111$ classes. Likewise, the dimension of $H^4(A\hilb{n},\Q)$ is $111$ for all $n\geq 4$, according to G\"ottsche's formula \cite{Gottsche}. However, for $n=3$, the $8$ classes $\G_0(x)$, $\G_1(b_i)$ and $\G_2(1) $ can be expressed as linear combinations of the others, so we are left with $103$ linearly independent classes that form a basis of $H^4(A\hilb{3},\Q)$. Multiplication with the class $[\X]$ is given by the operator $\G_0(a_1)\G_0(a_2)\G_0(a_3)\G_0(a_4) $ and annihilates every class that contains an operator of the form $\G_0(a_i)$. There are $75$ such classes, so by Proposition \ref{annihilator}, $\ker \theta^*\subset H^4(A\hilb{3},\Q)$ has dimension at least $75$ and $\im\theta^*$ has dimension at most $103-75=28$. However, since the image of $\theta^*$ must contain $\Sym^2H^2(\X,\Q)$, which is $28$-dimensional, equality follows.
\end{proof}


\begin{proposition} \label{ChernSym}
We have:
\begin{equation}
\cc= 4 u_1u_2 + 4v_1v_2 + 4 w_1 w_2 - \frac{1}{3} e^2. 
\end{equation}
In particular, $\cc\in \Sym \otimes\Q $.
\end{proposition}
We shall give two different proofs. The first one uses Nakajima operators, the second one is based on results of \cite{Hassett}.
\begin{proof}[Proof 1]
First note that the defining diagram (\ref{square}) of the Kummer manifold is the pullback of the inclusion of a point, so the normal bundle of $\X$ in $A\hilb{3}$ is trivial. The Chern class of the tangent bundle of $\X$ is therefore given by the pullback from $A\hilb{3}$: $c(\X) = \theta^*\left(c(A\hilb{3})\right)$. Proposition \ref{ImSym} allows now to conclude that $\cc\in \Sym \otimes \Q$.

To obtain the precise formula, we use a result of Boissi\`ere, \cite[Lemma 3.12]{Boissiere}, giving a commutation relation for the Chern character multiplication operator on the Hilbert scheme. We get:
\begin{align*}
c_2(A\hilb{3}) & = 3\q_1(1)\mathfrak L_2(1)\vac - \frac{1}{3} \q_3(1)\vac \\
 & =\frac{8}{3}\q_1(1)\mathfrak L_2(1)\vac - \frac{1}{3} \delta^2 .
\end{align*}
With Corollary \ref{KummerEquality} one shows now, that $\cc$ is given as stated.
\end{proof}
\begin{proof}[Proof 2]
It follows from (\ref{sumZ}) that $\cc\in \Pi'^\perp$, so $\cc \in \Sym\otimes\Q$. Moreover, together with (\ref{ZT}) we get that
\begin{equation*}
\cc\cdot D_{1}\cdot D_{2}=54\cdot B(D_{1},D_{2})
\end{equation*}
for all $D_{1}$, $D_{2}$ in $H^{2}(\X,\Z)$. Using the non-degeneracy of the Poincar\'e pairing and our knowledge about the Beauville--Bogomolov form on $\X$, we can calculate that
$$
\cc= 4 u_1u_2 + 4v_1v_2 + 4 w_1 w_2 - \frac{1}{3} e^2.  \qedhere
$$
\end{proof}


\begin{corollary}\label{Pi'}
The intersection $\Sym\cap \Pi$ is one-dimensional and spanned by $3\cc$. 
\end{corollary}
\begin{proof}
By Proposition \ref{ChernSym} and (\ref{sumZ}), $3\cc\in \Sym\cap \Pi$. Since the ranks of $\Sym$, $\Pi$ and $H^4(\X,\Z)$ are $28$, $81$ and $108$, respectively, the intersection cannot contain more.
\end{proof}

\begin{corollary}\label{Classuvw}
\begin{equation} \label{YSym}
Y_p =  \frac{1}{6}\Big(u_1u_2 + v_1v_2 +  w_1 w_2 \Big).
\end{equation}
\end{corollary}
\begin{remark}\label{afterClassuvw}
Using Nakajima operators, we may write
\begin{gather}
Y_p = \frac{1}{9} \theta^*\Big( \q_1(1)\mathfrak L_2(1) \vac \Big) =  \frac{1}{2}\theta^*\Big(\q_1(x)\q_1(1)^2\vac\Big).
\end{gather}
\end{remark}
Now, we can summary all the divisible classes found in $\Sym$ in the following proposition. 
\begin{proposition}\label{classedivisibleSym}
\begin{itemize}
\item[(i)]
The class $e^2$ is divisible by $3$ and the class $u_1u_2 + v_1v_2+w_1w_2$ is divisible by 6.
\item[(ii)]
For $y\in\{u_1,u_2,v_1,v_2,w_1,w_2\}$, the class 
$
e \cdot y
$
is divisible by $3$ and 
$
 y^2 - \frac{1}{3} e\cdot y
$
is divisible by $2$.
\end{itemize}
\end{proposition}
\begin{proof}
\begin{itemize}
\item[(i)]
From Proposition \ref{ChernSym} we see that
$e^2$ is divisible by $3$ and by Corollary \ref{Classuvw}
the class $u_1u_2 + v_1v_2+w_1w_2$ is divisible by 6.
\item[(ii)]
We have $y= \theta^*\left(\q_1(a)\q_1(1)^2\vac\right)$ for some $a\in H^2(A,\Z)$. A computation yields:
$$
e\cdot y = 3\cdot \theta^*\Big( \q_2(a)\q_1(1)\vac \Big)
\quad \text{and}\quad
y^2 = \theta^*\Big(\q_1(a)^2\q_1(1)\vac\Big)
$$
so $e\cdot y$ is divisible by $3$. Furthermore, by Corollary \ref{IntegralOperatorsTorus}, the class 
$
\frac{1}{2} \q_1(a)^2\q_1(1)\vac - \frac{1}{2} \q_2(a)\q_1(1)\vac 
$
is contained in $H^4(A\hilb{3},\Z)$, so its pullback
$
 \frac{1}{2}y^2 - \frac{1}{6} e\cdot y
$
is an integral class, too.
\end{itemize}
\end{proof}
\subsection{Integral basis of $H^4(K_2(A),\Z)$}

From the intersection properties $Z_\tau \cdot Z_{\tau'} = 1$ for $\tau\neq \tau'$ and $Z_\tau^2 = 4$ from Section 4 of \cite{Hassett}, we compute
\begin{equation}
 \discr \Pi' = 3^{84}.
 \label{discrPi}
\end{equation}
On the other hand, a formula developed in \cite{Kapfer} evaluates
\begin{equation} \label{DiscrSym}
\discr \Sym \ = 2^{14}\cdot 3^{38},
\end{equation}
so the lattices $\Pi'$ cannot be primitive. Denote $\Sym^{sat}$ and $\Pi'^{sat}$ the respective primitive overlattices of $\Sym$ and $\Pi'$. $\Sym \oplus\Pi'$ is a sublattice of $H^4(\X,\Z)$ of index $2^{7}\cdot 3^{61}$ and we claim that $\Sym^{sat}\oplus \Pi'^{sat}$ has index $3^{22}$. 
We have already found
\begin{itemize}
 \item $7$ linearly independent classes in $\Sym$ divisible by $2$,
 \item $8$ linearly independent classes in $\Sym$ divisible by $3$,
\end{itemize}
To obtain a basis of $H^4(\X,\Z)$, 
we are now going to find
\begin{itemize}
 \item $31$ linearly independent classes in $\Pi'$ divisible by $3$ and
 \item $20$ linearly independent classes in $\Sym^{sat}\oplus \Pi'^{sat}$, one divisible by $3^3$ and $19$ divisible by $3$.
\end{itemize}


%Now we come to $\Pi'$. 
The first thing to note is that $\Pi'$ is defined topologically for all deformations of $\X$ and the primitive overlattice of $\Pi'$ is a topological invariant, too.  
By applying a suitable deformation, we may therefore assume without loss of generality that $A$ is the product of two elliptic curves $A=E_1\times E_2$. 

Hassett and Tschinkel in Proposition 7.1 of \cite{Hassett} provide the class of a Lagrangian plane $P\subset K_2(A)$, which can be expressed as follows:
\begin{equation}
\left[P\right]=\frac{1}{216}(6u_1-3e)^2+\frac{1}{8}\cc-\frac{1}{3}\sum_{\tau\in \Lambda'} Z_{\tau},
\label{LagrangianPlaneClass}
\end{equation}
where $\Lambda'=E_1[3]\times 0\subset A[3]$.
Hence by translating this plane by an element $\tau'\in A[3]$, we obtain another plane $P'$ that can be written:
$$\left[P'\right]=\frac{1}{216}(6u_1-3e)^2+\frac{1}{8}\cc-\frac{1}{3}\sum_{\tau\in \Lambda'+\tau'} Z_{\tau}.$$
By substracting these two expressions, we obtain a first class divisible by 3 in $\Pi'$:
\begin{equation}
 \frac{1}{3}\sum_{\tau\in\Lambda'} \Big(Z_{\tau} - Z_{\tau+\tau'}\Big)
 \label{first31}
\end{equation}

%After \cite[Eq.~(12)]{Hassett}, for a non-isotropic plane $\Lambda \subset A[3]$ and any $\tau_0\in A[3]$, the classes 
%\begin{equation}
% \frac{1}{3}\sum_{\tau\in\Lambda} \Big(Z_{\tau} - Z_{\tau+\tau_0}\Big)
%\end{equation}
%are integral. 
By Proposition \ref{Hassettmonodromy}, the image of the monodromy representation on the $Z_\tau$ contains the symplectic group $\Sp(4,\mathbb F_3)$. We know by Proposition \ref{transitively} that $\Sp(4,\mathbb F_3)$ acts transitively on the non-isotropic planes of $A[3]$. Hence, modulo $\Pi'$, the orbit by $\Sp(4,\mathbb F_3)$ of the classes (\ref{first31}) is a $\mathbb F_3$-vector space naturally isomorphic to $D$ as introduced in Definition \ref{SymplecticIdeal}, so by Proposition \ref{SymplecticIdealsDimension}, we get a subspace of $\Pi'$ of rank $31$ of classes divisible by $3$.
This subspace can be determined by a computer.
\begin{prop}\label{XXXI}
The 31 following classes of $\Pi'$ are divisible by 3 in $H^{4}(K_2(A),\Z)$ and their thirds span a $\mathbb F_3$-vector space of dimension 31 in $\frac{\Pi'^{sat}}{\Pi'}$.
$$\sum_{\tau\in\Lambda} \Big(Z_{\tau} - Z_{\tau+\tau'}\Big), \text{ with }$$
\begin{itemize}
\item[(i)]
$\Lambda=\plan{1\\0\\0\\0}{0\\1\\0\\0} \text{ and } 0\neq \tau'\in P^\perp = \plan{0\\0\\1\\0}{0\\0\\0\\1} $,

\item[(ii)] 
$\Lambda=\plan{0\\0\\1\\0}{0\\0\\0\\1}  \text{ and } 0\neq \tau' \in P^\perp = \plan{1\\0\\0\\0}{0\\1\\0\\0} \setminus \vect{1\\0\\0\\0}$,

\item[(iii)] 
$\Lambda=\plan{1\\0\\0\\1}{0\\1\\2\\1} \text{ and } \tau' \in \left\{ \vect{0\\1\\1\\2},\vect{1\\0\\0\\2},\vect{1\\1\\1\\1},\vect{2\\2\\2\\2} \right\}$,

\item[(iv)] 
$\Lambda=\plan{1\\0\\0\\0}{0\\1\\0\\1} \text{ and } \tau' \in \left\{ \vect{0\\0\\0\\1},\vect{2\\0\\1\\2},\vect{1\\0\\2\\0},\vect{1\\0\\2\\1} \right\}$,
\item[(v)]
$\Lambda=\plan{1\\0\\0\\0}{0\\1\\1\\1} \text{ and } \tau' \in \left\{ \vect{0\\0\\1\\1},\vect{1\\0\\0\\1} \right\}$,

\item[(vi)] 
$\Lambda=\plan{1\\0\\1\\1}{0\\1\\0\\1} \text{ and } \tau' \in \left\{ \vect{0\\1\\0\\2},\vect{1\\0\\2\\2} \right\}$,

\item[(vii)]
$\Lambda=\plan{1\\0\\1\\0}{0\\1\\0\\1} \text{ and } \tau' \in \left\{ \vect{0\\1\\0\\2},\vect{1\\0\\2\\0} \right\}$,

\item[(viii)]
$\Lambda=\plan{1\\0\\0\\0}{0\\1\\0\\2} \text{ and } \tau' = \vect{1\\0\\1\\0}$,

\item[(ix)]
$\Lambda=\plan{1\\0\\1\\1}{0\\1\\2\\2} \text{ and } \tau' = \vect{1\\1\\0\\2}$.
\end{itemize}
\end{prop}

Now we are going to find the classes divisible by 3 in $\Sym^{sat}\oplus \Pi'^{sat}$.
The class $Z_0$ is not contained in $\Sym$ nor in $\Pi'$.
It can be written as follows:
$$
Z_0=\frac{\sum_{\tau\in A[3]}Z_\tau-\sum_{\tau\in A[3]}(Z_\tau-Z_0)}{81}
\stackrel{(\ref{sumZ})}{=} \frac{c_2(K_2(A))-\frac{1}{3}\sum_{\tau\in A[3]}(Z_\tau-Z_0)}{27},
$$
where $\frac{1}{3}\sum_{\tau\in A[3]}(Z_\tau-Z_0)$ can be expressed as a linear combination of the 31 classes of Proposition \ref{XXXI} by Remark \ref{c2}.
Hence $Z_0$ is the class in $\Sym^{sat}\oplus \Pi'^{sat}$ divisible by $27$. 
Let us now find the remaining $19$ classes divisible by $3$.


%Hassett and Tschinkel in Proposition 7.1 of \cite{Hassett} provide the class of a Lagrangian plane $P\subset K_2(A)$, which can be expressed as follows:
%$$\left[P\right]=\frac{1}{216}(6u_1-3e)^2+\frac{1}{8}\cc-\frac{1}{3}\sum_{\tau\in \Lambda'} Z_{\tau},$$
%where $\Lambda'=E_1[3]\times 0\subset A[3]$. 
We rearrange (\ref{LagrangianPlaneClass}) a bit using (\ref{WW}):
\begin{align*}
\left[P\right]&=\frac{1}{216}(6u_1-3e)^2+\frac{1}{8}\cc-\frac{1}{3}\sum_{\tau\in \Lambda'} Z_{\tau}\\
&=\frac{36u_1^2+9e^2-36u_1\cdot e}{216} +\frac{W}{3}-\frac{3}{8}e^2-\frac{1}{3}\sum_{\tau\in \Lambda'} Z_{\tau}\\
%&=\frac{36u_1^2-72e^2-36u_1\cdot e}{216} +\frac{W}{3}-\frac{1}{3}\sum_{\tau\in \Lambda'} Z_{\tau}\\
&=\frac{u_1^2-2e^2-u_1\cdot e}{6} +\frac{W}{3}-\frac{1}{3}\sum_{\tau\in \Lambda'} Z_{\tau}.
\end{align*}
By Proposition \ref{classedivisibleSym}, the classes $e^2$ and $u_1\cdot e$ are both divisible by 3 and by (\ref{WW}), $W$ is divisible by 3, so the following class is integral:
$$
\mathfrak{N}:=\frac{u_1^2+\sum_{\tau\in \Lambda'} (Z_{\tau}-Z_0)}{3}
.
$$
From the considerations in Section \ref{monodromyexplication}, we know that the group $\Sp(A[3])\ltimes A[3]\subset \Mon(\Pi)$ is a natural extension of 
$\Sp(H_1(A,\Z))\ltimes A[3]\subset \Mon(H_1(A,\Z))$. Isometries of the image of the monodromy representation on $H_1(A,\Z)$ extend naturally to isometries of the image of the monodromy representation of $H^4(K_2(A),\Z)$ acting on $\Pi$ as describe in Proposition \ref{Hassettmonodromy} and acting on $\Sym$ by commuting with the map $j$ defined in Notation \ref{BasisH2KA}.
Hence the group $\Sp(H_1(A,\Z))\ltimes A[3]$ can be seen as a subgroup of $\Mon(H^4(K_2(A),\Z))$. 

Now we will conclude using this monodromy action of $\Sp(H_1(A,\Z))\ltimes A[3]$ on the element $\mathfrak{N}$ and the considerations from Section \ref{Section_Symplectic}. 
%Before, we have to introduce some notation.
%Let denote by $O_1$ the $\Ft$ vector space generated by the elements of the orbit of $u_1^2+\sum_{\tau\in \Lambda'} Z_{\tau}$ under the action of $\Sp(A[3])\ltimes A[3]$. We denote by $O_2$ the $\Ft$ vector space generated by the orbit, under the action of $\Sp(A[3])$, of the elements $u_1^2+\sum_{\tau\in\Lambda'} (Z_{\tau} - Z_{\tau+\tau_0})$, with $\tau_0\in A[3]$.
%We denote by $\Sym^{o}=\Sym\cap c_2(K_2(A))^{\bot}\otimes $ and the projection $\Pr_1:\Sym^{o} $
Proposition \ref{CombinedSymplectic} states now that
the orbit of $\mathfrak{N}$ under the action of $\Sp(A[3])\ltimes A[3]$ is spanning a space of rank $51$ modulo $\Sym\oplus\Pi'$. However, by Lemma \ref{cleffinclassesdiv}, the intersection of that space with $\Sym^{sat}$ is one-dimensional and the intersection with $\Pi'^{sat}$ has dimension $31$, so we are left with $19$ linearly independent elements which provide 19 elements in $\frac{H^4(K_2(A),\Z)}{\Sym^{sat}\oplus\Pi'^{sat}}$.
These 19 independent classes can be enumerated using a computer.
\begin{prop}\label{XIX}
We use Notation \ref{BasisH2KA}.
The 19 following classes are divisible by 3 in $H^{4}(K_2(A),\Z)$ and their thirds provide a sub-vector space of dimension 19 of $\frac{H^4(K_2(A),\Z)}{\Sym^{sat}\oplus\Pi'^{sat}}$.
\begin{itemize}
\item[(i)]
$u_2^2+\sum_{\tau\in \Lambda} Z_\tau-Z_0$, for $\Lambda = \plan{0\\0\\0\\1}{0\\0\\1\\0}$,
\item[(ii)]
$v_2^2+v_2u_2+u_2^2+\sum_{\tau\in \Lambda} Z_\tau-Z_0$, for $\Lambda= \plan{0\\0\\0\\1}{0\\1\\1\\0}$,
\item[(iii)]
$w_2^2+w_2u_2+u_2^2+\sum_{\tau\in \Lambda} Z_\tau-Z_0$, for $\Lambda= \plan{0\\0\\1\\0}{0\\1\\0\\1}$,
\item[(iv)]
$w_2^2-w_2u_2+u_2^2+\sum_{\tau\in \Lambda} Z_\tau-Z_0$, for  $\Lambda= \plan{0\\0\\1\\0}{0\\1\\0\\2}$,
\item[(v)]
$w_2^2-w_2v_2+w_2u_2+v_2^2+v_2u_2+u_2^2+\sum_{\tau\in \Lambda} Z_\tau-Z_0$, for $\Lambda= \plan{0\\0\\1\\2}{0\\1\\0\\1}$,
\item[(vi)]
$w_1^2+w_1u_2+u_2^2+\sum_{\tau\in \Lambda} Z_\tau-Z_0$, for  $\Lambda= \plan{0\\0\\0\\1}{1\\0\\2\\0}$,
\item[(vii)]
$w_1^2-w_1u_2+u_2^2+\sum_{\tau\in \Lambda} Z_\tau-Z_0$, for $\Lambda= \plan{0\\0\\0\\1}{1\\0\\1\\0}$,
\item[(viii)]
$v_1^2+v_1u_2+u_2^2+\sum_{\tau\in \Lambda} Z_\tau-Z_0$, for $\Lambda= \plan{0\\0\\1\\0}{1\\0\\0\\1}$,
\item[(ix)]
$v_1^2-v_1u_2+u_2^2+\sum_{\tau\in \Lambda} Z_\tau-Z_0$, for $\Lambda= \plan{0\\0\\1\\0}{1\\0\\0\\2}$,
\item[(x)]
$v_1^2+v_1w_1-v_1u_2+w_1^2+w_1u_2+u_2^2+\sum_{\tau\in \Lambda} Z_\tau-Z_0$, for $\Lambda = \plan{0\\0\\1\\2}{1\\0\\0\\2}$,
\item[(xi)]
$v_1^2+v_1w_1-v_1w_2-v_1v_2+v_1u_2+w_1^2+w_1w_2+w_1v_2-w_1u_2+w_2^2-w_2v_2+w_2u_2+v_2^2+v_2u_2+u_2^2+\sum_{\tau\in \Lambda} Z_\tau-Z_0$, for $\Lambda = \plan{0\\0\\1\\2}{1\\1\\0\\1}$,
\item[(xii)]
$v_1^2-v_1w_1+v_1w_2-v_1v_2+v_1u_2+w_1^2+w_1w_2-w_1v_2+w_1u_2+w_2^2+w_2v_2-w_2u_2+v_2^2+v_2u_2+u_2^2+\sum_{\tau\in \Lambda} Z_\tau-Z_0$, for $\Lambda = \plan{0\\0\\1\\1}{1\\2\\0\\1}$,
\item[(xiii)]
$u_1^2+\sum_{\tau\in \Lambda} Z_\tau-Z_0$, for  $\Lambda = \plan{0\\1\\0\\0}{1\\0\\0\\0}$,
\item[(xiv)]
$u_1^2-u_1v_2+v_2^2+ \sum_{\tau\in \Lambda} Z_\tau-Z_0$, for  $\Lambda = \plan{0\\1\\0\\0}{1\\0\\0\\1}$,
\item[(xv)]
$u_1^2+u_1v_2+v_2^2+\sum_{\tau\in \Lambda} Z_\tau-Z_0$, for $\Lambda= \plan{0\\1\\0\\0}{1\\0\\0\\2}$,
\item[(xvi)]
$u_1^2+u_1w_1+w_1^2+\sum_{\tau\in \Lambda} Z_\tau-Z_0$, for $\Lambda = \plan{0\\1\\0\\2}{1\\0\\0\\0}$,
\item[(xvii)]
$u_1^2+u_1w_1-u_1v_2+w_1^2+w_1v_2+v_2^2+\sum_{\tau\in \Lambda} Z_\tau-Z_0$, for $\Lambda = \plan{0\\1\\0\\2}{1\\0\\0\\1}$,
\item[(xviii)]
$u_1^2-u_1w_1+u_1w_2-u_1u_2+w_1^2+w_1w_2-w_1u_2+w_2^2+w_2u_2+u_2^2+\sum_{\tau\in \Lambda} Z_\tau-Z_0$, for  $\Lambda = \plan{0\\1\\0\\1}{1\\0\\1\\0}$,
\item[(xix)]
$u_1^2+u_1v_1-u_1w_1+v_1^2+v_1w_1+w_1^2+\sum_{\tau\in \Lambda} Z_\tau-Z_0$, for $\Lambda = \plan{0\\1\\2\\1}{1\\0\\0\\0}$.
\end{itemize}
\end{prop}
It only remains to check that all the classes above generate $H^4(K_2(A),\Z)$.
Let $\Sym^{over}$ be the overlattice of $\Sym$ obtained by adding all the classes from Proposition \ref{classedivisibleSym}. Hence by (\ref{DiscrSym}) and Proposition \ref{squareDiscr}, the lattice $\Sym^{over}$ has discriminant $3^{22}$. Let $\Pi'^{over}$ be the overlattice of $\Pi'$ obtained by adding all the thirds of the classes of Proposition \ref{XXXI}. Then, by (\ref{discrPi}) and Proposition \ref{squareDiscr}, the lattice $\Pi'^{over}$ has discriminant $3^{22}$. Therefore, the lattice $\Sym^{over}\oplus \Pi'^{over}$ has discriminant $3^{44}$. Finally, let $F$ be the overlattice of $\Sym^{over}\oplus \Pi'^{over}$ obtained by adding $Z_0$ and the thirds of all classes of Proposition \ref{XIX}. Using one more time Proposition \ref{squareDiscr}, we find that $\discr F=1$. Hence necessarily $F=H^4(K_2(A),\Z)$. Moreover, from Proposition \ref{XIX}, we obtain: $$\Sym^{over}=\Sym^{sat} \text{ and } \Pi'^{over}=\Pi'^{sat}.$$ We summarize the description of the integral basis of $H^{4}(K_2(A),\Z)$ in the following theorem.
%\begin{proposition}
%Using Notation \ref{BasisH2KA},
%\item[(i)]
%The following lattice is primitive:
%\item[(ii)]
%\end{proposition}
%These are the 19 classes which were missing. 
\begin{thm}\label{integralbasistheorem}
Let $A$ be an abelian variety. We use Notation \ref{BasisH2KA} and \ref{TheZs}. 
\begin{itemize}
\item[(i)]
Let $\Sym^{sat}$ be the primitive overlattice of $\Sym^2\left(H^2\left(\X,\Z\right)\right)$ in $H^4(K_2(A),\Z)$.
The group $\frac{\Sym^{sat}}{\Sym^2\left(H^2\left(\X,\Z\right)\right)}=(\Z/2\Z)^{7}\oplus(\Z/3\Z)^{8}$ is generated by the elements:
$$\frac{e \cdot y}{3},\ \frac{y^2 - \frac{1}{3} e\cdot y}{2} \text{ for } y\in\{u_1,u_2,v_1,v_2,w_1,w_2\},\ 
\frac{e^2}{3} \text{ and } \frac{u_{1}\cdot u_{2}+v_{1}\cdot v_{2}+w_{1}\cdot w_{2}}{6}.$$
\item[(ii)]
Let $\Pi'$ be the lattice from Definition \ref{defiPi} and let $\Pi'^{sat}$ be the primitive over lattice of $\Pi'$ in $H^4(K_2(A),\Z)$.
The group $\frac{\Pi'^{sat}}{\Pi'\ \ \ }=(\Z/3\Z)^{31}$ is generated by the classes:
$$\frac{1}{3}\sum_{\tau\in\Lambda} \Big(Z_{\tau} - Z_{\tau+\tau'}\Big),
$$
with $\Lambda$ a non-isotropic group and $\tau'\in A[3]$. Moreover a basis of $\frac{\Pi'^{sat}}{\Pi'\ \ \ }$ is provided by the 31 classes described in Proposition \ref{XXXI}. 
\item[(iii)]
We have $$\frac{H^4(K_2(A),\Z)}{\Sym^{sat}\oplus\Pi'^{sat}}=(\frac{\Z}{27\Z})\oplus(\frac{\Z}{3\Z})^{19}.$$
Moreover this group is generated by the class $Z_0$ and the 19 classes described in Proposition \ref{XIX}.
\end{itemize}
\end{thm}
Moreover since $\Sym^{over}=\Sym^{sat}$, from the proofs of Proposition \ref{ChernSym}, \ref{classedivisibleSym} and Remark \ref{afterClassuvw}, we obtain the following corollary.
\begin{corollary}\label{SymSatImage}
The image of $H^4(A\hilb{3},\Z)$ under $\theta^*$ is equal to $\Sym^{sat}$. \qed
\end{corollary}