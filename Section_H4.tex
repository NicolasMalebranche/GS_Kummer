\section{Middle cohomology}
The middle cohomology $H^4(\X,\Z)$ has been studied by Hassett and Tschinkel in \cite{Hassett}. We first recall some of their results,
then we proceed by using $\theta^*$ to give a partial description of $H^4(\X,\Z)$ in terms of the well-understood cohomology of $A\hilb{3}$. 
Finally, we find a basis of $H^4(\X,\Z)$ using the action of the monodromy group.
\begin{notation}
For each $\tau \in A$, denote $W_\tau$ the Brian\c con subscheme of $A\hilb{3}$ supported enitrely at the point $\tau$. If $\tau\in A[3]$ is a point of three-torsion, $W_\tau$ is actually a subscheme of $\X$. We will also use the symbol $W_\tau$ for the corresponding class in $H^4(\X,\Z)$. Further, set 
$$
W := \sum_{\tau\in A[3]} W_\tau.
$$
For $p\in A$, denote $Y_p$ the locus of all $\{x,y,p\}$ in $\X$. The corresponding class $Y_p \in H^4(\X,\Z)$ is independent of the choice of the point $p$. Then set $Z_\tau := Y_p - W_\tau$ and denote $\Pi$ the lattice generated by all $Z_\tau$, $\tau \in A[3]$.
\end{notation}
\begin{proposition}
Denote by $S := \Sym^2\left(H^2\left(\X,\Z\right)\right) \subset H^4\left(\X,\Z\right) $ the span of products of integral classes in degree $2$.
Then 
$$
S+\Pi \ \subset\  H^4\left(\X,\Z\right)
$$
is a sublattice  of full rank.  
\end{proposition}
\begin{proof}
This follows from \cite[Proposition 4.3]{Hassett}.
\end{proof}

\begin{proposition}
The class $W$ can be written with the help of the square of half the diagonal as
\begin{align} 
W &= \theta^*\Big( \q_3(1)\vac \Big) \\
\label{WFormula}
&= 9 Y_p + e^2.
\end{align}
The second Chern class is given by 
\begin{align}
\label{sumZ}
\cc &= \frac{1}{3}\sum_{\tau\in A[3] } Z_\tau \\
\label{ChernY}
&= \frac{1}{3} \Big(72Y_p-e^2 \Big). 
\end{align}
\end{proposition}
\begin{proof} 
In Section 4 of \cite{Hassett} one finds the equations
\begin{gather}
W = \frac{3}{8}\Big(\cc + 3e^2\Big), \\
Y_p = \frac{1}{72}\Big(\cc + e^2\Big),
\end{gather}
from which we deduce (\ref{WFormula}) and (\ref{ChernY}).
Equation (\ref{sumZ}) is from \cite[Proposition 5.1]{Hassett}.
\end{proof}

\begin{proposition} \label{ChernSym}
We have:
\begin{equation}
\cc= 4 u_1u_2 + 4v_1v_2 + 4 w_1 w_2 - \frac{1}{3} e^2. 
\end{equation}
In particular, $\cc\in S $.
\end{proposition}

\begin{corollary}
The intersection $S\cap \Pi$ is one-dimensional and spanned by $\cc$. 
\end{corollary}
\begin{proof}
By Proposition \ref{ChernSym} and (\ref{sumZ}), $\cc\in S\cap \Pi$. Since the ranks of $S$, $\Pi$ and $H^4(\X,\Z)$ are $28$, $81$ and $108$, respectively, the intersection cannot contain more.
\end{proof}

\begin{corollary}
\begin{equation} \label{YSym}
Y_p =  \frac{1}{6}\Big(u_1u_2 + v_1v_2 +  w_1 w_2 \Big).
\end{equation}
\end{corollary}
\begin{remark}
Using Nakajima operators, we may write
\begin{gather}
Y_p = \frac{1}{9} \theta^*\Big( \q_1(1)\mathfrak L_2(1) \vac \Big).
\end{gather}

\end{remark}
\begin{definition}
We set $\Pi' := S^\perp \subset \Pi$. This lattice can be described as the span of all classes of the form $Z_\tau -Z_0$ or alternatively as the set of all
$
\sum_\tau \alpha_\tau Z_\tau $, such that $ \sum_\tau \alpha_\tau =0$.
\end{definition}
From the intersection properties $Z_\tau \cdot Z_{\tau'} = 1$ for $\tau\neq \tau'$ and $Z_\tau^2 = 4$ from Section 4 of \cite{Hassett}, we compute
\begin{equation}
 \discr \Pi' = 3^{84}.
\end{equation}
Since $S$ has discriminant $2^{14}\cdot 3^{38}$, the lattices $S$ and $\Pi'$ cannot be primitive. To obtain a basis of $H^4(\X,\Z)$, we are now going to find
\begin{itemize}
 \item $7$ classes in $S$ divisible by $2$,
 \item $8$ classes in $S$ divisible by $3$,
 \item $31$ classes in $\Pi'$ divisible by $3$ and
 \item $22$ classes in $S\oplus\Pi'$ divisible by $3$.
\end{itemize}

\begin{proposition}
For $y\in\{u_1,u_2,v_1,v_2,w_1,w_2\}$, the class
$
e \cdot y
$
is divisible by $3$ and 
$
 y^2 - \frac{1}{3} e\cdot y
$
is divisible by $2$.
\end{proposition}
\begin{proof}
 
\end{proof}
\begin{proposition}
The class $e^2$ is divisible by $3$.
\end{proposition}
\begin{proof}
Look at Proposition \ref{ChernSym}.
\end{proof}
\begin{proposition}
The class $u_1u_2 + v_1v_2+w_1w_2$ is divisible by 6, by (\ref{YSym}). \qed
\end{proposition}

Now we come to $\Pi'$. For a non-isotropic plane $\Lambda \subset A[3]$ and any $\tau_0\in A[3]$, the classes 
\begin{equation}
 \frac{1}{3}\sum_{\tau\in\Lambda} \Big(Z_{\tau} - Z_{\tau+\tau_0}\Big)
\end{equation}
are integral (cf. (12) of \cite{Hassett}). By the considerations after Definition \ref{SymplecticIdeal}, these give a space of rank $31$ of classes in $\Pi'$ divisible by $3$.

Furthermore, to obtain the remaining $22$ classes in $S\oplus \Pi'$, we ask Gr\'egoire for the argument that I've forgotten.




