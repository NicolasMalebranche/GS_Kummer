
\section{Integral cohomology of the generalized Kummer fourfold}
Now we come to the special case $n=3$, so we study $\kum{A}{2}$, the generalized Kummer fourfolds.
\begin{proposition}
The Betti numbers of $\kum{A}{2}$ are:
$
1,\,0,\,7,\,8,\,108,\,8,\,7,\,0,\,1.
$
\end{proposition}
\begin{proof}
This follows from G\"ottsche's formula~\cite[page 49]{Gottsche}.
\end{proof}

\section{Cohomology of the Generalized Kummer fourfold}
Now we come to the special case $n=3$, so we study $\kum{A}{2}$, the Generalized Kummer fourfolds.
\begin{notation} We give the following names for classes in $H^2(\kum{A}{2},\Z)$:
\begin{align*}
u_1 &:= j(a_1 a_2), & v_1 &:= j(a_1 a_3), & w_1 &:= j(a_1 a_4), \\ 
u_2 &:= j(a_3 a_4), & v_2 &:= j(a_4 a_2), & w_2 &:= j(a_2 a_3),
\end{align*}
Further, we set $e:=\theta^*(\delta)$.
These elements form a basis of $H^2(\kum{A}{2},\Z)$ with the following orthogonality relations:
\begin{align*}
q(u_1,u_2) &= 1, & q(v_1,v_2) &= 1, & q(w_1,w_2) &= 1,  &
q(e,e)&= -6,
\end{align*}
and all other pairs of basis elements are orthogonal.
\end{notation}

Let us summarize our results on $\theta^*$:
\begin{theorem}
The homomorphism $\theta^* : H^*(A\hilb{3},\Q)\rightarrow H^*(\kum{A}{2},\Q)$ of graded rings is surjective in every degree except $4$. Moreover, the image of $H^4(A\hilb{3},\Q)$ is equal to $\Sym^2(H^2(\kum{A}{2},\Q))$. 
The kernel of $\theta^*$ is the ideal generated by $H^1(A\hilb{3},\Q)$.
%Proof: look at the ranks of H^*(A\hilb{3}):
%Rank H1*H4 = 188
%Rank H1*H5 = 239
%Rank H1*H6 = 196
%Rank H1*H7 = 102
%Rank H1*H8 = 40
\end{theorem}

The middle cohomology $H^4(\X,\Z)$ has been studied by Hassett and Tschinkel in \cite{Hassett}. We recall some of their results in Section \ref{HassetTschinkelSection},
then we proceed by using $\theta^*$ to give a partial description of $H^4(\X,\Z)$ in terms of the well-understood cohomology of $A\hilb{3}$ in Section \ref{SyminH4}. 
Finally in Section \ref{integralbasisH4}, we find a basis of $H^4(\X,\Z)$ using the action of the image of monodromy representation.
In order to use monodromy representation, we start by recalling notion of monodromy on abelian surfaces in Section \ref{monodromyexplication}. We will also need some technical calculation related the the action of the symplectic group over finite fields (Section \ref{Section_Symplectic}).
\subsection{A monodromy representation on abelian surfaces and generalized Kummer fourfolds}\label{monodromyexplication}

%\subsection{Homology and Cohomology}\label{monodromyexplication}
%The fundamental group $\pi_1(A,\Z) = H_1(A,\Z)$ is a free $\Z$-module of rank $4$, which is canonically identified with the lattice $\Lambda$. Indeed, the projection of every path in $\C^2$ from $0$ to $v\in \Lambda$ gives a unique element of $\pi_1(A,\Z)$. Conversely, any closed path in $A$ with basepoint $0$ lifts to a unique path in $\C^2$ from $0$ to some $v\in\Lambda$.
%So the first cohomology $H^1(A,\Z)$ is freely generated by four elements, too. Moreover, by~\cite[Sect.~I.1]{Mumford}, the cohomology ring is isomorphic to the exterior algebra
%$$
%H^*(A,\Z) = \Lambda^*(H^1(A,\Z)).
%$$



Let $A$ be an abelian surface. We recall that a \emph{principal polarization} of $A$ is a polarization $L$ such that there exists a basis of $H_1(A,\Z)$, with respect to which the symplectic bilinear form on $H_1(A,\Z)$ induced by $c_1(L)$:
\begin{equation}
\omega_L(x,y)=x\cdot c_1(L)\cdot y,
\label{symplecticprinc}
\end{equation}
is given by the matrix:
$$\left( {\begin{array}{cccc}
   0 & 0 & 1 & 0 \\    0 &  0 & 0 & 1\\ -1 & 0 & 0 & 0\\ 0 & -1 & 0 & 0     
   \end{array} } \right).$$
%We remark that a principal polarization $L$ provides a symplectic bilinear form $\omega_L$ on $H_1(A,\Z)$ as follows:

%for all $x$ and $y$ in $H_1(A,\Z)$.
%TODO: Principal polarization/Jacobians \\
%Let $p$ be a prime number, $n\in\mathbb{N}^{*}$, $\mu_p^n$ the group of $p^n$-th roots of the unity and $\Z_p(1)\defIs \underleftarrow{\lim}\mu_p^n$. 
%We recall that the \emph{Weil pairing} $e_p^L$ can be defined as follows.
%$$e_p^L(x,y)=\varsigma^{-x\cdot c_1(L)\cdot y},$$
%for all $x$, $y$ in $H_1(X,\Z)$ and $\varsigma=(...,e^{\frac{2i\pi}{p^n}},...)$.
We recall the following result. 
\begin{prop}
Let $(A,L)$ be a principally polarized abelian surface. The group $H_1(A,\Z)$ is endowed with the symplectic from $\omega_L$ defined in (\ref{symplecticprinc}). Let $\Mon (H_1(A,\Z))$ be the image of monodromy representations on $H_1(A,\Z)$.
Then $\Mon (H_1(A,\Z))\supset\Sp(H_1(A,\Z))$.
%TODO: Weil pairing for tori
\end{prop}
\begin{proof}
It can be seen as follows.
Let $\mathcal{M}_2$ be the moduli space of curves of genus $2$ and $\mathcal{A}_2$ be the moduli space of principally polarized abelian surfaces.
By the Torelli theorem (see for instance~\cite[Theorem 12.1]{Milne}), we have an injection $J:\mathcal{M}_2\hookrightarrow \mathcal{A}_2$ given by taking the Jacobian of the curve endowed with its canonical polarization. Moreover, the moduli spaces $\mathcal{M}_2$ and $\mathcal{A}_2$ are both of dimension 3. 
%$J(\mathcal{M}_2)=\mathcal{A}_2\smallminus \mathcal{A}_1\times\mathcal{A}_1$.

Now if $\mathscr{C}_2$ is a curve of genus 2, we have by Theorem 6.4 of~\cite{Farb}: 
$$\Mon (H_1(\mathscr{C}_2,\Z))\supset \Sp(H_1(\mathscr{C}_2,\Z)),$$
where the symplectic form on $H_1(\mathscr{C}_2,\Z)$ is given by the cup product. 
Then the result follows from the fact that the lattices $H_1(\mathscr{C}_2,\Z)$ and $H_1(J(\mathscr{C}_2),\Z)$ are isometric.
%, where $H_1(\mathscr{C}_2,\Z)$ is endowed with the cup product and $H_1(J(\mathscr{C}_2),\Z)$ with $\omega_$ with $L$ the principal polarization.
\end{proof}
\begin{rmk}\label{SPA2}
Let $(A,L)$ be a principally polarized abelian surface and $p$ a prime number. The group $H_1(A,\Z)$ tensorized by $\mathbb{F}_p$ can be seen as the group $A[p]$ of points of $p$-torsion on $A$ and the form $\omega_L\otimes\mathbb{F}_p$ provides a symplectic form on $A[p]$. Then $\Mon (A[p])$, the image of the monodromy representation on $A[p]$ contains the group $\Sp(A[p])$. 
\end{rmk}
Now, we are ready to recall Proposition 5.2 of~\cite{Hassett} on the monodromy of the generalized Kummer fourfold.
\begin{prop}\label{Hassettmonodromy}
Let $A$ be an abelian surface and $K_2(A)$ the associated generalized Kummer fourfold. 
The image of the monodromy representation on $\Pi=\left\langle\left. Z_\tau\right|\ \tau\in A[3]\right\rangle$ contains the semidirect product
$\Sp(A[3])\ltimes A[3]$ which acts as follows:
$$f\cdot Z_\tau= Z_{f(\tau)}\ \text{and}\ \tau'\cdot Z_\tau= Z_{\tau+\tau'},$$
for all $f\in \Sp(A[3])$ and $\tau'\in A[3]$.
\end{prop}
\section{Actions of the symplectic group over finite fields}
Let $V$ be a symplectic vector space of dimension $n\in 2\mathbb{N}$ over a field $k$ with a nondegenerate symplectic form $\omega : \Lambda^2 V \rightarrow k$. A line is a one-dimensional subspace ov $V$, a plane is a two-dimensional subspace of $V$. A plane $P\subset V$ is called isotropic, if $\omega (x,y)=0$ for any $x,y\in P$, otherwise non-isotropic.  The symplectic group $\Sp V$ is the set of all linear maps $\phi : V\rightarrow V$ with the property $\omega(\phi(x),\phi(y)) = \omega(x,y)$ for all $ x,y\in V$.
\begin{proposition}
The symplectic group $\Sp V$ acts transitively on the set of non-isotropic planes as well as on the set of isotropic planes.
\end{proposition}
\begin{proof}
Given two planes $P_1$ and $P_2$, we may choose vectors $v_1,v_2,w_1,w_2$ such that $v_1,v_2$ span $P_1$, $w_1,w_2$ span $P_2$ and $\omega(u_1,u_2) =\omega(w_1,w_2)$. We complete $\{v_1,v_2\}$ as well as $\{w_1,w_2\}$ to a symplectic basis of $V$.
Then define $\phi(v_1)=w_1$ and $\phi(v_2)=w_2$. 
It is now easy to see that the definition of $\phi$ can be extended to the remaining basis elements to give a symplectic morphism.
\end{proof}
\begin{remark}
The set of planes in $V$ can be identified with the simple tensors in $\Lambda^2V$ up to multiples. Indeed, given a simple tensor $v\wedge w \in \Lambda^2 V$, the span of $v$ and $w$ yields the corresponding plane. Conversely, any two spanning vectors $v$ and $w$ of a plane give the same element $v\wedge w$ (up to multiples).
\end{remark}
\begin{proposition}
If $\phi\in\Sp V$ acts through multiplication of a scalar, $\phi(v) = \alpha v$, then $\alpha = \pm 1$ (this is immediate from the definition). Moreover, if $\phi(v)\wedge \phi(w) = \alpha v\wedge w$, then $\alpha=1$.
\end{proposition}
\begin{proof}
We may assume that $V$ is two-dimensional, generated by $v$ and $w$. Our condition on $\phi$ reads then $\det\phi = \alpha$. But the condition on $\phi$ being symplectic is $\det\phi = 1$, because on a two-dimensional vector space there is only one symplectic form up to scalar multiple. 
\end{proof}
\begin{remark} \label{PlaneTriple}
 If $k$ is the field with two elements, then the set of planes in $V$ can be identified with the set $\{\{x,y,z\}\;|\;x,y,z\in V\backslash\{0\},\,x+y+z=0\}$. Observe that for such a $\{x,y,z\}$, $\omega(x,y)=\omega(x,y)=\omega(y,x)$ and this value gives the criterion for isotropy.
\end{remark}

\begin{proposition}\label{OrbitesSp}
Assume that $k$ is finite of cardinality $q$.
\begin{align}
&\text{The number of lines in $V$ is }\frac{q^n-1}{q-1}, \\
&\text{the number of planes in $V$ is }\frac{(q^n-1)(q^{n-1}-1)}{(q^2-1)(q-1)}, \\
&\text{the number of isotropic planes in $V$ is }\frac{(q^n-1)(q^{n-2}-1)}{(q^2-1)(q-1)}, \\
&\text{the number of non-isotropic planes in $V$ is }\frac{q^{n-2}(q^n-1)}{q^2-1}.
\end{align}
\end{proposition}
\begin{proof}
A line $\ell$ in $V$ is determined by a nonzero vector. There are $q^n - 1$ nonzero vectors in $V$ and $q-1$ nonzero vectors in $\ell$. A plane $P$ is determined by a line $\ell_1 \subset V$ and a unique second line $\ell_2\in V/\ell_1$. We have $\frac{q^2-1}{q-1}$ choices for $\ell_1$ in $P$. The number of planes is therefore
$$
\frac{ \frac{q^n-1}{q-1} \cdot\frac{q^{n-1}-1}{q-1}}{\frac{q^2-1}{q-1} } = \frac{(q^n-1)(q^{n-1}-1)}{(q^2-1)(q-1)}.
$$
For an isotropic plane we have to choose the second line from $\ell_1^\perp/\ell_1$. This is a space of dimension $n-2$, hence the formula. The number of non-isotropic planes is the difference of the two previous numbers.
\end{proof}

Assume now that $V$ is a four-dimensional vector space over $k=\mathbb F_q$. Consider the free $k$-module $k[V]$ with basis $\{X_i \,|\, i\in V\}$. It carries a natural $k$-algebra structure, given by
$X_i\cdot X_j := X_{i+j}$ with unit $1=X_0$. This algebra is local with maximal ideal $\mathfrak m$ generated by all elements of the form $(X_i-1)$.

We introduce an action of $\Sp (4,k)$ on $k[V]$ by setting $\phi(X_i) = X_{\phi(i)}$. Furthermore, the underlying additive group of $V$ acts on $k[V]$ by $v( X_i) = X_{i+v} =X_iX_v$. 
\begin{definition} \label{SymplecticIdeal}
We define a subset of $k[V]$:
$$
N  := \left\{\sum_{i\in P}X_i \,|\, P\subset V \text{ non-isotropic plane}\right\}.
$$
Denote by $\left< N \right>$ and by $(N)$ the linear span of $N$ and the ideal generated by $N$, respectively. Note that $(N) $ is the linear span of $ \{ v\cdot b \,|\, b\in N, v\in V \}$.
Further, let $D$ be the linear span of $\{v(b) - b \,|\, b\in N, v\in V \}$. Then $D$ is in fact an ideal, namely the product of ideals $\mathfrak m\cdot (N)$.
\end{definition}
The following table illustrates the dimensions of these objects for some fields $k$:
\vspace{2mm}
\begin{center}
\begin{tabular}{c||c|c|c}
 $k$ & $\dim_k \left<N\right>$ & $\dim_k(N)$ & $\dim_k D$ \\
\hline
$\mathbb F_2$ & 10 & 11 &  5  \\
$\mathbb F_3$ & 30 & 50 & 31  \\
$\mathbb F_5$ &121 &355 &270
\end{tabular}
\end{center}
\begin{rmk}\label{c2}
When $k=F_3$, we remark that $\sum_{i\in V}X_i\in D$.
\end{rmk}
Let us now consider some special orthogonal sums.
Set $S:=\Sym^2 (\Lambda^2V)$. Take two vectors $v,w\in V$ with $\omega(v,w)=1$ and set $x:= (v\wedge w)^2\in S$. Denote $P$ the plane spanned by $v$ and $w$ and set $y:= \sum_{i\in P}X_i\in  k[V]$.
We set $Y':=y\cdot \mathfrak{m} = \{\sum_{i\in P} X_{i+j}-X_i\,|\, j\in V \} $.

We consider now the action of $\Sp V$ on $S\oplus k[V]$. 
Denote $O_1$ the vector space spanned by the elements $\phi(x)\oplus \phi(z),$ for $\phi \in \Sp V,$ $z \in (y)$,
denote $O_2$ the vector space spanned by the elements $\phi(x)\oplus \phi(y'),$ for $\phi \in \Sp V,$ $y' \in Y'$
and $U$ the vector space spanned by the elements $\phi(x),$ for $\phi \in \Sp V$.
Then we have:
\vspace{2mm}
\begin{center}
\begin{tabular}{c||c|c|c}
 $k$ & $\dim_k O_1$ & $\dim_k O_2 $  & $\dim_k U$\\
\hline
$\mathbb F_2$ & 11 & 10 &  \\
$\mathbb F_3$ & 51 & 50 & 20 \\
$\mathbb F_5$ & 375 & 289 & 
\end{tabular}
\end{center}
Now, we prove the following lemma.
\begin{lemme}\label{cleffinclassesdiv}
We assume that $k=\mathbb F_3$. Let $\pr_1: S\oplus k[V]\rightarrow S$ and $\pr_2: S\oplus k[V]\rightarrow k[V]$ the projection. 
We have: 
%\begin{itemize}
%\item[(i)]
$\dim \Ker\pr_{2|O_1}=1$.
%\item[(ii)]
%$\dim \Ker\pr_{1|O_1}=31$.
%or 32.
%\end{itemize}
\end{lemme}
\begin{proof}
%\begin{itemize}
%\item[(i)]
We have $\pr_{2}(O_1)=(N)$, so $\dim \pr_{2}(O_1)=50$.
Since $\dim O_1=51$, it follows $\dim \Ker\pr_{2|O_1}=1$. 
%\item[(ii)]
%We have $D\subset O_1$.
%Indeed, we have $x+\sum_{i\in P} X_i$ and $x+\sum_{i\in P} X_{i+j}$ in $O_1$, for all $j\in V$. 
%Hence $\sum_{i\in P} X_{i+j}-X_i\in O_1$.
%Moreover, $\Sp V$ acts transversally on the set of the non-isotropic plans, it follows $D\subset O_1$.
%Hence:
%\begin{equation}
%\dim \Ker\pr_{1|O_{1}}\geq \dim D=31.
%\label{pfff}
%\end{equation}
%Moreover, by definition, $\pr_2(O_2)=D$.
%Since $\dim D=31$ and $\dim O_2=50$, we obtain
%$\dim \Ker \pr_{2|O_2}=19$. 
%Hence, necessarily, $$\dim \pr_{1}(O_2)\geq 19.$$
%Since $\pr_{1}(O_2)=\pr_{1}(O_1)$,
%we obtain:
%$$\dim \pr_{1}(O_1)\geq 19.$$
%It follows 
%$$\dim \Ker\pr_{1|O_1}\leq 51-19=32.$$
%We conclude with (\ref{pfff}).
%%We have $\pr_{1}(O_1)=U.$
%%So $$\dim \pr_{1}(O_1)=20.$$
%%Since $\dim O_1=51$,
%%it follows $\dim \Ker\pr_{1|O_1}=31$.
%\end{itemize}
\end{proof}
\subsection{Recall of Hassett and Tschinkel's results}\label{HassetTschinkelSection} \label{Middle}
\begin{notation}\label{TheZs}
For each $\tau \in A$, denote $W_\tau$ the Brian\c con subscheme of $A\hilb{3}$ consisting of the elements supported entirely at the point $\tau$. If $\tau\in A[3]$ is a point of three-torsion, $W_\tau$ is actually a subscheme of $\X$. We will also use the symbol $W_\tau$ for the corresponding class in $H^4(\X,\Z)$. Further, set 
$$
W := \sum_{\tau\in A[3]} W_\tau.
$$
For $p\in A$, denote $Y_p$ the locus of all $\{x,y,p\}$ in $\X$. The corresponding class $Y_p \in H^4(\X,\Z)$ is independent of the choice of the point $p$. Then set $Z_\tau := Y_p - W_\tau$ and denote $\Pi$ the lattice generated by all $Z_\tau$, $\tau \in A[3]$.
\end{notation}
\begin{proposition}
Denote by $\Sym := \Sym^2\left(H^2\left(\X,\Z\right)\right) \subset H^4\left(\X,\Z\right) $ the span of products of integral classes in degree $2$.
Then 
$$
\Sym+\Pi \ \subset\  H^4\left(\X,\Z\right)
$$
is a sublattice  of full rank.  
\end{proposition}
\begin{proof}
This follows from \cite[Proposition 4.3]{Hassett}.
\end{proof}

In Section 4 of \cite{Hassett}, one finds the following formula:
\begin{equation}
Z_{\tau}\cdot D_{1}\cdot D_{2}=2\cdot B(D_{1},D_{2}),
\label{ZT}
\end{equation}
for all $D_{1}$, $D_{2}$ in $H^{2}(\X,\Z)$, $\tau\in A[3]$ and $B$ the Beauville-Bogomolov form on $\X$.

\begin{definition}\label{defiPi}
We define $\Pi' := \Pi \cap \Sym^{\perp}$. It follows from (\ref{ZT}) that $\Pi'$ can be described as the span of all classes of the form $Z_\tau -Z_0$ or alternatively as the set of all
$
\sum_\tau \alpha_\tau Z_\tau $, such that $ \sum_\tau \alpha_\tau =0$.
Note that in \cite{Hassett} the symbol $\Pi'$ denotes something different.
\end{definition}
\begin{remark}
Since $\rk \Sym = 28$ and $\rk \Pi' = 80$, the lattice $\Sym\oplus\Pi'\subset H^4\left(\X,\Z\right)$ has full rank.
\end{remark}

\begin{proposition}
The class $W$ can be written with the help of the square of half the diagonal as
\begin{align} 
W &= \theta^*\Big( \q_3(1)\vac \Big) \\
\label{WFormula}
&= 9 Y_p + e^2.
\end{align}
The second Chern class is non-divisible and given by 
\begin{align}
\label{sumZ}
\cc &= \frac{1}{3}\sum_{\tau\in A[3] } Z_\tau \\
\label{ChernY}
&= \frac{1}{3} \Big(72Y_p-e^2 \Big). 
\end{align}
\end{proposition}
\begin{proof} 
In Section 4 of \cite{Hassett} one finds the equations
\begin{gather}
W = \frac{3}{8}\Big(\cc + 3e^2\Big),\label{WW} \\
Y_p = \frac{1}{72}\Big(3\cc + e^2\Big),
\end{gather}
from which we deduce (\ref{WFormula}) and (\ref{ChernY}).
Equation (\ref{sumZ}) and the non-divisibility are from \cite[Proposition 5.1]{Hassett}.
\end{proof}
\subsection{Properties of $\Sym^2H^2(K_2(A),\Z)$ in $H^4(K_2(A),\Z)$}\label{SyminH4}
\begin{proposition}\label{ImSym}
The image of $H^4(A\hilb{3},\Q)$ under the morphism $\theta^*$ is equal to $\Sym^2H^2(\X,\Q)$.
\end{proposition}
\begin{proof}
We start by giving a set of generators of $H^4(A\hilb{n},\Q)$. Theorem 5.30 of \cite{LiQinWang} ensures that it is possible to do this in terms of multiplication operators. To enumerate elements of $H^*(A,\Q)$, we follow Notation \ref{TorusClasses}. Basis elements of $H^2(A,\Q)$ will be denoted by $b_i$ for $1\leq i\leq 6$. Then our set of generators is given by:
\begin{center}
\begin{tabular}{l|c}
multiplication operator & number of classes \\ \hline
$\G_0(a_1)\G_0(a_2)\G_0(a_3)\G_0(a_4) $  & $1$ \\
$\G_0(a_i)\G_0(a_j)\G_0(b_k)$ for $i<j$  & $\binom{4}{2}\cdot 6 = 36$ \\
$\G_0(a_i)\G_0(a_j^*)$ & $4\cdot 4 = 16$ \\
$\G_0(b_i)\G_0(b_j)$ for $i\leq j$& $\binom{6+1}{2}= 21$ \\
$\G_0(x)$ &  $1$  \\ \hline
$\G_0(a_i)\G_0(a_j)\G_1(1)$ for $i<j$ & $\binom{4}{2} = 6$ \\
$\G_0(a_i)\G_1(a_j)$ & $4\cdot 4=16$ \\
$\G_0(b_i)\G_1(1)$ & $6$ \\
$\G_1(b_i)$ & $6$ \\
$\G_1(1)^2 $ & $1$ \\ \hline
$\G_2(1) $ &$1$
\end{tabular}
\end{center}
Any multiplication operator of degree $4$ can be written as a linear combination of these $111$ classes. Likewise, the dimension of $H^4(A\hilb{n},\Q)$ is $111$ for all $n\geq 4$, according to G\"ottsche's formula \cite{Gottsche}. However, for $n=3$, the $8$ classes $\G_0(x)$, $\G_1(b_i)$ and $\G_2(1) $ can be expressed as linear combinations of the others, so we are left with $103$ linearly independent classes that form a basis of $H^4(A\hilb{3},\Q)$. Multiplication with the class $[\X]$ is given by the operator $\G_0(a_1)\G_0(a_2)\G_0(a_3)\G_0(a_4) $ and annihilates every class that contains an operator of the form $\G_0(a_i)$. There are $75$ such classes, so by Proposition \ref{annihilator}, $\ker \theta^*\subset H^4(A\hilb{3},\Q)$ has dimension at least $75$ and $\im\theta^*$ has dimension at most $103-75=28$. However, since the image of $\theta^*$ must contain $\Sym^2H^2(\X,\Q)$, which is $28$-dimensional, equality follows.
\end{proof}


\begin{proposition} \label{ChernSym}
We have:
\begin{equation}
\cc= 4 u_1u_2 + 4v_1v_2 + 4 w_1 w_2 - \frac{1}{3} e^2. 
\end{equation}
In particular, $\cc\in \Sym \otimes\Q $.
\end{proposition}
We shall give two different proofs. The first one uses Nakajima operators, the second one is based on results of \cite{Hassett}.
\begin{proof}[Proof 1]
First note that the defining diagram (\ref{square}) of the Kummer manifold is the pullback of the inclusion of a point, so the normal bundle of $\X$ in $A\hilb{3}$ is trivial. The Chern class of the tangent bundle of $\X$ is therefore given by the pullback from $A\hilb{3}$: $c(\X) = \theta^*\left(c(A\hilb{3})\right)$. Proposition \ref{ImSym} allows now to conclude that $\cc\in \Sym \otimes \Q$.

To obtain the precise formula, we use a result of Boissi\`ere, \cite[Lemma 3.12]{Boissiere}, giving a commutation relation for the Chern character multiplication operator on the Hilbert scheme. We get:
\begin{align*}
c_2(A\hilb{3}) & = 3\q_1(1)\mathfrak L_2(1)\vac - \frac{1}{3} \q_3(1)\vac \\
 & =\frac{8}{3}\q_1(1)\mathfrak L_2(1)\vac - \frac{1}{3} \delta^2 .
\end{align*}
With Corollary \ref{KummerEquality} one shows now, that $\cc$ is given as stated.
\end{proof}
\begin{proof}[Proof 2]
It follows from (\ref{sumZ}) that $\cc\in \Pi'^\perp$, so $\cc \in \Sym\otimes\Q$. Moreover, together with (\ref{ZT}) we get that
\begin{equation*}
\cc\cdot D_{1}\cdot D_{2}=54\cdot B(D_{1},D_{2})
\end{equation*}
for all $D_{1}$, $D_{2}$ in $H^{2}(\X,\Z)$. Using the non-degeneracy of the Poincar\'e pairing and our knowledge about the Beauville--Bogomolov form on $\X$, we can calculate that
\[
\cc= 4 u_1u_2 + 4v_1v_2 + 4 w_1 w_2 - \frac{1}{3} e^2.  \qedhere
\]
\end{proof}


\begin{corollary}\label{Pi'}
The intersection $\Sym\cap \Pi$ is one-dimensional and spanned by $3\cc$. 
\end{corollary}
\begin{proof}
By Proposition \ref{ChernSym} and (\ref{sumZ}), $3\cc\in \Sym\cap \Pi$. Since the ranks of $\Sym$, $\Pi$ and $H^4(\X,\Z)$ are $28$, $81$ and $108$, respectively, the intersection cannot contain more.
\end{proof}

\begin{corollary}\label{Classuvw}
\begin{equation} \label{YSym}
Y_p =  \frac{1}{6}\Big(u_1u_2 + v_1v_2 +  w_1 w_2 \Big).
\end{equation}
\end{corollary}
\begin{remark}\label{afterClassuvw}
Using Nakajima operators, we may write
\begin{gather}
Y_p = \frac{1}{9} \theta^*\Big( \q_1(1)\mathfrak L_2(1) \vac \Big) =  \frac{1}{2}\theta^*\Big(\q_1(x)\q_1(1)^2\vac\Big).
\end{gather}
\end{remark}
Now, we can summarize all the divisible classes found in $\Sym$ in the following proposition. 
\begin{proposition}\label{classedivisibleSym}
\begin{enumerate}
\item
The class $e^2$ is divisible by $3$ and the class $u_1u_2 + v_1v_2+w_1w_2$ is divisible by 6.
\item
For $y\in\{u_1,u_2,v_1,v_2,w_1,w_2\}$, the class 
$
e \cdot y
$
is divisible by $3$ and 
$
 y^2 - \frac{1}{3} e\cdot y
$
is divisible by $2$.
\end{enumerate}
\end{proposition}
\begin{proof}
\begin{enumerate}
\item
From Proposition \ref{ChernSym} we see that
$e^2$ is divisible by $3$ and by Corollary \ref{Classuvw}
the class $u_1u_2 + v_1v_2+w_1w_2$ is divisible by 6.
\item
We have $y= \theta^*\left(\q_1(a)\q_1(1)^2\vac\right)$ for some $a\in H^2(A,\Z)$. A computation yields:
$$
e\cdot y = 3\cdot \theta^*\Big( \q_2(a)\q_1(1)\vac \Big)
\quad \text{and}\quad
y^2 = \theta^*\Big(\q_1(a)^2\q_1(1)\vac\Big)
$$
so $e\cdot y$ is divisible by $3$. Furthermore, by Corollary \ref{IntegralOperatorsTorus}, the class 
$
\frac{1}{2} \q_1(a)^2\q_1(1)\vac - \frac{1}{2} \q_2(a)\q_1(1)\vac 
$
is contained in $H^4(A\hilb{3},\Z)$, so its pullback
$
 \frac{1}{2}y^2 - \frac{1}{6} e\cdot y
$
is an integral class, too.
\end{enumerate}
\end{proof}
\subsection{Integral basis of $H^4(K_2(A),\Z)$}\label{integralbasisH4}

From the intersection properties $Z_\tau \cdot Z_{\tau'} = 1$ for $\tau\neq \tau'$ and $Z_\tau^2 = 4$ from Section 4 of \cite{Hassett}, we compute
\begin{equation}
 \discr \Pi' = 3^{84}.
 \label{discrPi}
\end{equation}
On the other hand, a formula developed in \cite{Kapfer} evaluates
\begin{equation} \label{DiscrSym}
\discr \Sym \ = 2^{14}\cdot 3^{38},
\end{equation}
so the lattices cannot be primitive. Denote $\Sym^{sat}$ and $\Pi'^{sat}$ the respective primitive overlattices of $\Sym$ and $\Pi'$. $\Sym \oplus\Pi'$ is a sublattice of $H^4(\X,\Z)$ of index $2^{7}\cdot 3^{61}$ and we claim that $\Sym^{sat}\oplus \Pi'^{sat}$ has index $3^{22}$. 
We have already found in Proposition~\ref{classedivisibleSym}
\begin{itemize}
 \item $7$ linearly independent classes in $\Sym$ divisible by $2$,
 \item $8$ linearly independent classes in $\Sym$ divisible by $3$,
\end{itemize}
To obtain a basis of $H^4(\X,\Z)$, 
we are now going to find
\begin{itemize}
 \item $31$ linearly independent classes in $\Pi'$ divisible by $3$ and
 \item $20$ linearly independent classes in $\Sym^{sat}\oplus \Pi'^{sat}$, one divisible by $3^3$ and $19$ divisible by $3$.
\end{itemize}


%Now we come to $\Pi'$. 
The first thing to note is that $\Pi'$ is defined topologically for all deformations of $\X$ and the primitive overlattice of $\Pi'$ is a topological invariant, too.  
By applying a suitable deformation, we may therefore assume without loss of generality that $A$ is the product of two elliptic curves $A=E_1\times E_2$. 

Hassett and Tschinkel in Proposition 7.1 of \cite{Hassett} provide the class of a Lagrangian plane $P\subset K_2(A)$, which can be expressed as follows:
\begin{equation}
\left[P\right]=\frac{1}{216}(6u_1-3e)^2+\frac{1}{8}\cc-\frac{1}{3}\sum_{\tau\in \Lambda'} Z_{\tau},
\label{LagrangianPlaneClass}
\end{equation}
where $\Lambda'=E_1[3]\times 0\subset A[3]$.
Hence by translating this plane by an element $\tau'\in A[3]$, we obtain another plane $P'$ that can be written:
$$\left[P'\right]=\frac{1}{216}(6u_1-3e)^2+\frac{1}{8}\cc-\frac{1}{3}\sum_{\tau\in \Lambda'+\tau'} Z_{\tau}.$$
By substracting these two expressions, we obtain a first class divisible by 3 in $\Pi'$:
\begin{equation}
 \frac{1}{3}\sum_{\tau\in\Lambda'} \Big(Z_{\tau} - Z_{\tau+\tau'}\Big)
 \label{first31}
\end{equation}

%After \cite[Eq.~(12)]{Hassett}, for a non-isotropic plane $\Lambda \subset A[3]$ and any $\tau_0\in A[3]$, the classes 
%\begin{equation}
% \frac{1}{3}\sum_{\tau\in\Lambda} \Big(Z_{\tau} - Z_{\tau+\tau_0}\Big)
%\end{equation}
%are integral. 
By Proposition \ref{Hassettmonodromy}, the image of the monodromy representation on the $Z_\tau$ contains the symplectic group $\Sp(4,\mathbb F_3)$. We know by Proposition \ref{transitively} that $\Sp(4,\mathbb F_3)$ acts transitively on the non-isotropic planes of $A[3]$. Hence, modulo $\Pi'$, the orbit by $\Sp(4,\mathbb F_3)$ of the classes (\ref{first31}) is a $\mathbb F_3$-vector space naturally isomorphic to $D$ as introduced in Definition \ref{SymplecticIdeal}, so by Proposition \ref{SymplecticIdealsDimension}, we get a subspace of $\Pi'$ of rank $31$ of classes divisible by $3$.
This subspace can be determined by a computer. We describe it in Proposition~\ref{XXXI}.

Now we are going to find the classes divisible by 3 in $\Sym^{sat}\oplus \Pi'^{sat}$.
The class $Z_0$ is not contained in $\Sym$ nor in $\Pi'$.
It can be written as follows:
$$
Z_0=\frac{\sum_{\tau\in A[3]}Z_\tau-\sum_{\tau\in A[3]}(Z_\tau-Z_0)}{81}
\stackrel{(\ref{sumZ})}{=} \frac{c_2(K_2(A))-\frac{1}{3}\sum_{\tau\in A[3]}(Z_\tau-Z_0)}{27},
$$
where $\frac{1}{3}\sum_{\tau\in A[3]}(Z_\tau-Z_0)$ can be expressed as a linear combination of the 31 classes of Proposition \ref{XXXI} by Remark \ref{c2}.
Hence $Z_0$ is the class in $\Sym^{sat}\oplus \Pi'^{sat}$ divisible by $27$. 
Let us now find the remaining $19$ classes divisible by $3$.


%Hassett and Tschinkel in Proposition 7.1 of \cite{Hassett} provide the class of a Lagrangian plane $P\subset K_2(A)$, which can be expressed as follows:
%$$\left[P\right]=\frac{1}{216}(6u_1-3e)^2+\frac{1}{8}\cc-\frac{1}{3}\sum_{\tau\in \Lambda'} Z_{\tau},$$
%where $\Lambda'=E_1[3]\times 0\subset A[3]$. 
We rearrange (\ref{LagrangianPlaneClass}) a bit using (\ref{WW}):
\begin{align*}
\left[P\right]&=\frac{1}{216}(6u_1-3e)^2+\frac{1}{8}\cc-\frac{1}{3}\sum_{\tau\in \Lambda'} Z_{\tau}\\
&=\frac{36u_1^2+9e^2-36u_1\cdot e}{216} +\frac{W}{3}-\frac{3}{8}e^2-\frac{1}{3}\sum_{\tau\in \Lambda'} Z_{\tau}\\
%&=\frac{36u_1^2-72e^2-36u_1\cdot e}{216} +\frac{W}{3}-\frac{1}{3}\sum_{\tau\in \Lambda'} Z_{\tau}\\
&=\frac{u_1^2-2e^2-u_1\cdot e}{6} +\frac{W}{3}-\frac{1}{3}\sum_{\tau\in \Lambda'} Z_{\tau}.
\end{align*}
By Proposition \ref{classedivisibleSym}, the classes $e^2$ and $u_1\cdot e$ are both divisible by 3 and by (\ref{WW}), $W$ is divisible by 3, so the following class is integral:
$$
\mathfrak{N}:=\frac{u_1^2+\sum_{\tau\in \Lambda'} (Z_{\tau}-Z_0)}{3}
.
$$
From the considerations in Section \ref{monodromyexplication}, we know that the group $\Sp(A[3])\ltimes A[3]\subset \Mon(\Pi)$ is a natural extension of 
$\Sp(H_1(A,\Z))\ltimes A[3]\subset \Mon(H_1(A,\Z))$. Isometries of the image of the monodromy representation on $H_1(A,\Z)$ extend naturally to isometries of the image of the monodromy representation of $H^4(K_2(A),\Z)$ acting on $\Pi$ as describe in Proposition \ref{Hassettmonodromy} and acting on $\Sym$ by commuting with the map $j$ defined in Notation \ref{BasisH2KA}.
Hence the group $\Sp(H_1(A,\Z))\ltimes A[3]$ can be seen as a subgroup of $\Mon(H^4(K_2(A),\Z))$. 

Now we will conclude using this monodromy action of $\Sp(H_1(A,\Z))\ltimes A[3]$ on the element $\mathfrak{N}$ and the considerations from Section \ref{Section_Symplectic}. 
%Before, we have to introduce some notation.
%Let denote by $O_1$ the $\Ft$ vector space generated by the elements of the orbit of $u_1^2+\sum_{\tau\in \Lambda'} Z_{\tau}$ under the action of $\Sp(A[3])\ltimes A[3]$. We denote by $O_2$ the $\Ft$ vector space generated by the orbit, under the action of $\Sp(A[3])$, of the elements $u_1^2+\sum_{\tau\in\Lambda'} (Z_{\tau} - Z_{\tau+\tau_0})$, with $\tau_0\in A[3]$.
%We denote by $\Sym^{o}=\Sym\cap c_2(K_2(A))^{\bot}\otimes $ and the projection $\Pr_1:\Sym^{o} $
Proposition \ref{CombinedSymplectic} states now that
the orbit of $\mathfrak{N}$ under the action of $\Sp(A[3])\ltimes A[3]$ is spanning a space of rank $51$ modulo $\Sym\oplus\Pi'$. However, by Lemma \ref{cleffinclassesdiv}, the intersection of that space with $\Sym^{sat}$ is one-dimensional and the intersection with $\Pi'^{sat}$ has dimension $31$, so we are left with $19$ linearly independent elements which provide 19 elements in $\frac{H^4(K_2(A),\Z)}{\Sym^{sat}\oplus\Pi'^{sat}}$.
These 19 independent classes can be enumerated using a computer, see Proposition~\ref{XIX}.
%It only remains to check that all the classes above generate $H^4(K_2(A),\Z)$.
Let $\Sym^{over}$ be the overlattice of $\Sym$ obtained by adding all the classes from Proposition \ref{classedivisibleSym}. Hence by (\ref{DiscrSym}) and (\ref{squareDiscr}), the lattice $\Sym^{over}$ has discriminant $3^{22}$. Let $\Pi'^{over}$ be the overlattice of $\Pi'$ obtained by adding all the thirds of the classes of Proposition \ref{XXXI}. Then, by (\ref{discrPi}) and Proposition \ref{squareDiscr}, the lattice $\Pi'^{over}$ has discriminant $3^{22}$. Therefore, the lattice $\Sym^{over}\oplus \Pi'^{over}$ has discriminant $3^{44}$. Finally, let $F$ be the overlattice of $\Sym^{over}\oplus \Pi'^{over}$ obtained by adding $Z_0$ and the thirds of all classes of Proposition \ref{XIX}. Using one more time (\ref{squareDiscr}), we find that $\discr F=1$. Hence necessarily $F=H^4(K_2(A),\Z)$. Moreover, from Proposition \ref{XIX}, we obtain: $$\Sym^{over}=\Sym^{sat} \text{ and } \Pi'^{over}=\Pi'^{sat}.$$ 
We summarize the description of the integral basis of $H^{4}(K_2(A),\Z)$ in the following theorem.
\begin{thm}\label{integralbasistheorem}
Let $A$ be an abelian variety. We use Notation \ref{BasisH2KA} and \ref{TheZs}. 
\begin{enumerate}
\item 
Let $\Sym^{sat}$ be the primitive overlattice of $\Sym^2\left(H^2\left(\X,\Z\right)\right)$ in $H^4(K_2(A),\Z)$.
The group $\frac{\Sym^{sat}}{\Sym^2\left(H^2\left(\X,\Z\right)\right)}=(\Z/2\Z)^{7}\oplus(\Z/3\Z)^{8}$ is generated by the elements:
$$\frac{e \cdot y}{3},\ \frac{y^2 - \frac{1}{3} e\cdot y}{2} \text{ for } y\in\{u_1,u_2,v_1,v_2,w_1,w_2\},\ 
\frac{e^2}{3} \text{ and } \frac{u_{1}\cdot u_{2}+v_{1}\cdot v_{2}+w_{1}\cdot w_{2}}{6}.$$
\item
Let $\Pi'$ be the lattice from Definition \ref{defiPi} and let $\Pi'^{sat}$ be the primitive over lattice of $\Pi'$ in $H^4(K_2(A),\Z)$.
The group $\frac{\Pi'^{sat}}{\Pi'\ \ \ }=(\Z/3\Z)^{31}$ is generated by the classes:
$$\frac{1}{3}\sum_{\tau\in\Lambda} \Big(Z_{\tau} - Z_{\tau+\tau'}\Big),
$$
with $\Lambda$ a non-isotropic group and $\tau'\in A[3]$. Moreover a basis of $\frac{\Pi'^{sat}}{\Pi'\ \ \ }$ is provided by the 31 classes described in Proposition \ref{XXXI}. 
\item
We have 
$$\frac{H^4(K_2(A),\Z)}{\Sym^{sat}\oplus\Pi'^{sat}}=\left(\frac{\Z}{27\Z}\right)\oplus\left(\frac{\Z}{3\Z}\right)^{\oplus 19}.$$
Moreover, this group is generated by the class $Z_0$ and the 19 classes described in Proposition \ref{XIX}.
\end{enumerate}
\end{thm}
Moreover since $\Sym^{over}=\Sym^{sat}$, from the proofs of Proposition \ref{ChernSym}, \ref{classedivisibleSym} and Remark \ref{afterClassuvw}, we obtain the following corollary.
\begin{corollary}\label{SymSatImage}
The image of $H^4(A\hilb{3},\Z)$ under $\theta^*$ is equal to $\Sym^{sat}$. \qed
\end{corollary}
