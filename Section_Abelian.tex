
%\subsection{Homology and Cohomology}\label{monodromyexplication}
%The fundamental group $\pi_1(A,\Z) = H_1(A,\Z)$ is a free $\Z$-module of rank $4$, which is canonically identified with the lattice $\Lambda$. Indeed, the projection of every path in $\C^2$ from $0$ to $v\in \Lambda$ gives a unique element of $\pi_1(A,\Z)$. Conversely, any closed path in $A$ with basepoint $0$ lifts to a unique path in $\C^2$ from $0$ to some $v\in\Lambda$.
%So the first cohomology $H^1(A,\Z)$ is freely generated by four elements, too. Moreover, by~\cite[Sect.~I.1]{Mumford}, the cohomology ring is isomorphic to the exterior algebra
%$$
%H^*(A,\Z) = \Lambda^*(H^1(A,\Z)).
%$$



Let $A$ be a abelian surface. We recall that a \emph{principal polarization} of $A$ is a polarization $L$ such that there exists a basis of $H_1(A,\Z)$, with respect to which the symplectic bilinear form on $H_1(A,\Z)$ induced by $c_1(L)$:
\begin{equation}
\omega_L(x,y)=x\cdot c_1(L)\cdot y,
\label{symplecticprinc}
\end{equation}
is given by the matrix:
$$\left( {\begin{array}{cccc}
   0 & 0 & 1 & 0 \\    0 &  0 & 0 & 1\\ -1 & 0 & 0 & 0\\ 0 & -1 & 0 & 0     
   \end{array} } \right).$$
%We remark that a principal polarization $L$ provides a symplectic bilinear form $\omega_L$ on $H_1(A,\Z)$ as follows:

%for all $x$ and $y$ in $H_1(A,\Z)$.
%TODO: Principal polarization/Jacobians \\
%Let $p$ be a prime number, $n\in\mathbb{N}^{*}$, $\mu_p^n$ the group of $p^n$-th roots of the unity and $\Z_p(1)\defIs \underleftarrow{\lim}\mu_p^n$. 
%We recall that the \emph{Weil pairing} $e_p^L$ can be defined as follows.
%$$e_p^L(x,y)=\varsigma^{-x\cdot c_1(L)\cdot y},$$
%for all $x$, $y$ in $H_1(X,\Z)$ and $\varsigma=(...,e^{\frac{2i\pi}{p^n}},...)$.
We recall the following result. 
\begin{prop}
Let $(A,L)$ be a principally polarized abelian surface. The group $H_1(A,\Z)$ is endowed with the symplectic from $\omega_L$ defined in (\ref{symplecticprinc}). Let $\Mon (H_1(A,\Z))$ be the image of monodromy representations on $H_1(A,\Z)$.
Then $\Mon (H_1(A,\Z))\supset\Sp(H_1(A,\Z))$.
%TODO: Weil pairing for tori
\end{prop}
\begin{proof}
It can be seen as follows.
Let $\mathcal{M}_2$ be the moduli space of curves of genus $2$ and $\mathcal{A}_2$ be the moduli space of principally polarized abelian surfaces.
By the Torelli theorem (see for instance~\cite[Theorem 12.1]{Milne}), we have an injection $J:\mathcal{M}_2\hookrightarrow \mathcal{A}_2$ given by taking the Jacobian of the curve endowed with its canonical polarization. Moreover, the moduli spaces $\mathcal{M}_2$ and $\mathcal{A}_2$ are both of dimension 3. 
%$J(\mathcal{M}_2)=\mathcal{A}_2\smallminus \mathcal{A}_1\times\mathcal{A}_1$.

Now if $\mathscr{C}_2$ is a curve of genus 2, we have by Theorem 6.4 of~\cite{Farb}: 
$$\Mon (H_1(\mathscr{C}_2,\Z))\supset \Sp(H_1(\mathscr{C}_2,\Z)),$$
where the symplectic form on $H_1(\mathscr{C}_2,\Z)$ is given by the cup product. 
Then the result follows from the fact that the lattices $H_1(\mathscr{C}_2,\Z)$ and $H_1(J(\mathscr{C}_2),\Z)$ are isometric.
%, where $H_1(\mathscr{C}_2,\Z)$ is endowed with the cup product and $H_1(J(\mathscr{C}_2),\Z)$ with $\omega_$ with $L$ the principal polarization.
\end{proof}
\begin{rmk}\label{SPA2}
Let $(A,L)$ be a principally polarized abelian surface and $p$ a prime number. The group $H_1(A,\Z)$ tensorized by $\mathbb{F}_p$ can be seen as the group $A[p]$ of points of $p$-torsion on $A$ and the form $\omega_L\otimes\mathbb{F}_p$ provides a symplectic form on $A[p]$. Then $\Mon (A[p])$, the image of the monodromy representation on $A[p]$ contains the group $\Sp(A[p])$. 
\end{rmk}
Now, we are ready to recall Proposition 5.2 of~\cite{Hassett} on the monodromy of the generalized Kummer fourfold.
\begin{prop}\label{Hassettmonodromy}
Let $A$ be an abelian surface and $K_2(A)$ the associated generalized Kummer fourfold. 
The image of the monodromy representation on $\Pi=\left\langle\left. Z_\tau\right|\ \tau\in A[3]\right\rangle$ contains the semidirect product
$\Sp(A[3])\ltimes A[3]$ which acts as follows:
$$f\cdot Z_\tau= Z_{f(\tau)}\ \text{and}\ \tau'\cdot Z_\tau= Z_{\tau+\tau'},$$
for all $f\in \Sp(A[3])$ and $\tau'\in A[3]$.
\end{prop}