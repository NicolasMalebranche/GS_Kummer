\section{Complex abelian surfaces}\label{AbelianSection}
Denote $A$ a complex abelian surface (a torus of dimension $2$). As such, it always can be written as a quotient
$$
A = \C^2/\Lambda,
$$
where $\Lambda\subset \C^2$ is a lattice of rank $4$, embedded in $\C^2$. 
Depending on the imbedding, we get different complex manifolds, projective or not. Of course, all of them are equivalent by deformation.
\subsection{Morphisms and special cases}
\begin{definition}
An isogeny between abelian surfaces $A=\C^2/\Lambda\rightarrow A'=\C^2/\Lambda'$ means a surjective holomorphic map that preserves the group structure. It is given by a complex linear map, that maps $\Lambda$ to a sublattice of $\Lambda'$. 
\end{definition}
\begin{example}
For a number $n\neq 0$, the multiplication map $n: A\rightarrow A$, $x\mapsto n\cdot x$ is an isogeny.
\end{example}


By an automorphism of $A$ we mean a biholomorphism preserving the group structure. It can be represented by a $\C$-linear map $ M:\C^2 \rightarrow \C^2$ with $M\Lambda =\Lambda$. Have a look in~\cite{Fujiki} or the appendix of~\cite{Ghys} for some reference.
Let us now come to the very special case that $A=E\times E$ can be written as the square of an elliptic curve. Note that $A$ is projective, because every elliptic curve is. 
Now write $E$ as $E=\C/\Lambda_0$. We may assume that $\Lambda_0 $ is spanned by $1$ and a vector $\tau\in\C\backslash\R$. The automorphism group, up to isogeny, is given by (\cite{Ghys})
$\GL(2,\End(\Lambda_0))$, where
$\End(\Lambda_0)$ is the set $\{z\in\C \;|\; z\Lambda_0\subset \Lambda_0\}$.
\begin{proposition} \label{EndLambda}
There are two possibilities for $\End(\Lambda_0)$, depending on $\tau$:
\begin{enumerate}
 \item Both the real part and the square norm of $\tau$ are rational numbers, say $2\Re(\tau) = \frac{p}{r}$ and $\|\tau\|^2 = \frac{q}{r}$ with $r>0$ as small as possible. Then $\End(\Lambda_0)= \Z+ r\tau\Z$.
 \item At least one of $\Re(\tau), \|\tau\|^2$ is irrational. Then $\End(\Lambda_0)=\Z$.
\end{enumerate}
\end{proposition}
\begin{proof}
Given $z\in\End(\Lambda_0)$, we have $$z\cdot 1 = a + b\tau\text{ and }z\cdot \tau = c+ d\tau\text{ with }a,b,c,d\in \Z.$$ 
We get the condition
$$
(a+b\tau)\tau = c+d\tau\quad \Leftrightarrow \quad b\tau^2 + (a-d)\tau -c =0.
$$
Up to scalar multiples, there is a unique real quadratic polynomial that annihilates $\tau$, namely $ (x -\tau )(x-\bar{\tau})=x^2 - 2\Re(\tau)x+ \|\tau\|^2$. 
If all coefficients of that polynomial are rational numbers, then $z=a+b\tau$ gives a solution for arbitrary $a\in\Z$, $b\in r\Z$. Otherwise, the condition must be the zero polynomial, so $b=0$.
\end{proof}
Now we study the action of automorphisms on torsion points in a very special case. This will be needed in the technical proof of Theorem~\ref{fin}.
\begin{definition}\label{elliptic6}
Denote $\xi\in\C$ a primitive sixth root of unity and $E_\xi$ the elliptic curve given by the choice $\Lambda_0 = \left<1,\xi\right>$, so by Proposition~\ref{EndLambda}, $\End(\Lambda_0)=\Lambda_0$ is the ring of Eisenstein integers. Define a group $G_\xi$ of automorphisms of $E_\xi\times E_\xi$ by the following generators in $\GL(2,\End(\Lambda_0))$:
\begin{align*}
g_1 &= \left( {\begin{array}{cc}
   \xi & 0 \\       0 & 1      
   \end{array} } \right),
 &
g_2 &= \left( {\begin{array}{cc}
   0 & 1 \\       1 & 0      
   \end{array} } \right),
 &
g_3 &= \left( {\begin{array}{cc}
   1 & 1 \\       0 & 1     
   \end{array} } \right).
\end{align*}
\end{definition}
For $A=E_\xi\times E_\xi$, let $V =A[2]$ be the (fourdimensional) $\mathbb F_2$-vector space of $2$-torsion points on $A$ and let $\mathfrak T$ be the set of planes in $V$. Note that by Remark~\ref{PlaneTriple} a plane in $V$ can be identified with an unordered triple $\{x,y,z\}$ with $0\neq x,y,z\in V$ and $x+y+z=0$. The action of $G_\xi$ on $A$ induces actions of $G_\xi$ on $A[2]$ and $\mathfrak T$. 
\begin{lemma}\label{orbitesG}
There are two orbits of $G_\xi$ on $\mathfrak T$, of cardinalities $5$ and $30$.
\end{lemma}
\begin{proof}
Note that the generators $g_2$ and $g_3$ exist because $A$ is of the form $E\times E$, while $g_1$ exists only in the special case $E=E_\xi$. Indeed, multiplication with $\xi$ induces a cyclic permutation on $E_\xi[2]$. 
The orbits can be explicitely determined by a suitable computer program. For verification, we give one of the orbits explicitely.
Denote $x_1,x_2,x_3$ the non-zero points in $E_\xi[2]$. The orbit of cardinality five is then given by
\begin{gather*}
\{(0,x_1),(0,x_2),(0,x_3)\} , \quad  \{(x_1,0),(x_2,0),(x_3,0)\},\quad  \{(x_1,x_1),(x_2,x_2),(x_3,x_3)\}  \\
 \{(x_1,x_2),(x_2,x_3),(x_3,x_1)\}, \quad  \{(x_1,x_3),(x_2,x_1),(x_3,x_2)\}. \qedhere
\end{gather*}
\end{proof}

\subsection{Homology and Cohomology}\label{monodromyexplication}
The fundamental group $\pi_1(A,\Z) = H_1(A,\Z)$ is a free $\Z$-module of rank $4$, which is canonically identified with the lattice $\Lambda$. Indeed, the projection of every path in $\C^2$ from $0$ to $v\in \Lambda$ gives a unique element of $\pi_1(A,\Z)$. Conversely, any closed path in $A$ with basepoint $0$ lifts to a unique path in $\C^2$ from $0$ to some $v\in\Lambda$.
So the first cohomology $H^1(A,\Z)$ is freely generated by four elements, too. Moreover, by~\cite[Sect.~I.1]{Mumford}, the cohomology ring is isomorphic to the exterior algebra
$$
H^*(A,\Z) = \Lambda^*(H^1(A,\Z)).
$$
\begin{notation} \label{TorusClasses}
We denote the generators of $H^1(A,\Z)$ by $a_i$, $1\leq i\leq 4$ and their respective duals by $a_i^*\in H^3(A,\Z)$. 
If $A=E\times E$ is the product of two elliptic curves, we choose the $a_i$ in a way such that $\{a_1,a_2\}$ and $\{a_3,a_4\}$ give bases of $H^1(E,\Z)$ in the decomposition $H^1(A) = H^1(E)\oplus H^1(E)$.
We denote the generator of the top cohomology $H^4(A,\Z)$ by $x \defIs  a_1 a_2 a_3 a_4$.
A basis of $H^2(A,\Z)$ will be denoted by $(b_i)_{1\leq i \leq 6}$.
\end{notation}



Let $A$ be a abelian surface. We recall that a \emph{principal polarization} of $A$ is a polarization $L$ such that there exists a basis of $H_1(A,\Z)$, with respect to which the symplectic bilinear form on $H_1(A,\Z)$ induced by $c_1(L)$:
\begin{equation}
\omega_L(x,y)=x\cdot c_1(L)\cdot y,
\label{symplecticprinc}
\end{equation}
is given by the matrix:
$$\left( {\begin{array}{cccc}
   0 & 0 & 1 & 0 \\    0 &  0 & 0 & 1\\ -1 & 0 & 0 & 0\\ 0 & -1 & 0 & 0     
   \end{array} } \right).$$
%We remark that a principal polarization $L$ provides a symplectic bilinear form $\omega_L$ on $H_1(A,\Z)$ as follows:

%for all $x$ and $y$ in $H_1(A,\Z)$.
%TODO: Principal polarization/Jacobians \\
%Let $p$ be a prime number, $n\in\mathbb{N}^{*}$, $\mu_p^n$ the group of $p^n$-th roots of the unity and $\Z_p(1)\defIs \underleftarrow{\lim}\mu_p^n$. 
%We recall that the \emph{Weil pairing} $e_p^L$ can be defined as follows.
%$$e_p^L(x,y)=\varsigma^{-x\cdot c_1(L)\cdot y},$$
%for all $x$, $y$ in $H_1(X,\Z)$ and $\varsigma=(...,e^{\frac{2i\pi}{p^n}},...)$.
We recall the following result. 
\begin{prop}
Let $(A,L)$ be a principally polarized abelian surface. The group $H_1(A,\Z)$ is endowed with the symplectic from $\omega_L$ defined in (\ref{symplecticprinc}). Let $\Mon (H_1(A,\Z))$ be the image of monodromy representations on $H_1(A,\Z)$.
Then $\Mon (H_1(A,\Z))\supset\Sp(H_1(A,\Z))$.
%TODO: Weil pairing for tori
\end{prop}
\begin{proof}
It can be seen as follows.
Let $\mathcal{M}_2$ be the moduli space of curves of genus $2$ and $\mathcal{A}_2$ be the moduli space of principally polarized abelian surfaces.
By the Torelli theorem (see for instance~\cite[Theorem 12.1]{Milne}), we have an injection $J:\mathcal{M}_2\hookrightarrow \mathcal{A}_2$ given by taking the Jacobian of the curve endowed with its canonical polarization. Moreover, the moduli spaces $\mathcal{M}_2$ and $\mathcal{A}_2$ are both of dimension 3. 
%$J(\mathcal{M}_2)=\mathcal{A}_2\smallminus \mathcal{A}_1\times\mathcal{A}_1$.

Now if $\mathscr{C}_2$ is a curve of genus 2, we have by Theorem 6.4 of~\cite{Farb}: 
$$\Mon (H_1(\mathscr{C}_2,\Z))\supset \Sp(H_1(\mathscr{C}_2,\Z)),$$
where the symplectic form on $H_1(\mathscr{C}_2,\Z)$ is given by the cup product. 
Then the result follows from the fact that the lattices $H_1(\mathscr{C}_2,\Z)$ and $H_1(J(\mathscr{C}_2),\Z)$ are isometric.
%, where $H_1(\mathscr{C}_2,\Z)$ is endowed with the cup product and $H_1(J(\mathscr{C}_2),\Z)$ with $\omega_$ with $L$ the principal polarization.
\end{proof}
\begin{rmk}\label{SPA2}
Let $(A,L)$ be a principally polarized abelian surface and $p$ a prime number. The group $H_1(A,\Z)$ tensorized by $\mathbb{F}_p$ can be seen as the group $A[p]$ of points of $p$-torsion on $A$ and the form $\omega_L\otimes\mathbb{F}_p$ provides a symplectic form on $A[p]$. Then $\Mon (A[p])$, the image of the monodromy representation on $A[p]$ contains the group $\Sp(A[p])$. 
\end{rmk}
Now, we are ready to recall Proposition 5.2 of~\cite{Hassett} on the monodromy of the generalized Kummer fourfold.
\begin{prop}\label{Hassettmonodromy}
Let $A$ be an abelian surface and $K_2(A)$ the associated generalized Kummer fourfold. 
The image of the monodromy representation on $\Pi=\left\langle\left. Z_\tau\right|\ \tau\in A[3]\right\rangle$ contains the semidirect product
$\Sp(A[3])\ltimes A[3]$ which acts as follows:
$$f\cdot Z_\tau= Z_{f(\tau)}\ \text{and}\ \tau'\cdot Z_\tau= Z_{\tau+\tau'},$$
for all $f\in \Sp(A[3])$ and $\tau'\in A[3]$.
\end{prop}