\section{Complex abelian surfaces}
Denote $A$ a complex abelian surface (a torus of dimension $2$). As such, it always can be written as a quotient
$$
A = \C^2/\Lambda,
$$
where $\Lambda\subset \C^2$ is a lattice of rank $4$, embedded in $\C^2$. 
Depending on the imbedding, we get different complex manifolds, projective or not. Of course, all of them are equivalent by monodromy.
\subsection{Morphisms}
\begin{definition}
An isogeny between abelian surfaces $A=\C^2/\Lambda\rightarrow A'=\C^2/\Lambda'$ means a surjective holomorphic map that preserves the group structure. It is given by a complex linear map, that maps $\Lambda$ to a sublattice of $\Lambda'$. 
\end{definition}
\begin{example}
For a number $n\neq 0$, the multiplication map $n: A\rightarrow A$, $x\mapsto n\cdot x$ is an isogeny.
\end{example}


By an automorphism of $A$ we mean a biholomorphism preserving the group structure. It can be represented by a $\C$-linear map $ M:\C^2 \rightarrow \C^2$ with $M\Lambda =\Lambda$. Have a look in \cite{Fujiki} or the appendix of \cite{Ghys} for some reference.
Let us now come to the very special case that $A=E\times E$ can be written as the square of an elliptic curve. Note that $A$ is projective, because every elliptic curve is. 
Now write $E$ as $E=\C/\Lambda_0$. We may assume that $\Lambda_0 $ is spanned by $1$ and a vector $\tau\in\C\backslash\R$. The automorphism group, up to isogeny, is given by (\cite{Ghys})
$\GL(2,\End(\Lambda_0))$, where
$\End(\Lambda_0)$ is the set $\{z\in\C \;|\; z\Lambda_0\subset \Lambda_0\}$.
\begin{proposition} There are two possibilities for $\End(\Lambda_0)$, depending on $\tau$:
\begin{enumerate}
 \item Both the real part and the square norm of $\tau$ are rational numbers, say $2\Re(\tau) = \frac{p}{r}$ and $\|\tau\|^2 = \frac{q}{r}$ with $r>0$ as small as possible. Then $\End(\Lambda_0)= \Z+ r\tau\Z$.
 \item At least one of $\Re(\tau), \|\tau\|^2$ is irrational. Then $\End(\Lambda_0)=\Z$.
\end{enumerate}
\end{proposition}
\begin{proof}
Given $z\in\End(\Lambda_0)$, we have $$z\cdot 1 = a + b\tau\text{ and }z\cdot \tau = c+ d\tau\text{ with }a,b,c,d\in \Z.$$ 
We get the condition
$$
(a+b\tau)\tau = c+d\tau\quad \Leftrightarrow \quad b\tau^2 + (a-d)\tau -c =0.
$$
Up to scalar multiples, there is a unique real quadratic polynomial that annihilates $\tau$, namely $ (x -\tau )(x-\bar{\tau})=x^2 - 2\Re(\tau)x+ \|\tau\|^2$. 
If all coefficients of that polynomial are rational numbers, then $z=a+b\tau$ gives a solution for arbitrary $a\in\Z$,$b\in r\Z$. Otherwise, the condition must be the zero polynomial, so $b=0$.
\end{proof}

\begin{definition}
Denote $\xi\in\C$ a primitive sixth root of unity and $E_\xi$ the elliptic curve given by the choice $\Lambda_0 = \left<1,\xi\right>$, so $\End(\Lambda_0)=\Lambda_0$ is the ring of Eisenstein integers. Define a group $G_\xi$ of automorphisms of $E_\xi\times E_\xi$ by the following generators in $\GL(2,\End(\Lambda_0))$:
\begin{align*}
g_1 &= \left( {\begin{array}{cc}
   \xi & 0 \\       0 & 1      
   \end{array} } \right),
 &
g_2 &= \left( {\begin{array}{cc}
   0 & 1 \\       1 & 0      
   \end{array} } \right),
 &
g_3 &= \left( {\begin{array}{cc}
   1 & 1 \\       0 & 1     
   \end{array} } \right).
\end{align*}
\end{definition}
For $A=E_\xi\times E_\xi$, let $V =A[2]$ be the (fourdimensional) $\mathbb F_2$-vector space of $2$-torsion points on $A$ and let $\mathfrak T$ be the set of planes in $V$. Note that by Remark \ref{PlaneTriple} a plane in $V$ can be identified with an unordered triple $\{x,y,z\}$ with $0\neq x,y,z\in V$ and $x+y+z=0$. The action of $G_\xi$ on $A$ induces actions of $G_\xi$ on $A[2]$ and $\mathfrak T$. 
\begin{proposition} 
There are two orbits of $G_\xi$ on $\mathfrak T$, of cardinalities $5$ and $30$.
\end{proposition}
\begin{proof}
Note that the action of the multiplication with $\xi$ induces a cyclic permutation on $E_\xi[2]$. The orbits can be explicitely computed.
\end{proof}

\subsection{Homology and Cohomology}
The fundamental group $\pi_1(A,\Z) = H_1(A,\Z)$ is a free $\Z$-module of rank $4$, which is canonically identified with the lattice $\Lambda$. Indeed, the projection of every path in $\C^2$ from $0$ to $v\in \Lambda$ gives a unique element of $\pi_1(A,\Z)$. Conversely, any closed path in $A$ with basepoint $0$ lifts to a unique path in $\C^2$ from $0$ to some $v\in\Lambda$.
So the first cohomology $H^1(A,\Z)$ is freely generated by four elements, too. Moreover, by \cite[Sect.~I.1]{Mumford}, the cohomology ring is isomorphic to the exterior algebra
$$
H^*(A,\Z) = \Lambda^*(H^1(A,\Z)).
$$
\begin{notation} \label{TorusClasses}
We denote the generators of $H^1(A,\Z)$ by $a_i$, $1\leq i\leq 4$. If $A=E\times E$ is the product of two elliptic curves, we choose the $a_i$ in a way such that $\{a_1,a_2\}$ and $\{a_3,a_4\}$ give bases of $H^1(E,\Z)$ in the decomposition $H^1(A) = H^1(E)\oplus H^1(E)$.
Further, we set  
\begin{align*}
u_1 &= a_1 a_2, & v_1 &= a_1 a_3, & w_1 &= a_1 a_4, \\ 
u_2 &= a_3 a_4, & v_2 &= a_4 a_2, & w_2 &= a_2 a_3,
\end{align*}
and these elements form a basis of $H^2(A,\Z)$. We denote the generator of the top cohomology $H^4(A,\Z)$ by $x := a_1 a_2 a_3 a_4$.
\end{notation}




TODO: Principal polarization/Jacobians \\
TODO: Weil pairing for tori