\documentclass[landscape]{amsart}
\usepackage[a4paper, left=2cm, right=2cm, top=2cm]{geometry}
\usepackage{amsmath,amssymb,amsfonts,amscd}
\usepackage[all]{xy}
\usepackage{appendix,listings,hyperref}

\DeclareMathOperator{\rank}{rank}
\DeclareMathOperator{\trace}{tr}
\DeclareMathOperator{\Tor}{Tor}
\DeclareMathOperator{\Ext}{Ext}
\DeclareMathOperator{\Aut}{Aut}
\DeclareMathOperator{\Sp}{Sp}
\DeclareMathOperator{\GL}{GL}
\DeclareMathOperator{\End}{End}
\DeclareMathOperator{\id}{id}
\DeclareMathOperator{\Hom}{Hom}
\DeclareMathOperator{\im}{Im}
\DeclareMathOperator{\Ker}{Ker}
\DeclareMathOperator{\Sym}{Sym}
\DeclareMathOperator{\Hilb}{Hilb}
\DeclareMathOperator{\ch}{ch}
\DeclareMathOperator{\rk}{rk}
\DeclareMathOperator{\ad}{ad}
\DeclareMathOperator{\td}{td}
\DeclareMathOperator{\supp}{supp}


\newcommand{\hilb}[1]{^{[#1}}
\newcommand{\ie}{{\it i.e. }}
\newcommand{\eg}{{\it e.g. }}
\newcommand{\loccit}{{\it loc. cit. }}
\newcommand{\vac}{|0\rangle}
\newcommand{\odd}{{\rm{odd}}}
\newcommand{\even}{{\rm{even}}}
\newcommand{\tors}{{\rm{tors}}}


\newcommand{\coloneqq}{:=}
\newcommand{\kum}[2]{K_{ #2 }( #1 )}


%%%%%%%%%%%%%%%%%%%%%%%%%%%%%%

\newcommand{\C}{\mathbb{C}}
\renewcommand{\H}{\mathbb{H}}
\newcommand{\R}{\mathbb{R}}
\newcommand{\Q}{\mathbb{Q}}
\newcommand{\Z}{\mathbb{Z}}
\newcommand{\F}{{\mathbb{ F }_3}}

%%%%%%%%%%%%%%%%%%%%%%%%%%%%%

\newcommand{\kS}{\mathfrak{S}}

\newcommand{\km}{\mathfrak{m}}
\newcommand{\kq}{\mathfrak{q}}

\newcommand{\vect}[1]{\left( \begin{smallmatrix} #1 \end{smallmatrix} \right)}
\newcommand{\plan}[2]{\left< \vect{ #1 }, \vect{ #2 } \right>}

%%%%%%%%%%%%%%%%%%%%%%%%%%%%%%

\newcommand{\lra}{\longrightarrow}
\newcommand{\ra}{\rightarrow}

%%%%%%%%%%%%%%%%%%%%%%%%%%%%%

\theoremstyle{plain}
\newtheorem{theorem}{Theorem}[section]
\newtheorem{lemma}[theorem]{Lemma}
\newtheorem{proposition}[theorem]{Proposition}
\newtheorem{conjecture}[theorem]{Conjecture}
\newtheorem{question}[theorem]{Question}
\theoremstyle{definition}
\newtheorem{definition}[theorem]{Definition}
\newtheorem{problem}[theorem]{Problem}
\theoremstyle{remark}
\newtheorem{remark}[theorem]{Remark}
\newtheorem{example}[theorem]{Example}


%%%%%%%%%%%%%%%%%%%%%%%%%%%%%
\allowdisplaybreaks
\begin{document}

\title{19 classes}

\author{Simon Kapfer}

\date{\today}

%\keywords{}

\maketitle

We denote by $u_1,u_2,v_1,v_2,w_1,w_2$ the basis of $U^{\oplus 3} \subset H^2(\kum{A}{2},\Z)$. Look at the following classes:

%\begin{align}
%u_2^2  &\ +\ \sum_{\tau\in P_1} Z_\tau, & P_1 &= \plan{0\\0\\1\\0}{0\\0\\0\\1}, \\
%v_2^2-v_2u_2  &\ +\ \sum_{\tau\in P_2} Z_\tau-\sum_{\tau\in P_1} Z_\tau, & P_2 &= \plan{0\\1\\2\\0}{0\\0\\0\\1}, \\
%v_2u_2 &\ +\ \sum_{\tau\in P_1} Z_{\tau+\tau_1} - \sum_{\tau\in P_2} Z_{\tau+\tau_1} - \sum_{\tau\in P_3} Z_{\tau+\tau_2},& P_3 &= \plan{0\\1\\0\\0}{0\\0\\0\\1},\ \tau_1 = \vect{-1\\-1\\0\\0}, \tau_2 =\vect{0\\0\\-1\\0}, \\
%w_2^2+w_2u_2 &\ +\ \sum_{\tau\in P_4} Z_\tau -\sum_{\tau\in P_1} Z_\tau , & P_4 &= \plan{0\\1\\0\\1}{0\\0\\1\\1}, \\
%w_2u_2 &\ +\ \sum_{\tau\in P_5} Z_{\tau+\tau_3} - \sum_{\tau\in P_1} Z_{\tau+\tau_4} - \sum_{\tau\in P_6} Z_{\tau+\tau_5},& P_5 &=\plan{0\\0\\1\\0}{0\\1\\0\\1},P_6=\plan{0\\1\\0\\0}{0\\0\\1\\0}, \tau_3 =\vect{2\\0\\0\\2},\tau_4=\vect{2\\1\\0\\0},\tau_5= \vect{0\\0\\0\\2} ,\\
%w_2v_2 &\ +\ \sum_{\tau\in P_7} Z_\tau-\sum_{\tau\in P_8} Z_\tau,&P_7 &=\plan{0\\1\\0\\0}{0\\0\\1\\2},P_8=\plan{0\\1\\0\\0}{0\\0\\1\\1}, 
%\end{align}

\begin{align}\label{initial}
u_2^2& \ + \ \sum_{i\in P} X_i & \text{for } P &= \plan{0\\0\\0\\1}{0\\0\\1\\0}\\
v_2^2+v_2u_2+u_2^2& \ + \ \sum_{i\in P} X_i & \text{for } P &= \plan{0\\0\\0\\1}{0\\1\\1\\0}\\
w_2^2+w_2u_2+u_2^2& \ + \ \sum_{i\in P} X_i & \text{for } P &= \plan{0\\0\\1\\0}{0\\1\\0\\1}\\
w_2^2-w_2u_2+u_2^2& \ + \ \sum_{i\in P} X_i & \text{for } P &= \plan{0\\0\\1\\0}{0\\1\\0\\2}\\
w_2^2-w_2v_2+w_2u_2+v_2^2+v_2u_2+u_2^2& \ + \ \sum_{i\in P} X_i & \text{for } P &= \plan{0\\0\\1\\2}{0\\1\\0\\1}\\
w_1^2+w_1u_2+u_2^2& \ + \ \sum_{i\in P} X_i & \text{for } P &= \plan{0\\0\\0\\1}{1\\0\\2\\0}\\
w_1^2-w_1u_2+u_2^2& \ + \ \sum_{i\in P} X_i & \text{for } P &= \plan{0\\0\\0\\1}{1\\0\\1\\0}\\
v_1^2+v_1u_2+u_2^2& \ + \ \sum_{i\in P} X_i & \text{for } P &= \plan{0\\0\\1\\0}{1\\0\\0\\1}\\
v_1^2-v_1u_2+u_2^2& \ + \ \sum_{i\in P} X_i & \text{for } P &= \plan{0\\0\\1\\0}{1\\0\\0\\2}\\
v_1^2+v_1w_1-v_1u_2+w_1^2+w_1u_2+u_2^2& \ + \ \sum_{i\in P} X_i & \text{for } P &= \plan{0\\0\\1\\2}{1\\0\\0\\2}\\
v_1^2+v_1w_1-v_1w_2-v_1v_2+v_1u_2+w_1^2+w_1w_2+w_1v_2-w_1u_2+w_2^2-w_2v_2+w_2u_2+v_2^2+v_2u_2+u_2^2& \ + \ \sum_{i\in P} X_i & \text{for } P &= \plan{0\\0\\1\\2}{1\\1\\0\\1}\\
v_1^2-v_1w_1+v_1w_2-v_1v_2+v_1u_2+w_1^2+w_1w_2-w_1v_2+w_1u_2+w_2^2+w_2v_2-w_2u_2+v_2^2+v_2u_2+u_2^2& \ + \ \sum_{i\in P} X_i & \text{for } P &= \plan{0\\0\\1\\1}{1\\2\\0\\1}\\
u_1^2& \ + \ \sum_{i\in P} X_i & \text{for } P &= \plan{0\\1\\0\\0}{1\\0\\0\\0}\\
u_1^2-u_1v_2+v_2^2& \ + \ \sum_{i\in P} X_i & \text{for } P &= \plan{0\\1\\0\\0}{1\\0\\0\\1}\\
u_1^2+u_1v_2+v_2^2& \ + \ \sum_{i\in P} X_i & \text{for } P &= \plan{0\\1\\0\\0}{1\\0\\0\\2}\\
u_1^2+u_1w_1+w_1^2& \ + \ \sum_{i\in P} X_i & \text{for } P &= \plan{0\\1\\0\\2}{1\\0\\0\\0}\\
u_1^2+u_1w_1-u_1v_2+w_1^2+w_1v_2+v_2^2& \ + \ \sum_{i\in P} X_i & \text{for } P &= \plan{0\\1\\0\\2}{1\\0\\0\\1}\\
u_1^2+u_1w_1-u_1w_2+u_1v_2-u_1u_2+w_1^2+w_1w_2-w_1v_2+w_1u_2+w_2^2+w_2v_2-w_2u_2+v_2^2+v_2u_2+u_2^2& \ + \ \sum_{i\in P} X_i & \text{for } P &= \plan{0\\1\\0\\2}{1\\0\\2\\2}\\
\label{superfluous}
u_1^2-u_1w_1+u_1w_2-u_1u_2+w_1^2+w_1w_2-w_1u_2+w_2^2+w_2u_2+u_2^2& \ + \ \sum_{i\in P} X_i & \text{for } P &= \plan{0\\1\\0\\1}{1\\0\\1\\0}\\
u_1^2+u_1v_1-u_1w_1+v_1^2+v_1w_1+w_1^2& \ + \ \sum_{i\in P} X_i & \text{for } P &= \plan{0\\1\\2\\1}{1\\0\\0\\0}
\end{align}
These 20 classes are contained in the orbit of (\ref{initial}) under the action of the symplectic group. If we project onto $Sym^2$, we get 20 linearly independent vectors. 
Since the element $u_1u_2 + v_1v_2+ w_1w_2$ is already counted, the element (\ref{superfluous}), for instance, may be left out.

\end{document}
