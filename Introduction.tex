\section{Introduction}
%\subsection{Context and main results}
In algebraic geometry \emph{irreducible holomorphic symplectic (IHS) manifolds} became important objects of study in recent years, after fundamental results by Beauville \cite{Beauville} and Huybrechts \cite{Huybrechts2}.
Among all the developments concerning this field, integral cohomology plays an inescapable role. 
This is primarily due to the \emph{Beauville--Bogomolov form} which is a non-degenerated symmetric integral and primitive bilinear pairing on the second cohomology group. 
This form endows the second cohomology group with a lattice structure establishing lattice theory as a fundamental tool omnipresent in all the last developments. 
As examples, we can cite works on classifications of automorphisms \cite{Mongardi}, \cite{MongWanTari}, \cite{BCS} or the important survey of Markman \cite{Markmansurvey} with results on the K�hler cone and the monodromy. 
In a more modest term, the fourth integral cohomology group is also quite useful. As examples, we can underline Theorem 1.2 of \cite{BNS} providing formulas which apply for the classification of automorphism on IHS manifolds of $K3^{[2]}$-type, in particularly used in \cite{BCS}; furthermore Theorem 1.10 of \cite{Markman2} provides a description of the monodromy group of the IHS manifolds of $K3^{[n]}$-type; we can also cite \cite{Lol2}, where the second author provide the Beauville--Bogomolov lattice of the Markushevich--Tikhomirov varieties constructed in \cite{Markou}. 
Taking $X$ a IHS manifold of $K3^{[2]}$-type, in all these works a description of $\frac{H^{4}(X,\Z)}{\Sym^2(H^{2}(X,\Z))}$ was essential.

Until now, no complete description of the \emph{integral cohomology of the generalized Kummer fourfold} was existing. In particular, the relation between the fourth cohomology group and the image of the symmetric power of the second cohomology group via cup-product was not known. For all reasons mentioned above, it appeared to us that it was an interesting gap to fill.

%It is the main result of this paper (Theorem \ref{thetaTheorem}):  
%Let $A$ be an abelian surface. Let $W_\tau\subset K_2(A)$ be the sub-variety consisting in the subschemes supported entirely at a 3-torsion point $\tau\in A[3]$. 
Let $\kum{A}{2}$ be the generalized Kummer fourfold over a torus $A$. There are three main theorems in this paper. Two of them describe the integral cohomology of the generalized Kummer fourfold:
\begin{itemize}
\item\textbf{Theorem \ref{integralbasistheorem}}
which provides an integral basis of $H^4(K_2(A),\Z)$ in terms of $\Sym^2(H^2(K_2(A),$
$\Z))$ and certain classes of Brian\c con subschemes with support on three-torsion points, introduced in \cite{Hassett}.
\item\textbf{Theorem \ref{thetaTheorem}}
which claims that $\theta^*:H^{*}(A^{[3]},\Z)\rightarrow H^{*}(K_2(A),\Z)$ is surjective except in degree 4, provides an integral basis of $\im \theta^*$ 
and shows that the kernel of $\theta^*$ is the ideal generated by $H^1(A\hilb{3},\Z)$.
\end{itemize}


The third theorem is related to \emph{irreducible symplectic V-manifolds}; it can be seen as an application of Theorem \ref{integralbasistheorem} and a generalization of \cite{Lol2}. A V-manifold is a compact analytic complex space with at worst finite quotient singularities. A V-manifold will be called symplectic if its nonsingular locus is endowed with an everywhere non-degenerate holomorphic 2-form which extends to a resolution of singularities. 
A symplectic V-manifold will be called irreducible if it is complete, simply connected, and if the holomorphic 2-form is unique up to $\mathbb{C}^*$. Such varieties are good candidates to generalize the short list of known IHS manifolds, since some aspects of the theory were already generalized in \cite{Nanikawa} and \cite{Mat}, for instance the Beauville--Bogomolov form, the local Torelli theorem and the Fujiki formula. 

%Concretely, let $X$ be an irreducible symplectic fourfold of Kummer type and $\iota$ a symplectic involution on $X$. 
%By results of Mongardi, Tari and Wandel \cite{MongWanTari} and \cite{Tari}, it can be proved 
%that the fixed locus of $\iota$ is the union of 36 points and a K3 surface $Z_0$. Then the singular locus of $K:=X/\iota$ is the union of a K3 surface and 36 points. A more interesting variety to consider is the partial resolution $K'$ of $K$ obtained by blowing up the image of $Z_0$.
%By Section 2.3 and Lemma 1.2 of \cite{Fujiki2}, this variety is again an irreducible symplectic V-manifold. 
%It is remarkable that the moduli space $\mathcal{M}_{K'}$ of irreducible symplectic V-manifolds deformation equivalent to $K'$ will be of dimension 6 (see Proposition \ref{b}).
%However, the space of V-manifolds in $\mathcal{M}_{K'}$ coming from a partial resolution of the quotient $X/\iota$ is of dimension 5. 
%This means that $\mathcal{M}_{K'}$ contains mostly V-manifolds which are completely unknown, not related to any quotient of some smooth irreducible symplectic manifold. By the local Torelli theorem of \cite{Nanikawa}, the moduli space $\mathcal{M}_{K'}$ will be related to the Beauville--Bogomolov lattice that we provide in Theorem \ref{theorem}. 
% the moduli space of irreducible symplectic manifolds deformation equivalent to generalized Kummer fourfolds can be send into $\mathcal{M}_{K'}$ by the transformation explained above $X\rightarrow K'$.
%\begin{thm}\label{BeauvilleIntro}
%Let $X$ be an irreducible symplectic fourfold of Kummer type and $\iota$ a symplectic involution on $X$.
%Let $Z_0$ be the K3 surface which is in the fixed locus of $\iota$.
%We denote $K=X/\iota$ and $K'$ the partial resolution of singularities of $K$ obtained by blowing up the image of $Z_0$.
%Then the Beauville--Bogomolov lattice $H^2(K',\Z)$ is isomorphic to $U(3)^{3}\oplus\left(
%\begin{array}{cc}
%-5 & -4\\
%-4 & -5 
%\end{array} \right)$, and the Fujiki constant $c_{K'}$ is equal to $8$.
%\end{thm}
In \cite{Nanikawa}, Namikawa proposes a definition of the Beauville-Bogomolov form for some singular irreducible symplectic varieties. He assumes that the singularities are only $\mathbb{Q}$-factorial with a singular locus of codimension $\geq 4$. Under these assumptions, he proves a local Torelli theorem. 
This result was completed by a generalization of the Fujiki formula by Matsushita in \cite{Mat} (see also Theorem 1.2.4 of \cite{Lol} for a summaring satement). 


These results were further generalized by Kirschner for symplectic complex spaces in \cite{Tim}. 
In \cite[Theorem 2.5]{Lol2} has appeared the first concrete example of Beauville--Bogomolov lattice for a singular irreducible symplectic variety. 
The variety studied in \cite{Lol2} is a partial resolution of an irreducible symplectic manifold of $K3^{[2]}$-type quotiented by a symplectic involution. The objective of this paper is to provide a new example of a Beauville--Bogomolov lattice replacing the manifold of $K3^{[2]}$-type by a fourfold of Kummer type. 
Knowing the integral basis of the cohomology group of the generalized Kummer provided by Theorem \ref{integralbasistheorem}, this calculation becomes possible. 
Moreover, the calculation will be much simpler as in \cite{Lol2} because of the general techniques for calculating integral cohomology of quotients developed in \cite{Lol} and the new technique using monodromy developed in Lemma \ref{Ddelta}. 
The other techniques developed in \cite{Lol2} are also contained in \cite{Lol}, so to simplify the reading, we will only cite \cite{Lol2} in the rest of the section.

Concretely, let $X$ be an irreducible symplectic fourfold of Kummer type and $\iota$ a symplectic involution on $X$. Theorem \ref{SymplecticInvo} establishes that the fixed locus of $\iota$ is the union of 36 points and a K3 surface $Z_0$. Then the singular locus of $K:=X/\iota$ is the union of a K3 surface and 36 points. The singular locus is not of codimension four. We will lift to a partial resolution of singularities,
$K'$ of $K$, obtained by blowing up the image of $Z_0$. By Section 2.3 and Lemma 1.2 of \cite{Fujiki2}, the variety $K'$ is an irreducible symplectic V-manifold which has singular locus of codimension four.


%All Section \ref{BeauvilleForm} is devoted to prove the following theorem.
\begin{thm}\label{theorem}
Let $X$ be an irreducible symplectic fourfold of Kummer type and $\iota$ a symplectic involution on $X$.
Let $Z_0$ be the K3 surface which is in the fixed locus of $\iota$.
We denote $K=X/\iota$ and $K'$ the partial resolution of singularities of $K$ obtained by blowing up the image of $Z_0$.
Then the Beauville--Bogomolov lattice $H^2(K',\Z)$ is isomorphic to $U(3)^{3}\oplus\left(
\begin{array}{cc}
-5 & -4\\
-4 & -5 
\end{array} \right)$, and the Fujiki constant $c_{K'}$ is equal to $8$.
\end{thm}
We remark that it is the first example of a Beauville--Bogomolov form which is not even.

%Another illustration of Theorem \ref{thetaTheorem} is Theorem \ref{SymplecticInvo}, where the knowledge on the third integral cohomology group of a generalized Kummer fourfold $X$ (see Corollary \ref{actionH3}), allows us to end the classification of symplectic involutions on $X$ as a corollary of the lattice classification by Mongardi, Tari and Wandel in \cite{MongWanTari}.

The paper is organised as follows. In Section \ref{OddHilb2} we describe the odd integral cohomology of $A^{[2]}$ the Hilbert scheme of two points on a surface $A$ with torsion free cohomology. 
%The odd cohomology of $A^{[2]}$ is described in term of the cohomology of $A$ using techniques developed in \cite{Lol}.
Then, after recalling some notions on Nakajima operators in Section \ref{Section_Hilbert}, 
we are able to provide an integral basis of the Hilbert scheme of two points on an abelian surface in term of Nakajima operators (Proposition \ref{A2Basis}).
Section \ref{Section_GeneralKummer} studies the integral cohomology of generalized Kummer in any dimension.
In Section \ref{Middle}, we use all these preliminary results and monodromy technique developed in \cite{Hassett} to find an integral basis of the cohomology of the generalized Kummer fourfold $K_2(A)$.
%Therefore the kowledge of the integral cohomology of $K_2(A)$ is used in Section \ref{Involution} and \ref{BeauvilleForm} respectively to classified the symplectic involutions on $K_2(A)$ and to prove Theorem \ref{theorem}.
As a consequence, in Section \ref{Involution}, we are able to 
to end the classification of symplectic involutions on $K_2(A)$ as a corollary of the lattice classification by Mongardi, Tari and Wandel in \cite{MongWanTari}.
Finally, Section \ref{BeauvilleForm} is dedicated to the proof of Theorem \ref{theorem}.
~\\

\textbf{Acknowledgements.} 
We want to thank Samuel Boissi\`ere, David Chataur, Brendan Hassett, Giovanni Mongardi, Marc Nieper-Wi\ss kirchen and Ulrike Rie\ss\ for useful discussions.
We also thank Samuel Boissi\`ere, Daniel Huybrechts and Marc Nieper-Wi\ss kirchen for hospitality.
%We want to thank Samuel Boissi\`ere, Daniel Huybrechts and Marc Nieper-Wi\ss kirchen for their kind support during the development of the project.
%We also thank Brendan Hassett and Giovanni Mongardi for useful discussions.
GM is supported by Fapesp grant 2014/05733-9. SK was partially supported by a DAAD grant.
